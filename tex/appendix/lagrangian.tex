\label{appendix:lagrangian}

%by the means of the \emph{Lagrange multiplier method}\cite{Vadlamani.2020}.
% The introduction of the adjoint variable allows to reduce the computational complexity of sensitivity methods, as we will explore in this section and later in Section \ref{section:computing-adjoints}.

In this section we are going to derive the adjoint equation for both discrete and continuous methods using the Lagrange multiplier trick \cite{Vadlamani_Xiao_Yablonovitch_2020}. 
Conceptually, the method is the same in both discrete and continuous case, with the difference that we manipulate linear algebra objects for the former and continuous operators for the later. 

For the continuous adjoint method, we proceed the same way by writing a new loss function $I(\theta)$, sometimes known as the \textit{Lagrangian}, identical to $L(\theta)$ as 
\begin{equation}
    I(\theta) = L(\theta) - \inttime \lambda(t)^T \left( \frac{du}{dt} - f(u, \theta, t) \right) dt
\end{equation}
where $\lambda(t) \in \mathbb R^n$ is the Lagrange multiplier of the continuous constraint defined by the differential equation. Now, 
\begin{equation}
    \frac{dL}{d\theta} = \frac{dI}{d\theta} = 
    \inttime \left( \frac{\partial h}{\partial \theta} + \frac{\partial h}{\partial u} \frac{\partial u}{\partial \theta} \right) dt
    - 
    \inttime \lambda(t)^T \left( \frac{d}{dt} \frac{du}{d\theta} - \frac{\partial f}{\partial u} \frac{du}{d\theta} - \frac{\partial f}{\partial \theta} \right) dt.
\end{equation}
Notice that the term involved in the second integral is the same we found when deriving the sensitivity equations. 
We can derive an easier expression for the last term using integration by parts. 
Using our usual definition of the sensitivity $s = \frac{du}{d\theta}$, and performing integration by parts in the term $\lambda^T \frac{d}{dt} \frac{du}{d\theta}$ we derive 
\begin{multline}
    \frac{dL}{d\theta}
    = 
    \inttime \left( \frac{\partial h}{\partial \theta} + \lambda^T \frac{\partial f}{\partial \theta} \right) dt 
    - 
    \inttime \left( - \frac{d\lambda^T}{dt} - \lambda^T \frac{\partial f}{\partial u} - \frac{\partial h}{\partial u} \right) \, s(t) \, dt \\
    -
    \bigg ( \lambda(t_1)^T s(t_1) - \lambda(t_0)^T s(t_0) \bigg ).
    \label{eq:continous-adjoint-loss}
\end{multline}
Now, we can force some of the terms in the last equation to be zero by solving the following adjoint differential equation for $\lambda(t)^T$ in backwards mode
\begin{equation}
    \frac{d\lambda}{d\theta} = - \left(\frac{\partial f}{\partial u}\right)^T \lambda - \left( \frac{\partial h}{\partial u} \right)^T,
    \label{eq:continuous-adjoint}
\end{equation}
with final condition $\lambda(t_1) = 0$. 

It is easy to see that this derivation is equivalent to solving the Karush-Kuhn-Tucker (KKT) conditions. 