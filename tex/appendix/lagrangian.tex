\label{appendix:lagrangian}

%by the means of the \emph{Lagrange multiplier method}\cite{Vadlamani.2020}.
% The introduction of the adjoint variable allows to reduce the computational complexity of sensitivity methods, as we will explore in this section and later in Section \ref{section:computing-adjoints}.

The adjoint equation can be derived directly from the 
Following the analysis in \cite{Giles_Pierce_2000}, we decided to present both approaches here, although we prefer with the duality viewpoint introduced in the main text since we believe is more commonly used and easy to understand for newcomers. 

In this section we are going to derive the adjoint equation for both discrete and continuous methods using the Lagrangian formulation of the adjoint. 
It is important to mention that this is different than using the Lagrange multipliers approach, a common confusion in the literature\cite{Givoli_2021}.
Conceptually, the method is the same in both discrete and continuous case, with the difference that we manipulate linear algebra objects for the former and continuous operators for the later. 

\subsubsection{Discrete adjoint}

Following the notation introduced in Section \ref{section:discrete-adjoint}, we first define the Lagrangian function $I(U, \theta)$ as 
\begin{equation}
    I(U, \theta) = L(u; \theta) + \lambda^T G(U; \theta),
\end{equation}
where $\lambda \in \R^{nk}$ is, in principle, any choice of vector. 
Given that for every choice of the parameter $\theta$ we have $G(U; \theta) = 0$, we have that $I(U, \theta) = L(U, \theta)$.
Then, the gradient of $L$ with respect to the vector parameter $\theta$ can be computed as 
\begin{align}
    \frac{dL}{d\theta}
    = 
    \frac{dI}{d\theta}
    &= 
    \frac{\partial L}{\partial \theta} + \frac{\partial L}{\partial U} \frac{\partial U}{\partial \theta}
    + 
    \lambda^T
    \left( \frac{\partial G}{\partial U} + \frac{\partial G}{\partial U} \frac{\partial U}{\partial \theta} \right) \nonumber
    \\ 
    &= 
    \frac{\partial L}{\partial \theta} + \lambda^T \frac{\partial G}{\partial U} 
    + 
    \left( \frac{\partial L}{\partial \theta} + \lambda^T \frac{\partial G}{\partial U} \right) \frac{\partial U}{\partial \theta}. \label{eq:appendix-discrete-adjoint}
\end{align}
The important trick in the last term involved grouping all the terms involving the sensitivity $\frac{\partial U}{\partial \theta}$ together. 
In order to avoid the computation of the sensitivity at all, we can define $\lambda$ as the vector that makes the last term in the right hand side of Equation \eqref{eq:appendix-discrete-adjoint} which results in the same results we obtained in Equation \eqref{eq:adjoint-state-equation} and \eqref{eq:def_adjoint}, that is, 
\begin{equation}
    \frac{\partial L}{\partial \theta} = - \lambda^T \frac{\partial G}{\partial U}.
\end{equation}
Finally, the gradient can be computed as 
\begin{equation}
    \frac{dL}{d\theta} 
    = 
    \frac{\partial L}{\partial \theta} + \lambda^T \frac{\partial G}{\partial U} .
\end{equation}

\subsubsection{Continuous adjoint}

For the continuous adjoint method, we proceed the same way by defining the Lagrangian $I(\theta)$ as 
\begin{equation}
    I(\theta) = L(\theta) - \inttime \lambda(t)^T \left( \frac{du}{dt} - f(u, \theta, t) \right) dt
\end{equation}
where $\lambda(t) \in \mathbb R^n$ is any function. 
It is important to mention that, in principle, there is connection yet between $\lambda (t)$ and the Lagrange multiplier associated to the constraint \cite{Givoli_2021}. 
Instead, the condition 
\begin{equation}
    \inttime \lambda(t)^T \left( \frac{du}{dt} - f(u, \theta, t) \right) dt = 0 \qquad \text{for all function } \lambda : [t_0, t_1] \mapsto \R^n
\end{equation}
correspond to the weak formulation of the differential equation \eqref{eq:original_ODE} \cite{brezis2011functional}.
As long as the differential equation is satisfyed, we have $I(\theta) = L(\theta)$ for all choices of the vector parameter $\theta$. 
Now, 
\begin{equation}
    \frac{dL}{d\theta} = \frac{dI}{d\theta} = 
    \inttime \left( \frac{\partial h}{\partial \theta} + \frac{\partial h}{\partial u} \frac{\partial u}{\partial \theta} \right) dt
    - 
    \inttime \lambda(t)^T \left( \frac{d}{dt} \frac{du}{d\theta} - \frac{\partial f}{\partial u} \frac{du}{d\theta} - \frac{\partial f}{\partial \theta} \right) dt.
\end{equation}
Notice that the term involved in the second integral is the same we found when deriving the forward sensitivity equations. 
We can derive an easier expression for the last term using integration by parts. 
Using our usual definition of the sensitivity $s = \frac{du}{d\theta}$, and performing integration by parts in the term $\lambda^T \frac{d}{dt} \frac{du}{d\theta}$ we derive 
\begin{multline}
    \frac{dL}{d\theta}
    = 
    \inttime \left( \frac{\partial h}{\partial \theta} + \lambda^T \frac{\partial f}{\partial \theta} \right) dt 
    - 
    \inttime \left( - \frac{d\lambda^T}{dt} - \lambda^T \frac{\partial f}{\partial u} - \frac{\partial h}{\partial u} \right) \, s(t) \, dt \\
    -
    \bigg ( \lambda(t_1)^T s(t_1) - \lambda(t_0)^T s(t_0) \bigg ).
    \label{eq:continous-adjoint-loss}
\end{multline}
Now, we can force some of the terms in the last equation to be zero by solving the following adjoint differential equation for $\lambda(t)^T$ in backwards mode
\begin{equation}
    \frac{d\lambda}{d\theta} = - \left(\frac{\partial f}{\partial u}\right)^T \lambda - \left( \frac{\partial h}{\partial u} \right)^T,
    \label{eq:continuous-adjoint}
\end{equation}
with final condition $\lambda(t_1) = 0$. 

It is easy to see that this derivation is equivalent to solving the Karush-Kuhn-Tucker (KKT) conditions. 

% \subsubsection{The adjoint from a functional analysis perspective}