\documentclass[12pt]{article}

\setlength{\oddsidemargin}{0in}
\setlength{\textwidth}{6.5in}
\setlength{\topmargin}{-0.5in}
\setlength{\textheight}{9in}

%%%%%%%%%%%%%%%%%%%%%%%%%%%%%%%%%%%
%%%%%%%%% General packages %%%%%%%%
%%%%%%%%%%%%%%%%%%%%%%%%%%%%%%%%%%%

\usepackage{graphicx} % Include figure files
\usepackage{caption}
\usepackage{color} % Include colors for document elements
\usepackage{dcolumn} % Align table columns on decimal point
\usepackage{float}
\usepackage[hidelinks]{hyperref} % hidelinks added to not display red box for links
\usepackage{algorithm}
\usepackage{enumitem} % Necesarry for enumerating with romans (i), (ii), ...
\usepackage{todonotes} % Taking notes

%%%%%%%%%%%%%%%%%%%%%%%%%%%%%%%%%%%
%%%%%% Customized packages %%%%%%%%
%%%%%%%%%%%%%%%%%%%%%%%%%%%%%%%%%%%

% Authors information
\usepackage{contributors}

% Julia code
\usepackage{jlcode}

% Math 
\usepackage{mymath}

% Bibliography %
\usepackage{mybib}
% All the bib files are centralized here, but feel free to 
% add a new custom bib file inside ./bibs/
\addbibresource{bibliography.bib}
% Alternative bib source
% \addbibresource{bibs/bibliography-heimbach.bib}
% \addbibresource{bibs/bibliography-facu.bib}

%%%%%%%%%%%%%%%%%%%%%%%%%%%%%%%%%%%
%%%%%%%%%%%%% Document %%%%%%%%%%%%
%%%%%%%%%%%%%%%%%%%%%%%%%%%%%%%%%%%

\title{A Review of Sensitivity Methods \\ for Differential Equations}

\date{\today}

\begin{document}
\maketitle

\hfill \break
\thanks
\newpage

\begin{abstract}
{\footnotesize
The differentiable programming paradigm is a central component of modern scientific computing. 
A long tradition of this paradigm exists in the context of inverse methods, in particular in differential equation-constrained, gradient-based optimization.
Even more recently, there has been an increasing interest in hybrid models combining differential equations with machine learning. 
The recognition of strong synergies between inverse methods and machine learning offers the opportunity to lay out a coherent framework applicable to both fields.
However, for models described by differential equations, the calculation of the associated gradient loss requires to differentiate the numerical solution of the differential equation. 
This task is non-trivial and requires careful algebraic and numerical manipulations and computational implementations.
Here, we provide a comprehensive review of existing techniques to compute derivatives of numerical solutions of differential equations.
We first discuss the importance of gradients of solutions of differential equations in a variety of scientific domains.
% , covering computational fluid dynamics, electromagnetism, geosciences, meteorology, oceanograpgy, climate science, flux inversion, glaciology, solid earth geophysics, biology and ecology, and quantum physics.
Second, we lay out the mathematical foundations of the various approaches and compare them with each other. 
Third, we cover the computational considerations and explore the solutions available in modern scientific software.
We finalize our discussion with a series of recommendations for practitioners. 
% By delivering an exhaustive review of sensitivity methods, 
We hope that this work accelerates the fusion of scientific models and data, and fosters a modern approach to scientific modelling.
\\ \\
\noindent \textbf{Key words.} differentiable programming, sensitivity methods, differential equations, inverse modelling, scientific machine learning, automatic differentiation, adjoint method.
}

\end{abstract}

\vspace{250px}
\begin{quote}
    % \centering
    % \textbf{To the community, by the community.}
    \textit{This manuscript was conceived with the goal of shortening the gap between developers and practitioners of differentiable programming applied to modern scientific machine learning. 
    With the advent of new tools and new software, it is important to create pedagogical content that allows the broader community to understand and integrate these methods into their workflows. 
    We hope this encourages new people to be an active part of the ecosystem, by using and developing open-source tools. 
    This work was done under the premise \textbf{open-science from scratch}, meaning all the contents of this work, both code and text, have been in the open from the beginning and that any interested person can contribute to the project. 
    % You can contribute directly to the GitHub repository \url{github.com/ODINN-SciML/DiffEqSensitivity-Review}.
    }
\end{quote}

\normalsize

\clearpage

\tableofcontents
\clearpage

\section*{Plain language summary}
Scientific models are used to predict and understand a vast array of different dynamics, ranging from physical processes, ecological, biological, and social interactions or chemical reactions, among many. 
The combination of mechanistic models with data-driven models is becoming increasingly common in many scientific domains. 
In order to achieve so, these models need to leverage both domain knowledge and data, in order to have an accurate representation of the underlying dynamics. 
Being able to determine which model parameters are most influential and further compute derivatives of such a model is key to correctly assimilating and learning from data, but a myriad of sensitivity methods exist to do so. 
We provide an overview of the different sensitivity methods that exist, providing (i) guidelines on the best use cases for different scientific domain problems, (ii) detailed mathematical analyses of their characteristics, and (iii) computational implementations on how to solve them efficiently. 


\section{Introduction}
% Seems like the introduction is actually the scientific motivation
In this section we provide a general overview of different methodologies for computing the gradient/derivative of a loss function that includes the solution of differential equations. 
Without aiming at making an extensive and specialized review on the field, we find this resource useful for other researchers and students working on problems that combine optimization and sensitivity analysis with differential equations. 

Consider a system of ordinary differential equations given by
\begin{equation}
 \frac{du}{dt} = f(u, \theta, t),
 \label{eq:original_ODE}
\end{equation}
where $u \in \mathbb{R}^n$, $\theta \in \mathbb R^p$, and initial condition $u(t_0) = u_0$. 
Here $n$ denotes the total number of ordinary differential equations and $p$ the size of a parameter embedded in the functional form of the differential equation. 
Although we consider here the case of ordinary differential equations, that is, when the derivatives are just with respect to the time variable $t$, we will later include the case of partial differential equations (PDE).
We are interested in computing the gradient of a given loss function $L(u(\cdot, \theta))$ with respect to the parameter $\theta$. 
Examples of loss functions include
\begin{equation}
 L(u(\cdot, \theta)) = \| u(t_1, \theta) - u_1 \|_2^2,
\end{equation}
where $u_1$ is the desired target observation at some later time $t_1$; and
\begin{equation}
 L(u(\cdot, \theta)) = \int_{t_0}^{t_1} h( u(t;\theta), \theta) ) dt, 
\end{equation}
with $h$ some function that quantifies the contribution of the error term at time $t \in [t_0, t_1]$.
We are interested in computing the gradient of the loss function with respect to the parameter $\theta$, which can be written as
\begin{equation}
 \frac{dL}{d\theta} = \frac{dL}{du} \frac{du}{d\theta}.
 \label{eq:dLdtheta_VJP}
\end{equation} 
The first term is usually easy to evaluate, since it just involves the partial derivative of the scalar loss function to the solution. 
The second term on the right hand side is the actual bottleneck and it is usually referred to as the sensitivity,
\begin{equation}
 s = \frac{\partial u}{\partial \theta} \in \mathbb R^{n \times p}.
\end{equation}

\begin{figure}[]
    \centering
    \includegraphics[width=0.80\textwidth]{figures/scheme-methods.png}
    \caption{Schematic representation of the different methods available for differentiation involving differential equation solutions.}
    \label{fig:diff}
\end{figure}
Depending on the number of parameters and the complexity of the differential equation we are trying to solve, there are different methods to compute gradients with different numerical and computational advantages and that also scale differently depending of the number of differential equations $n$ and number of parameters $p$. 
These methods can be roughly classified as:
\begin{itemize}
    \item \textit{Discrete} vs \textit{continuous} methods.
    \item \textit{Forward} vs \textit{backwards} methods.
\end{itemize}
The first difference regards the fact that the method for computing the gradient can be either based on the manipulation of atomic operations that are easy to differentiate using the chain rule several times (discrete), in opposition to the approach of approximating the gradient as the numerical solution of a new set of differential equations (continuous). 
The second distinction is related to the fact that some methods compute gradients by resolving a new sequential problem that may move in the same direction of the original numerical solver - i.e. moving forward in time - or, instead, they solve a new system that goes backwards in time. 
Figure \ref{fig:diff} displays a classification of some methods under this two-fold classification. In the following section we are going to explore more in detail these methods.

\section{Inverse modelling in science}
\subsection{On the importance of differentiable programming}

Scientific models from many domains have often been based on mechanistic models, represented as differential equations, involving the use of numerical methods to solve them. 
Among many, this lead to fundamental advances in the physical sciences during the last century, with the combination of complex mathematical theories and a reduced amount of observations to validate them
(e.g., \cite{Wigner.1960, Rude:2018jv}). 
Nonetheless, in the 21st century, with the unstoppable wave of data flooding all scientific domains, progress with such traditional methods has become more complex. 

Alternatively, the field of statistics experienced a boom following the massive growth of data, signaling the era of data science and machine learning \cite{Cox:2017hv}.
With the advent of machine learning methods, it is possible to learn and capture extremely complex nonlinear patterns and information hidden in huge datasets. 
Machine learning models can be seen as the opposite of mechanistic models: they are flexible, data-driven and they do not necessarily respect domain-specific constraints.

At first sight, these two modelling philosophies can be seen as antagonistic, and this is more or less the way they have evolved in the last decades \cite{zdeborova_understanding_2020}. 
On the one hand, domain scientists have often been sceptical of adopting machine learning methods, judging them as opaque black boxes, unreliable, and not respecting domain-established knowledge \cite{Coveney:2016eb}.
On the other hand, the field of machine learning has mainly been developed around data-driven applications, without including any \textit{a priori} physical knowledge. 
However, there has been an increasing interest in making mechanistic models more flexible, as well as introducing domain-specific or physical constraints and interpretability in machine learning models (e.g., \cite{Molnar.2020sisk,Rudin.2022}).
% If both modelling approaches have different strengths, why not combine them and attempt to have the best of both worlds?

A key way to achieve this is through differentiable programming, i.e. being able to compute derivatives of any computer program describing a scientific model.
During the last decades, the backpropagation algorithm has enabled the fast-growing of deep learning by efficiently computing gradients of large and complex neural networks with many parameters \cite{griewank2012invented}.
Nowadays, the differentiation of hybrid models comprising data-driven models (e.g. neural networks, gaussian processes) with differential equations poses complex technical problems, which are only starting to be explored in recent years \cite{ma_comparison_2021}. 
Being able to accurately estimate model parameters, ranging from a few ones in classic inversion problems to millions of them in large neural networks, opens many new possibilities. 
Differentiable programming has the potential to revolutionize the way we approach and design scientific models and even the way we discover governing laws from data. 

\subsection{Domain-specific applications}

Differential equations can be used to describe a large variety of dynamical systems, while data-driven regression models (e.g., neural networks, Gaussian processes, reduced-order models, basis expansions) have been demonstrated to act as universal approximators, virtually learning any possible function if enough data is available \cite{gorban_1998}. 
This combined flexibility can be exploited by many different domain-specific problems to tailor modelling needs to both dynamics and data characteristics.

\subsubsection{Computational Fluid Dynamics}

\subsubsection{Electromagnetism}

\subsubsection{Geosciences}

Many geoscientific phenomena are governed by global and local conservation laws (conservation of mass, momentum, energy, tracers, etc.) along with a set of empirical constitutive laws and subgrid-scale parameterization schemes. Together, they enable efficient description of the system's spatio-temporal evolution in terms of a set of partial differential equations (PDEs).
Example are geophysical fluid dynamics \cite{Vallis:2016kv}, describing geophysical properties of many Earth systems, such as the atmosphere, oceans, or glaciers.
In such models, calibrating model parameters is extremely challenging, due to datasets being sparse in both space and time, heterogeneous, and noisy.
Moreover, many existing mechanistic models can only partially describe observations, with many detailed physical processes being ignored or poorly parameterized. 
The use of differentiable programming, combining PDEs and data-driven models (i.e. Universal Differential Equations) may add flexibility to mechanistic models in order to incorporate new governing laws from data (from either measurement or simulations) \cite{rackauckas2020universal}.

Arguably, the notion of ``differentiable programming'' has a long tradition in the geosciences in the context of solving large-scale geophysical inverse problems.
The overarching goal of such problems is to find a set of optimal model parameters that minimize a (usually weighted least-squares) objective or cost function quantifying the misfit between (usually sparse) observations and the simulated state, subject to the constraint that the model equations be fulfilled. The constrained optimization problem is transformed into an unconstrained problem by way of \emph{Lagrange multiplier method}, also referred to as the \emph{adjoint method}. 
The corresponding \textit{adjoint model} computes the gradient of the objective function with respect to all inputs. Gradient-based nonlinear optimization then enables us to
``invert'' for optimal values of the unknown or uncertain inputs.
Depending on the nature of the inputs, we may distinguish the following cases:
%
\begin{itemize}
\item \emph{Initial conditions:} Inverting for uncertain initial conditions, which, when integrated using the model, lead to an optimal match of the observations; variants thereof are used for optimal forecasting (see below);
\item \emph{Boundary conditions:} Inverting for uncertain surface, bottom, or lateral boundaries (e.g., open boundaries of a limited domain), which, when used in the model, produce an optimal match of the observations; variants thereof are used in tracer or boundary (air-sea) flux inversion problems, e.g., related to the global carbon cycle;
\item \emph{Model parameters:} Inverting for uncertain model parameters amounts to an optimal model calibration problem. As a ''learning of optimal parameters from data'' problem, it is the closest to machine learning applications.
\end{itemize}
%
In addition to the use of gradients or derivative information for optimization, inversion, estimation, or ``learning'', gradients have also proven powerful tools for 
\begin{itemize}
\item
computing \emph{comprehensive sensitivities} of quantities of interest,
\item 
computing \emph{optimal perturbations} (in initial or boundary conditions) that lead to maximum amplification of specific norms of interest,
\item
characterizing and quantifying uncertainties by way of second derivative (Hessian) information.
\end{itemize}
%
Within the framework of gradient-based inversion, all of these cases rely on the availability of an adjoint model of the (in general nonlinear) geophysical parent model to efficiently compute the gradient of the objective function with respect to a usually very high-dimensional (typically $O(10^3) - O(10^8)$) space of inputs.
In the following, we sketch how differentiable programming - from the perspective of adjoint modeling - has been used in different disciplines of geosciences, and how new concepts are emerging of combining inverse modeling and machine learning approaches where differentiable programming provides a key computational enabling framework. (Note that some authors have used the notion of ``scientific machine learning'' to capture some aspects of the latter approach [REFS]).

% Add oceanography example

\paragraph{Meteorology}
...

\paragraph{Oceanography}
...

\paragraph{Climate science}
...

\paragraph{Flux inversion}
...

\paragraph{Glaciology}
Glaciers act as slow fluids, flowing down-slope through the effects of gravity, and the understanding of their rheological properties (e.g. ice viscosity affecting internal deformation or sliding at the glacier-bedrock interface) is key to assessing their contribution to water resources and sea-level rise \cite{cuffey_physics_2010}. 
These rheological processes and their dependency on key large-scale environmental variables, such as the local climate or topography, are still not well understood.
The use of differentiable programming, combined with Universal Differential Equations, holds great potential to learn new empirical laws of these physical processes from large-scale remote sensing datasets. 
A recent study showed how Julia's differentiable programming capabilities can be used to optimize the parameters of a neural network, learning a function of the nonlinear ice diffusivity in a glacier ice flow PDE, to match observations \cite{bolibar_universal_2023}.

\paragraph{Solid Earth geophysics}
...

\subsubsection{Biology and Ecology}


\subsubsection{Quantum Physics}

\section{Statistical foundations of inverse modelling}
% here I would move the subsection "Preliminaries" from the next section.
% this is an alternative variant of the "preliminaries" section, which attempts to formalize the inverse modelling problem within a statistical framework. 
\newcommand{\M}{\mathcal{M}}
\newcommand{\by}{\textbf{y}}
\def\E{\mathbb{E}}

\subsection{Models and data}
System of ordinary differential equations (ODEs) can generally be described as
\begin{equation}\label{eq:prob_statement}
    \begin{split}
        &\dot{x}(t) = f(t, x(t), p)\\
        &x(0) = x_0\\
        &y(t) = h(x(t)) + \epsilon(t)\\
    \end{split}
\end{equation}
%
where $x(t) \in \R^m $ is a vector of state variables that might represent XXX, $y(t) \in \R^d$ is a vector of observables that contains a subset or aggregates of the state variables, and $p \in \R^q$ is the model parameter vector.
% 
$h$ is a function that maps the state variables to the observables, which may be contaminated with noise $\epsilon$, here of Gaussian type, with zero mean and variance--covariance matrix $\Sigma_y$.
%
Denoting by $\theta = (x_0, p)$ the vector containing the ICs and the parameters, the model may be viewed as a map $\M$ parametrized by time $t$ that takes the parameters $\theta$ to the state variables $x$
\begin{equation}
\begin{split}
        \M(t,\theta) &= x(t) \\
            &= \int_0^t f(s, x(s), p) ds + x_0
\end{split}
\end{equation}

\subsection{Maximum likelihood estimation and loss function}
% The following text could be mixed with the "empirical loss functions" bullet point of the "preliminiary", so that we show the connection between maximum likelihood estimation and the minimization of the RSS.

Taking expectations over the noise realizations yields $ \E [y(t)] = h(\M(t,\theta))$, and it follows that the conditional likelihood of each observation $y_k \equiv y(t_k)$, given the parameters $\theta$ and the model $\M$ denoted by $p( y_k | \theta, \M )$, follows the distribution of the residuals $\epsilon_k \equiv \epsilon(t_k) = y(t_k) - h\left(\M(t_k, \theta)\right)$, which corresponds to the multivariate normal distribution $\mathcal{N}_{0, \Sigma_y}$.
%
Following a Bayesian approach, the calibration of the model can be performed on the basis of the parameter and model posterior probability $p(\theta, \M | \by_{1:K})$, i.e. the conditional probability density of the parameter values $\theta$ and the model $\M$ given the data, given by 
\begin{equation}
    p(\theta, \M | \by_{1:K}) \propto p(\by_{1:K} | \theta, \M) p(\theta, \M)
\end{equation}
where $\by_{1:K} = (y_1,\dots,y_K)$, $p(\by_{1:K} | \theta, \M)$ is the product of the conditional likelihood of each observation $y_k$
\begin{equation}\label{eq:likelihood-std}
\begin{split}
    p(\by_{1:K} | \theta, \M) &= \prod_{i=1}^K p(y_{i} | \theta, \M)\\
                        &= \prod_{k=1}^K \frac{1}{\sqrt{(2\pi)^d|\Sigma_y|}} \exp \left(-\frac{1}{2} \epsilon_k^{T} \Sigma_y^{-1} \epsilon_k \right)
\end{split}
\end{equation}
and $p(\theta,\M)$ is the prior distribution of the model and its associated parameter values. The model $\M$ is included in the probabilistic quantities in order to accommodate multiple candidate models.

A variational method to obtain a Bayesian estimate of $\theta$ involves maximizing $p(\theta, \M | \by_{1:k})$ to obtain the maximum a posteriori (MAP) estimator \cite{Bocquet2019}, which is equivalent to a maximum likelihood approach under a uniform prior distribution of the parameters, i.e. when no prior information on the parameter values is used \cite{Schartau2017}. Observing that maximizing $p(\theta, \M | \by_{1:K})$ is equivalent to minimizing $- \log p(\theta | \by_{1:K}, \M)$ and assuming a normal prior distribution of the parameters $\mathcal{N}_{p_b, \Sigma_p}$, one can obtain the MAP $\hat{\theta}$
\begin{equation}\label{eq:argmin_xi}
    \hat{\theta} = \argmin{\theta} L_{\M}(\theta)
\end{equation}
%
where $L_{\M}$ is referred to as the loss function and defined as
\begin{equation}\label{eq:loss_fn}
    \begin{split}
    L_{\M}(\theta) &= \frac{1}{2} \left[ \sum_{k=1}^{K-1} \|y_k - h\left(\M(t_k,\theta)\right)\|_{\Sigma_y}^2 + \|p - p_b\|_{\Sigma_p}^2 \right] \\
   \end{split}
\end{equation}
\cite{Schneider2017,Raue2009} and where we use the notation $\|y\|_\Sigma^2 = y \Sigma^{-1}y^{\textbf{T}}$.
%
Eq. \ref{eq:loss_fn} is similar to a traditional least squares function commonly used in regression, where the second summand is the analogue of a regularization term for the weights and biases of e.g. a neural network.

%
Gradient-based optimizers can then be used to efficiently obtain $\hat{\theta}$ in Eq. \ref{eq:argmin_xi}, iteratively updating the parameter vector $\theta_m$ given the gradient of the loss function, denoted by $\nabla_\theta L_{\M}$, to navigate the surface defined by $L_\M$ with the aim to find the global minimum where $\nabla_\theta L_{\M}(\hat{\theta}) = 0$. 
% 
As an example, the plain vanilla gradient descent algorithm is given by
\begin{equation}\label{eq:SDG}
    \theta_{m+1} = \theta_{m} - \gamma \nabla_\theta L_\M({\theta_m})
\end{equation}
where $\gamma$ is the learning rate. Other gradient-based algorithms, such as the ADAM optimizer used in the section below, employ more advanced updating strategies to avoid convergence to local minima but stay in the spirit of Eq. \ref{eq:SDG}.


\subsection{Likelihood profiles}

\subsection{Quantity of interest}

\subsection{Diagnosis of the solution}

\section{Methods}
\label{section:methods}
\begin{figure}[]
    \centering
    \includegraphics[width=0.80\textwidth]{figures/scheme-methods.pdf}
    \caption{Schematic representation of the different methods available for differentiation involving differential equation solutions. These can be classified depending if they find the gradient by solving a new system of differential equations (\textit{continuous}) or if instead they manipulate unit algebraic operations (\textit{discrete}). Furthermore, depending if these methods run in the same direction than the numerical solver, we are going to be talking about \textit{backward} and \textit{forward} methods.}
    \label{fig:diff}
\end{figure}
Depending on the number of parameters and the complexity of the differential equation we are trying to solve, there are different methods to compute gradients with different numerical and computational advantages.
These methods can be roughly classified as:
\begin{itemize}
    \item \textit{Discrete} vs \textit{continuous} methods
    \item \textit{Forward} vs \textit{backwards} methods
\end{itemize}
The first difference regards the fact that the method for computing the gradient can be either based on the manipulation of atomic operations that are easy to differentiate using the chain rule several times (discrete), in opposition to the approach of approximating the gradient as the numerical solution of a new set of differential equations (continuous).
Another way of conceptualizing this difference is by comparing them with the discretize-optimize and optimize-discretize approaches \cite{bradley2013pde, Onken_Ruthotto_2020}.   
We can either discretize the original system of ODEs in order to numerically solve it and then define the set of adjoint equations on top of the numerical scheme; or instead define the adjoint equation directly using the differential equation and then discretize both in order to solve \cite{Giles_Pierce_2000}.

The second distinction is related to the fact that some methods compute gradients by resolving a new sequential problem that may move in the same direction of the original numerical solver - i.e. moving forward in time - or, instead, they solve a new system that goes backwards in time. 
Figure \ref{fig:diff} displays a classification of some methods under this two-fold classification. In the following section we are going to explore more in detail these methods.

% It is important to remark that all the 

\subsection{Preliminaries}
Consider a system of first-order ODEs given by
\begin{equation}
 \frac{du}{dt} = f(u, \theta, t)
 \label{eq:original_ODE}
\end{equation}
subject to the initial condition $u(t_0) = u_0$, where $u \in \mathbb{R}^n$ is the unknown solution vector of the ODEs, $f: \R^n \times \R^p \times \R \mapsto \R^n$ is a function that depends on the state $u$, $\theta \in \mathbb R^p$ is a vector parameter, and $t$ refers to time.
Here, $n$ denotes the total number of ODEs and $p$ the number of parameters of the differential equation.
% Although we here consider the case of ODEs, that is, when the derivatives are just with respect to the time variable $t$, the ideas presented here can be extended to partial differential equations (PDEs) (when discretized via e.g. the method of lines \cite{ascher2008numerical}), and to differential algebraic equations (DAEs) \cite{hairer-solving-2}.
% In fact, PDEs play an essential role when formulating equations of motion via local conservation (and constitutive) laws in physics-based simulations.
% Furthermore, the fact that both $u$ and $\theta$ are one-dimensional vectors does not prevent the use of higher-dimension objects (e.g. when $u$ is a matrix or a tensor). 
Except for a minority of functions $f(u,\theta, t)$, solutions to Equation \eqref{eq:original_ODE} need to be computed using numerical solvers. 

\subsubsection{Numerical solvers for ordinary differential equations}
\label{section:intro-numerical-solvers}
% Work in progress due to refactoring

Numerical solvers for the solution of ODEs or IVPs can be classified as one-step methods, among which Runge-Kutta methods are the most widely used, and multi-step methods \cite{hairer-solving-1}.
% With a long historical record in numerical analysis, Runge-Kutta methods generalize quadrature rules to iteratively solve for the time discretization $u^{n} \approx u(t_n)$. 
Given an integer $s$, $s$-stage Runge-Kutta methods are defined by generalizing numerical integration quadrature rules as follows
\begin{align}
\begin{split}
    u^{n+1} 
    &= 
    u^n 
    + 
    \Delta t_n \sum_{i=1}^s b_i k_i \\
    k_i 
    &= 
    f \left(u^n + \sum_{j=1}^s a_{ij} k_j , \, \theta , \, t_n + c_i \Delta t_n \right) \qquad i=1,2, \ldots, s.
    \label{eq:Runge-Kutta-scheme}
\end{split}
\end{align}
where $u^{n} \approx u(t_n)$ approximates the solution at time $t_n$, $\Delta t_n = t_{n+1}-t_n$, and $a_{ij}$, $b_i$, and $c_j$ are scalar coefficients with $i,j=1, 2,\ldots, j$, usually represented in the form of a tableau. 
A Runge-Kutta method is called explicit if $a_{ij}=0$ for $i \leq j$; diagonally implicit if $a_{ij}=0$ for $i < j$; and fully implicit otherwise. 
Different choices of the number of stages $s$ and coefficients give different orders of convergence of the numerical scheme \cite{Butcher_Wanner_1996, Butcher_2001}. 

In contrast, multi-step linear solvers are of the form 
\begin{equation}
    \sum_{i=0}^{k_1} \alpha_{ni} u^{n-i} 
    =
    \Delta t_n \sum_{j=0}^{k_2} \beta_{nj} f(u^{n-j}, \theta, t_{n-j})
\end{equation}
where $\alpha_{ni}$ and $\beta_{nj}$ are numerical coefficients \cite{hairer-solving-1}.
In most cases, including the Adams methods and backwards differentiation formulas (BDF), we have the coefficients $\alpha_{ni} = \alpha_i$ and $\beta_{nj}=\beta_j$, meaning that the coefficient do not depend on the iteration. 
Notice that multi-step linear methods are linear in the function $f$, which is not the case in Runge-Kutta methods with intermediate evaluations \cite{ascher2008numerical}.
Explicit methods are characterized by $\beta_{n, 0} = 0$ and are easy to solve by direct iterative updates. 
For implicit methods, the usually non-linear equation 
\begin{equation}
    g_i(u_i; \theta) = u_i - h \beta_{n0} f(u_i, \theta, t_i) - \alpha_i = 0,
    \label{eq:solver-constriant-example}
\end{equation}
with $\alpha_i$ a computed coefficient that includes the information of all past iterations, can be solved using predictor-corrector methods \cite{hairer-solving-1} or iteratively using Newton's method and preconditioned Krylov solvers at each nonlinear iteration \cite{SUNDIALS-hindmarsh2005sundials}.  
% \begin{equation}
%     u_i^{(j+1)} 
%     = 
%     u_i^{(j)} - \left( \frac{\partial g_i}{\partial u_i} (u_i^{(j)}; \theta) \right)^{-1} g(u_i^{(j)}; \theta).
%     \label{eq:newton-method}
% \end{equation}

% Stiffness
When choosing a numerical solver for differential equations, one crucial factor to consider is the stiffness of the equation.
Stiffness encompasses various definitions, reflecting its historical development and different types of instabilities \cite{Dahlquist_1985}.
Two definitions are noteworthy
\begin{enumerate}
    \item[$ \blacktriangleright$] Stiff equations are equations for which explicit methods do not work and implicit methods work better \cite{hairer-solving-2}.
    \item[$ \blacktriangleright$] Stiff differential equations are characterized by dynamics with different time scales \cite{hairer-solving-2, kim_stiff_2021}, also characterized by the phenomena of increasing oscillations \cite{Dahlquist_1985}.
\end{enumerate} 
Stability properties can be achieved by different means, for example by the use of implicit methods or stabilized explicit methods, such as Runge–Kutta–Chebyshev \cite{van1980internal, hairer-solving-2}. 
When using explicit methods, smaller timesteps may be required to guarantee stability. 

% Timestep controller
% Another important consideration is the choice of the time-steps $\Delta t_i$ in a numerical solver \cite{hairer-solving-1}. 
% Modern solvers include stepsize controllers that pick $\Delta t_i$ as large as possible to minimize the total number of steps while preventing large errors in the numerical solution controlled by adjustable relative and absolute tolerances (see Section \ref{section:AD-incorrect}). 

Numerical solvers usually estimate internally a scaled error computed as 
\begin{equation}
    \text{Err}_\text{scaled}^{n+1}
    =
    \sqrt{
    \frac{1}{n} \sum_{i=1}^n \left( \frac{\text{err}_i^{n+1}}{\mathfrak{abstol} + \mathfrak{reltol} \, \times \, M} \right)^2 },
    \label{eq:internal-norm-wrong}
\end{equation}
with $\mathfrak{abstol}$ and $\mathfrak{reltol}$ the adjustable absolute and relative solver tolerances, respectively, $M$ is the maximum expected value of the numerical solution, and $\text{err}_i^{n+1}$ is an estimation of the numerical error at step $n+1$ \cite{hairer-solving-1, Rackauckas_Nie_2016}. 
Common choices for these include $M = \max (u_i^{n+1}, \hat u_i^{n+1})$ and $\text{err}_i^{n+1} = u_i^{n+1} - \hat u_i^{n+1}$, but these can vary between solvers. 
Estimations of the local error $\text{err}_i^{n+1}$ can be based on two approximation to the solution based on Embedded Runge-Kutta pairs   \cite{Ranocha_Dalcin_Parsani_Ketcheson_2022, hairer-solving-1}, or in theoretical upper bounds provided by the numerical solver. 
% The choice of the norm $\frac{1}{\sqrt n} \| \cdot \|_2$ for computing the total error $\text{Err}_\text{scaled}$, sometimes known as Hairer norm, has been the standard for a long time \cite{Ranocha_Dalcin_Parsani_Ketcheson_2022} and it is based on the assumption that a small increase in the size of the systems of ODEs (e.g., by simply duplicating the ODE system) should not affect the stepsize choice, but other options can be considered \cite{hairer-solving-1}.   

Modern solvers include stepsize controllers that pick $\Delta t_i$ as large as possible to minimize the total number of steps while preventing large errors by keeping $\text{Err}_\text{scaled} \leq 1$. 
One of the most used methods to archive this is the proportional-integral controller (PIC) that updates the stepsize according to \cite{hairer-solving-2, Ranocha_Dalcin_Parsani_Ketcheson_2022}
\begin{equation}
    \Delta t_{n+1} = \eta \, \Delta t_n
    \qquad 
    \eta = w_{n+1}^{\beta_1 / q} w_n^{\beta_2 / q} w_{n-1}^{\beta_3 / q}
    \label{eq:PIC}
\end{equation}
with $w_{n+1} = 1 / \text{Err}_\text{scaled}^{n+1}$ the inverse of the scaled error estimates; $\beta_1$, $\beta_2$, and $\beta_3$ numerical coefficients defined by the controller; and $q$ the order of the numerical solver. 
If the stepsize $\Delta t_{n+1}$ proposed in Equation \eqref{eq:PIC} to update from $u^{n+1}$ to $u^{n+2}$ does not satisfy $\text{Err}_\text{scaled}^{n+2} \leq 1$, a new smaller stepsize is proposed. 
When $\eta < 1$ (which is the case for simple controllers with $\beta_2 = \beta_3 = 0$), Equation \eqref{eq:PIC} can be used for the local update. 
It is also common to restrict $\eta \in [\eta_\text{min}, \eta_\text{max}]$ so the stepsize does not change abruptly \cite{hairer-solving-1}. 



\subsubsection{What to differentiate?}

In most applications, the need for differentiating the solution of ODEs stems from the need to obtain the gradient of a function $L(u(\cdot, \theta))$ with respect to the parameter $\theta$, where $L$ can denote
\begin{itemize}
    \item[$ \blacktriangleright$] \textbf{A loss function or an empirical risk function}. This is usually a real-valued function that quantifies the level of agreement between the model prediction and observations. Examples of loss functions include the squared error
    \begin{equation}
         L(\theta) = \frac{1}{2} \| u(t_1; \theta) - u^{\text{target}}(t_1) \|_2^2,
         \label{eq:quadratic-loss-function}
    \end{equation}
    where $u^{\text{target}}(t_1)$ is the desired target observation at some later time $t_1$, and $\| \cdot \|_2$ is the Euclidean norm.
    More generally, we can evaluate the loss function at points of the time series for which we have observations, 
    \begin{equation}
        L(\theta) 
        = 
        \frac{1}{2} \sum_{i=1}^N 
        \, \omega_i \,
        \| u(t_i; \theta) - u^{\text{target}}(t_i) \|_2^2.
        \label{eq:quadratic-loss-point}
    \end{equation}
    with $\omega_i$ some arbitrary non-negative weights.
    More generally, misfit functions used in optimal estimation and control problems are composite maps from the parameter space $\theta$ via the model's state space, in this case, the solution $u(t)$, to the observation space defined by a new variable $y(t) = H(u(t, \theta))$, where $H: \R^n \mapsto \R^o$ is a given function mapping the latent state to observational space \cite{1975-Bryson-Ho-optimal-control}. 
    In these cases, the loss function generalizes to 
    \begin{equation}
        L(\theta) 
        =
        \frac{1}{2} 
        \sum_{i=1}^N
        \, \omega_i \,
        \| H(u(t_i; \theta)) - y^{\text{target}}(t_i) \|_2^2.
        \label{eq:loss-state-observation}
    \end{equation}
    We can also consider the continuous evaluated loss function of the form
    \begin{equation}
         L(u(\cdot, \theta)) = \int_{t_0}^{t_1} h( u(t;\theta), \theta)  dt, 
         \label{eq:integrated-loss-function}
    \end{equation}
    with $h$ being a function that quantifies the contribution of the error term at every time $t \in [t_0, t_1]$. 
    Defining a loss function where just the empirical error is penalized is known as trajectory matching \cite{ramsay2017dynamic}. 
    Other methods like gradient matching and generalized smoothing the loss depends on smooth approximations of the trajectory and their derivatives. 
    % \todo{this is unclear}
    \item[$ \blacktriangleright$] \textbf{The likelihood function or posterior probability.} From a statistical (and physical) perspective, it is common to assume that observations correspond to noisy observations of the underlying dynamical system, $y_i = H(u(t_i; \theta)) + \varepsilon_i$, with $\varepsilon_i$ errors or residual that are independent of each other and of the trajectory $u(\cdot ; \theta)$ \cite{ramsay2017dynamic}.
    When $H$ is the identity, each $y_i$ corresponds to the noisy observation of the state $u(t_i; \theta)$.
    If $p(Y | t , \theta)$ is the probability distribution of $Y=(y_1, y_2, \ldots, y_N)$, 
    the maximum likelihood estimator (MLE) of $\theta$ is defined as 
    \begin{equation}
        \theta^* 
        = 
        \argmax{\theta} \,\, \ell (Y | \theta) 
        = 
        \prod_{i=1}^n p(y_i | \theta, t_i) .
    \end{equation}
    When $\varepsilon_i \sim N(0, \sigma_i^2 \I)$ is the isotropic multivariate normal distribution, the maximum likelihood principle is the same as minimizing $- \log \ell(Y | \theta)$ which coincides with the mean squared error of Equation \eqref{eq:loss-state-observation} \cite{hastie2009elements},
    \begin{equation}
        \theta^* 
        = 
        \argmin{\theta} \, \left \{ - \log \ell (Y | \theta) \right \}
        = 
        \argmin{\theta} \, \sum_{i=1}^N 
        \, \frac{1}{2\sigma_i^2} \,
        \| y_i - H(u(t_i; \theta)) \|_2^2 .
        \label{eq:MLE}
    \end{equation}
    A Bayesian formulation of equation \eqref{eq:MLE} would consist in deriving a point estimate $\theta^*$, the posterior mean of the maximum a posteriori (MAP), based on the posterior distribution for $\theta$ following Bayes theorem as $p(\theta | Y) = {p(Y | \theta) \, p (\theta)}/{p(Y)}$ where $p(\theta)$ is the prior distribution \cite{pml1Book}.
    % \begin{equation}
    %     p(\theta | Y) = \frac{p(Y | \theta) \, p (\theta)}{p(Y)}. 
    % \end{equation}
    In most realistic cases, the posterior distribution is approximated using Markov chain Monte Carlo (MCMC) sampling methods \cite{gelman2013bayesian}. 
    Being able to further compute gradients of the likelihood allows to design more efficient sampling methods, such as Hamiltonian Monte Carlo \cite{Betancourt_2017}.
    % Add mention for variatinal inference problems
    \item[$ \blacktriangleright$] \textbf{A quantity of interest.} In some applications we are interested in quantifying how the solution of the differential equation changes as we vary the parameter values; or more generally when it returns the value of some variable that is a function of the solution of a differential equation. The later corresponds to the case in design control theory, a popular approach in aerodynamics modelling where goals include maximizing the speed of an airplane or the lift of a wing given the solution of the flow equation for a given geometry profile \cite{Jameson_1988,Giles:2000wp,Mohammadi:2004dg}. 
\end{itemize}
In the rest of the manuscript we will use letter $L$ to emphasize that in many cases this will be a loss function, but without loss of generality this includes the richer class of functions included in the previous examples. 

\subsubsection{Gradient-based optimization}

% Add one line adding general framework for optimization, since the following formula is no the more general optimization strategy.
In the context of optimization, the gradient of the loss allows us to perform gradient-based updates on the parameter $\theta$ by 
\begin{equation}\label{eq:gradient-descent}
    \theta^{k+1} 
    = 
    \theta^k 
    - 
    \alpha_k 
    \frac{dL}{d\theta^k}.
\end{equation}
Gradient-based methods tend to outperform gradient-free optimization schemes, as they are not prone to the curse of dimensionality \cite{Schartau2017}. 
A direct implementation of gradient descent following Equation \eqref{eq:gradient-descent} is prone to converge to a local minimum and slows down in a neighborhood of saddle points. 
To address these issues, variants of this scheme employing more advanced updating strategies have been proposed \cite{ruder2016overview-gradient-descent}.
These methods include Newton-type methods \cite{second-order-optimization}, quasi-Newton methods, acceleration techniques \cite{JMLR:v22:20-207}, and natural gradient descent methods \cite{doi:10.1137/22M1477805}. 
For instance, ADAM is an adaptive, momentum-based algorithm  that stores the parameter update at each iteration, and determines the next update as a linear combination of the gradient and the previous update \cite{Kingma2014}.
ADAM been widely adopted to train highly parametrized neural networks (up to the order of $10^8$ parameters \cite{NIPS2017_3f5ee243}).
Other widely employed algorithms are the Broyden–Fletcher–Goldfarb–Shanno (BFGS) and its limited-memory version algorithm (L-BFGS), which determine the descent direction by preconditioning the gradient with curvature information. 
ADAM is less prone to converge to a local minimum, while (L-)BFGS has a faster converge rate. 
Using ADAM for the first iterations followed by (L-)BFGS proves to be a successful strategy to minimize a loss function with best accuracy. 
% Furthermore, gradient-free methods (also known as global optimization techniques \todo{Some gradient free methods are not necessarily global optimization techniques, e.g. evolutionary algorithms \cite{wilke2001evolution,Rodriguez-Fernandez2006} }) rely in heuristics\cite{Pearl-heuristics} that are not guaranteed to find the solution. 
% I will mentioned that other methods for optimization based on the gradient exists (e.g., majorization) but not giving much details on it. 

% 

\subsubsection{Sensitivity matrix}

In general, loss functions considered are of the form $L(\theta) = L(u(\cdot, \theta), \theta)$. 
Using the chain rule we can derive 
\begin{equation} 
 \frac{dL}{d\theta} = \frac{\partial L}{\partial u} \frac{\partial u}{\partial \theta} + \frac{\partial L}{\partial \theta}.
 \label{eq:dLdtheta_VJP}
\end{equation} 
Notice here the distinction between the direct derivative $\frac{d}{d\theta}$ and partial derivative $\frac{\partial}{\partial \theta}$.
The two partial derivatives of the loss function on the right-hand side are usually easy to evaluate.
For example, for the loss function in Equation \eqref{eq:quadratic-loss-function} these are simply given by 
\begin{equation}
    \frac{\partial L}{\partial u} = u - u^{\text{target}}(t_1)
    \qquad 
    \frac{\partial L}{\partial \theta} = 0.
    \label{eq:dLdu}
\end{equation}
In most applications, the empirical component of the loss function $L(\theta)$, that is, the part of the loss that is a function on the data, will depend on $\theta$ just through $u$, meaning $\frac{\partial L}{\partial \theta} = 0$. 
However, regularization terms added to the loss can directly depend on the parameter $\theta$, that is $\frac{\partial L}{\partial \theta} \neq 0$.
In both cases, the complicated term to compute is the matrix of derivatives $\frac{\partial u}{\partial \theta}$, usually referred to as the \textit{sensitivity} $s$, and represents how much the full solution $u$ varies as a function of the parameter $\theta$, 
\begin{equation}
 s 
 = 
 \frac{\partial u}{\partial \theta} 
 =
 \begin{bmatrix}
   \frac{\partial u_1}{\partial \theta_1} & \dots & \frac{\partial u_1}{\partial \theta_p} \\
   \vdots & \ddots & \vdots \\
   \frac{\partial u_n}{\partial \theta_1} & \dots & \frac{\partial u_n}{\partial \theta_p}
 \end{bmatrix}
 \in \mathbb R^{n \times p}.
 \label{eq:sensitivity-definition}
\end{equation}
The sensitivity $s$ defined in Equation \eqref{eq:sensitivity-definition} is a \textit{Jacobian}, that is, a matrix of first derivatives of a vector-valued function. 
Methods involved in the calculation of $s$ are naturally part of sensitivity methods.
As mentioned earlier, most of forward sensitivity methods compute the full sensitivity matrix $s$, while reverse methods only deal with Jacobian-vector products (JVPs) of the form $\frac{\partial u}{\partial \theta} v$, for some vector $v \in \R^p$, saving unnecessary calculations at the expenses of larger memory cost.
The product $\frac{\partial u}{\partial \theta}v$ is the directional derivative of the function $u(\theta)$ in the direction $v$, given by 
\begin{equation}
    \frac{\partial u}{\partial \theta} v 
    = 
    \lim_{h \rightarrow 0} \frac{u(\theta + h v) - u(\theta)}{h},
    \label{eq:directional-derivative}
\end{equation}
representing how much the function $u$ changes when we perturb $\theta$ in the direction of $v$. 

% Notice here the distinction between the total derivative (indicated with the $d$) and partial derivative symbols ($\partial$). 
% When a function depends on more than one argument, we use the partial derivative symbol to emphasize this distinction (e.g., Equation \eqref{eq:sensitivity-definition}). 
% On the other hand, when this is not the case, we will use the total derivative symbol (e.g., Equation \eqref{eq:dLdu}).




\subsection{Finite differences}
Finite differences are arguably the simplest scheme to obtain the derivative of a function. 
% The simplest way of evaluating a derivative is by computing the difference between the evaluation of the function at a given point and a small perturbation of the function. 
In the case of the function $L : \R^p \mapsto \R$, a first-order Taylor expansion yields to the following expression for the directional derivative
\begin{equation}
 \frac{dL}{d\theta_i} (\theta) = \frac{L(\theta + \varepsilon e_i ) - L(\theta)}{\varepsilon} + \mathcal O (\varepsilon),
 \label{eq:finite_diff}
\end{equation}
with $e_i$ the $i$-th canonical vector and $\varepsilon$ the stepsize. 
Even better, the centered difference scheme leads to
\begin{equation}
 \frac{dL}{d\theta_i} (\theta) 
 =
 \frac{L(\theta + \varepsilon e_i ) - L(\theta - \varepsilon e_i)}{2\varepsilon}
 + \mathcal O (\varepsilon^2).
 \label{eq:finite_diff2}
\end{equation}
% leads to a more accurate estimation of the derivative. 
While Equation \eqref{eq:finite_diff} gives the derivative to an error of magnitude $\mathcal O (\varepsilon)$, the centered differences schemes improves the accuracy to $\mathcal O (\varepsilon^2)$ \cite{ascher2008-numerical-methods}. 
Further finite difference stencils of higher order exist in the literature \cite{Fornberg1988}. 
% This method falls under the forward and discrete category, because...
 
Finite difference scheme are subject to a number of issues, related to the parameter vector dimension and rounding errors.
Firstly, calculating directional derivatives requires at least one extra function evaluations per parameter dimension.
For the centered differences approach in Equation \eqref{eq:finite_diff2}, this requires a total of $2p$ function evaluations which demands solving the DE each time for a new set of parameters.
Second, finite differences involve the subtraction of two closely valued numbers, which can lead to floating point cancellation errors when the step size $\varepsilon$ is small \cite{Goldberg_1991_floatingpoint}. 
% Both Equations \eqref{eq:finite_diff} and \eqref{eq:finite_diff2} involve the subtraction of two numbers that are very close to each other, which leads to large cancellation errors for small values of $\varepsilon$ that are amplified by the division by $\varepsilon$.
While small values of $\varepsilon$ lead to cancellation errors, large values of the stepsize give inaccurate estimations of the derivative. 
Furthermore, numerical solutions of DEs have errors that are typically larger than machine precision, which leads to inaccurate estimations of the gradient when $\varepsilon$ is too small (see also Section \ref{section:software-finite-differences}).
Finding the optimal value of $\varepsilon$ that balances these two effects is sometimes known as the \textit{stepsize dilemma}, for which algorithms based on prior knowledge of the function to be differentiated or algorithms based on heuristic rules have been introduced \cite{mathur2012stepsize-finitediff, BARTON_1992_finite_diff, SUNDIALS-hindmarsh2005sundials}. 
% Victor suggestion: Second, finite differences involve the subtraction of two closely valued numbers, which can lead to significant cancellation errors when the step size $\varepsilon$ is small. numerical solutions of differential equations have errors that are typically larger than machine precision, which leads to inaccurate estimations of the gradient when $\varepsilon$ is too small (see also \ref{sec:computational-implementation}).  This error is exacerbated by the division by $\varepsilon$, leading to what is known as the "stepsize dilemma".This dilemma concerns finding an optimal $\varepsilon$ that balances the accuracy of the gradient estimation against the amplification of rounding errors. Various methods, including those based on prior knowledge of the function or heuristic rules, have been proposed to address this challenge

Despite these caveats, finite differences can prove useful in specific contexts, such as computing Jacobian-vector products (JVPs). 
Given a Jacobian matrix $J = \frac{\partial f}{\partial u}$ (or the sensitivity $s = \frac{\partial u}{\partial \theta}$) and a vector $v$, the product $Jv$ corresponding to the directional derivative and can be approximated as 
\begin{equation}
    Jv \approx \frac{f(u + \varepsilon v, \theta, t) - f(u, \theta, t)}{\varepsilon}
\end{equation}
This approach is used in numerical solvers based on Krylov methods, where linear systems are solved by iteratively solving matrix-vectors products \cite{Ipsen_Meyer_1998}.



\subsection{Complex step differentiation}
Other authors had suggested to use complex variable analysis to compute the derivative as the imaginary part of $L(\theta + i \varepsilon)/h$, which coincides with $L'(\theta)$ for the case where $\theta$ is a scalar \cite{Squire_Trapp_1998_complex_diff, Martins_Sturdza_Alonso_2003_complex_differentiation}.
% This approach was later generalized to the multivariable case 
Although this solves the problem of cancellation errors in Equation \eqref{eq:finite_diff2} for small $\varepsilon$, this approach is just valid for cases where the function $L$ is known analytically. 


\subsection{Automatic differentiation}
Automatic differentiation (AD) is a technology that allows computing gradients thought a computer program. 
The main idea is that every computer program manipulating numbers can be reduced to a sequence of simple algebraic operations that can be easily differentiable. 
The derivatives of the outputs of the computer program with respect to their inputs are then combined using the chain rule.
One advantage of AD systems is that we can automatically differentiate programs that include control flow, such as branching, loops or recursions. 
This is because at the end of the day, any program can be reduced to a trace of input, intermediate and output variables \cite{Baydin_Pearlmutter_Radul_Siskind_2015}.

Depending if the concatenation of these gradients is done as we execute the program (from input to output) or in a later instance were we trace-back the calculation from the end (from output to input), we are going to talk about \textit{forward} or \textit{backward} AD, respectively.

\subsubsection{Forward mode}

Forward mode AD can be implemented in different ways depending on the data structures we use at the moment of representing a computer program. Examples of these data structures include dual numbers and Wengert lists (see \cite{Baydin_Pearlmutter_Radul_Siskind_2015} for a good review on these methods). 

Let's first consider the case of dual numbers.
The idea is to extend the definition of a variable that takes a certain value to also carry information about the derivative with respect to certain scalar parameter $\theta \in \R$. 
We can define an abstract type, defined as a dual number, composed of two elements: a \textit{value} coordinate $x_1$ that carries the value of the variable and a \textit{derivative} coordinate $x_2$ with the value of the derivative $\frac{\partial x_1}{\partial \theta}$. 
Just as complex number do, we can represent dual numbers in the vectorial form $(x_1, x_2)$ or in the rectangular form 
\begin{equation}
 x_\epsilon = x_1 + \epsilon \, x_2
\end{equation}
where $\epsilon$ is an abstract number with the properties $\epsilon^2 = 0$ and $\epsilon \neq 0$.
This last representation is quite convenient since it naturally allow us to extend algebraic operations, like addition and multiplication, to dual numbers. 
For example, given two dual numbers $x_\epsilon = x_1 + \epsilon x_2$ and $y_\epsilon = y_1 + \epsilon y_2$, it is easy to derive using the fact $\epsilon^2=0$ that
\begin{equation}
 x_\epsilon + y_\epsilon = (x_1 + y_1) + \epsilon \, (x_2 + y_2)
 \qquad
 x_\epsilon y_\epsilon = x_1 y_1 + \epsilon \, (x_1 y_2 + x_2 y_1) .
 %\qquad
 %\frac{x_\epsilon}{y_\epsilon} = \frac{x_1}{y_1} + \epsilon \, \frac{x_2 y_1 - x_1 y_2}{y_1^2}.
\end{equation}
From these last examples, we can see that the derivative component of the dual number carries the information of the derivatives when combining operations.
For example, suppose than in the last example the dual variables $x_2$ and $y_2$ carry the value of the derivative of $x_1$ and $x_2$ with respect to a parameter $\theta$, respectively. 


Intuitively, we can think about $\epsilon$ as being a differential in the Taylor expansion:
\begin{equation}
 f(x + \epsilon) = f(x) + \epsilon f'(x) + \mathcal O (\epsilon^2).
\end{equation}
When computing first order derivatives, we can ignore everything of order $\epsilon^2$ or larger, which is represented in the condition $\epsilon^2 = 0$.
This implies that we can use dual numbers to implement forward AD through a numerical algorithm. Implementing such a type in a programming language implies defining what it means to perform basic operations and then combine them using the chain rule provided by the dual number properties.
For example, in Julia we create a dual number object and extend the definition of algebraic operations by relying in multiple distpach
\begin{jllisting}
using Base: @kwdef

@kwdef struct DualNumber{F <: AbstractFloat}
    value::F
    derivative::F
end

# Binary sum
Base.:(+)(a::DualNumber, b::DualNumber) = DualNumber(value = a.value + b.value, derivative = a.derivative + b.derivative)

# Binary product 
Base.:(*)(a::DualNumber, b::DualNumber) = DualNumber(value = a.value * b.value, derivative = a.value*b.derivative + a.derivative*b.value)
\end{jllisting}
and then we can simply evaluate derivatives by evaluating the derivative component of the dual number that results from the combination of operations
\begin{jllisting}
a = DualNumber(value=1.0, derivative=1.0)

b = DualNumber(value=2.0, derivative=0.0)
c = DualNumber(value=3.0, derivative=0.0)

result = a * b * c
println("The derivative of a*b*c with respect to a is: ", result.derivative)
\end{jllisting}
Notice that in this last example the dual numbers \texttt{b} and \texttt{c} where initialized with derivative value equals to zero, while \texttt{a} with value equals to one. 
This is because we were interested in computing the derivative with respect to \texttt{a}, and then $\frac{\partial a}{\partial a} = 1$, while $\frac{\partial b}{\partial a} = \frac{\partial c}{\partial a} = 0$. 

In the Julia ecosystem, \texttt{ForwardDiff.jl} implements forward mode AD with multidimensional dual numbers \cite{RevelsLubinPapamarkou2016}. 
Notice that a major limitation of the dual number approach is that we need a dual variable for each variable we want to differentiate. 

\subsubsection{Backward mode}

Backward mode AD is also known as the adjoint of cotangent linear mode, or backpropagation in the field of machine learning. Given a directional graph of operations defined by a Wengert list, we can compute gradients of any given function backwards as
\begin{equation}
 \bar v = \frac{\partial \ell}{\partial v_i}= \sum_{w : v \rightarrow w \in G} \frac{\partial w}{\partial v} \bar{w}.
\end{equation}

Another way of implementing backwards AD is by defining a \textit{pullback} function \cite{Innes_2018}, a method also known as \textit{continuation-passing style} \cite{Wang_Zheng_Decker_Wu_Essertel_Rompf_2019}. In the backward step, this executes a series of functions calls, one for each elementary operation.
If one of the nodes in the graph $w$ is the output of an operation involving the nodes $v_1, \ldots, v_m$, where $v_i \rightarrow w$ are all nodes in the graph, then the pullback $\bar v_1, \ldots, \bar v_m = \mathcal B_w(\bar w)$ is a function that accepts gradients with respect to $w$ (defined as $\bar w$) and returns gradients with respect to each $v_i$ ($\bar v_i$) by applying the chain rule. Consider the example of the multiplicative operation $w = v_1 \times v_2$. Then
\begin{equation}
 \bar v_1, \bar v_2 = v_2 \times \bar w , \quad
 v_1 \times \bar w = \mathcal{B}_w (\bar w),
\end{equation}
which is equivalent to using the chain rule as
\begin{equation}
 \frac{\partial \ell}{\partial v_1} = \frac{\partial}{\partial v_1}(v_1 \times v_2) \frac{\partial \ell}{\partial w}.
\end{equation}

The libraries \texttt{ReverseDiff.jl} and \texttt{Zygote.jl} use callbacks to compute gradients. When gradients are being computed with less than $\sim 100$ parameters, the former is faster (see documentation).

\subsubsection{AD connection with JVPs and VJPs}

When working with unit operations that involve matrix operations dealing with vectors of different dimensions, the order in which we apply the chain rule matters. When computing a gradient using AD, we can encounter vector-Jacobian products (VJPs) or Jacobian-vector products (JVP). As their name indicate, the difference between them regards the fact if the quantity we are interested in computing is described by the product of a Jacobian (the two dimensional matrix with the gradients as rows) by a vector on the left
side (VJP) or the right (JVP).

For the examples we care here, the Jacobian is described as the product of multiple Jacobian using the chain rule. In this case, the full gradient is computed as the chain product of vectors and Jacobians. Let's consider for example the case of a loss function $L : \mathbb R^n \mapsto \mathbb R$ that can be decomposed as $L(\theta) = \ell \circ g_{k} \circ \ldots \circ g_2 \circ g_1(\theta)$, with $\ell : \mathbb R^{d_k} \mapsto \mathbb R$ the final evaluation of the loss function after we apply in order a sequence of intermediate functions $g_i : \mathbb R^{d_{i-1}} \mapsto \mathbb R^{d_i}$, $d_0 = n$. Examples of this are neural networks and iterative differential equation solvers. Now, using the chain rule, we can calculate the gradient of the final loss funcition as
\begin{equation}
 \nabla_\theta L = \nabla \ell \cdot Dg_{k} \cdot Dg_{k-1} \cdot \ldots \cdot Dg_2 \cdot Dg_1.
\end{equation}
Notice that in the last equation, $\nabla \ell \in \mathbb R^{d_k}$ is a vector, while all the other term $Dg_i$ are full matrices. In order to compute $\nabla_\theta L$, we can solve the multiplication starting from the right side, which will correspond to multiple the Jacobians forward in time from $Dg_1$ to $Dg_k$, or from the left side, moving backwards in time. The important aspect of this last case is that we will always been computing VJPs, since the product of a vector and a matrix is a vector. Since VJP are easier to evaluate than full Jacobians, the backward mode will be in general faster (see Figure \ref{fig:vjp-jvp}). For general rectangular matrices $A\in \mathbb R^{d_1 \times d_2}$ and $B \in \mathbb R^{d_2 \times d_3}$, the cost of the matrix multiplication $AB$ is $\mathcal O (d_1 d_2 d_3)$. This implies that forward AD requires a total of
\begin{equation}
 d_2 d_1 n + d_3 d_2 n + \ldots + d_k d_{k-1} n + d_k n = \mathcal O (kn)
\end{equation}
operations, while backwards mode AD requires
\begin{equation}
 d_k d_{k-1} + d_{k-1} d_{k-2} + \ldots + d_2 d_1 + d_1 n = \mathcal O (k + n)
\end{equation}
operations.
In general, when the function we are trying to differentiate has a larger input space than output, which is usually the case when working with scalar loss functions, AD in backward mode is more efficient as it propagates the chain rule by computing VJPs. On the other side, when the output dimension is larger than the input space dimension, forwards AD is more efficient. This is the reason why in most machine learning application people use backwards AD. However, notice that backwards mode AD requires us to save the solution thought the forward run in order to run backwards afterwards, while in forward mode we can just evaluate the gradient as we iterate our sequence of functions. This means that for problems with a small number of parameters, forward mode can be faster and more memory-efficient that backwards AD.

\begin{figure}
    \centering
    \includegraphics[width=0.95\textwidth]{figures/VJP-JVP.png}
    \caption{Comparison between forward and backward AD. Changing the order how we multiply the Jacobians change the total number of floating-point operations, which leads to different computational complexities between forward and backward mode.}
    \label{fig:vjp-jvp}
\end{figure}

\subsubsection{Symbolic differentiation}

Sometimes AD is compared against symbolic differentiation.
According to \cite{Laue2020}, these two are the same and the only difference is in the data structures used to implement them, while \cite{Elliott_2018} suggests that AD is symbolic differentiation performed by a compiler.

\subsection{Symbolic differentiation}
In symbolic differentiation, functions are represented algebraically instead of algorithmically, which is why many symbolic differentiation tools are included inside computer algebra systems (CAS) \cite{Symbolics_jl_2022}. 
Instead of numerically evaluating the final value of a derivative, symbolic systems define \textit{algebraic} objects, including variable names, expressions, operations, and literals. 
For example, the relation $y = x^2$ is interpreted as expression with two variables, $x$ and $y$, and the symbolic system generates the derivative $y' = 2 \times x$ with $2$ a numeric literal, $\times$ a binary operation, and $x$ the same variable assignment as in the original expression.
When the function to differentiate is large, symbolic differentiation can lead to \textit{expression swell}, that is, exponentially large or complex symbolic expressions \cite{Baydin_Pearlmutter_Radul_Siskind_2015}.
Here, an important piece of CAS is simplification routines that reduce the size and complexity of algebraic expressions by finding common sub-expressions.  
This can make symbolic differentiation very efficient when computing derivatives multiple times and for different input values \cite{Dürrbaum_Klier_Hahn_2002}. 

It is important to remark on the close relationship between AD and symbolic differentiation.
There is no agreement as to whether symbolic differentiation should be classified as AD \cite{juedes1991taxonomy, Elliott_2018, Laue2020} or as a different method \cite{Baydin_Pearlmutter_Radul_Siskind_2015}.  
Both are equivalent in the sense that they perform the same operations but the underlying data structure is different \cite{Laue2020}. 
Here, expression swell is a consequence of the underlying representation when this does not allow for common sub-expressions. 
This can also be understood as if AD is symbolic differentiation performed by a compiler \cite{Elliott_2018}, meaning that different AD can be classified based on the level of integration with the underlying source language \cite{juedes1991taxonomy}.


\subsection{Sensitivity equations}
An easy way to derive an expression for the sensitivity $s$ is by deriving the sensitivity equations \cite{ramsay2017dynamic}, a method also referred to as continuous local sensitivity analysis (CSA). 
If we consider the original system of ODEs and we differentiate with respect to $\theta$, we then obtain
\begin{equation}
 \frac{d}{d\theta} \frac{du}{dt} 
 =
 \frac{d}{d\theta} f(u(\theta), \theta, t)
 = 
 \frac{\partial f}{\partial \theta}
 + 
 \frac{\partial f}{\partial u} \frac{\partial u}{\partial \theta},
\end{equation}
that is
\begin{equation}
 \frac{ds}{dt} = \frac{\partial f}{\partial u} s + \frac{\partial f}{\partial \theta}.
 \label{eq:sensitivity_equations}
\end{equation}
By solving the sensitivity equation at the same time we solve the original differential equation for $u(t)$, we ensure that by the end of the forward step we have calculated both $u(t)$ and $s(t)$. 
This also implies that as we solve the model forward, we can ensure the same level of numerical precision for the two of them.

In opposition to the methods previously introduced, the sensitivity equations find the gradient by solving a new set of continuous differential equations.
Notice also that the obtained sensitivity $s(t)$ can be evaluated at any given time $t$. This method can be labeled as forward, since we solve both $u(t)$ and $s(t)$ as we solve the differential equation forward in time, without the need of backtracking any operation though the solver.

For systems of equations with few number of parameters, this method is useful since the system of equations composed by Equations \eqref{eq:original_ODE} and \eqref{eq:sensitivity_equations} can be solved in $\mathcal O (np)$ using the same precision for both solution and sensitivity numerical evaluation. 
Furthermore, this method does not required saving the solution in memory, so it can be solved purely in forward mode without backtracking operations.
However, notice that the term $\frac{\partial f}{\partial u} s $ is in general difficult to compute. 

It is important to remark that the sensitivity equations can be also solved in discrete forward mode by numerically discretizing the original ODE and later deriving the discrete sensitivity equations. 
For most cases, this leads to the same result that in the continuous case \cite{FATODE2014}.

%Notice that since $s(t)$ is a matrix of size $n \times p$, the complexity of solving the sensitivity equation scales as $\mathcal O(np)$. From a computational perspective, the sensitivity $s$ appears in the sensitivity equation as a VJP, meaning the the term $\frac{\partial f}{\partial u} s$ can be efficiently computed using backwards automatic differentiation. 

\subsection{Adjoint methods}
\label{section:adjoint-methods}
The adjoint method is a very popular approach to compute the gradients of a loss function by first computing an intermediate variable (the adjoint) that serves as a bridge between the solution of the ODE and the final sensitivity. 
There is a large family of adjoint methods that a first order we can classify them between discrete and continuous adjoints. 
The former usually arises as the numerical discretization of the later, and when the discrete adjoint method is a consistent estimator of the continuous adjoint depends of the ODE and equation.  
Proofs of the consistency of discrete adjoint methods for Runge-Kutta methods had been provided in \cite{sandu2006properties, sandu2011solution}.
Depending the choice of the Runge-Kutta coefficients, we can have a numerical scheme that is both consistent for the original equation and consistent/inconsistent for the adjoint \cite{Hager_2000}.

\subsubsection{Discrete adjoint method}
The adjoint state method is another example of a discrete method that aims to find the gradient by solving an alternative system of linear equations, known as the \textit{adjoint equations}, at the same time that we solve the original system of linear equations defined by the numerical solver. 
These methods are extremely popular in optimal control theory in fluid dynamics, for example for the design of geometries for vehicles and airplanes that optimize performance \cite{Elliott_Peraire_1996, Giles_Pierce_2000}
This approach follows the discretize-optimize approach, meaning that we first discretize the system of continuous ODEs and then solve on top of these linear equations \cite{Giles_Pierce_2000}. 
Just as in the case of automatic differentiation, the set of adjoint equations can be solved in both forward and backward mode. 
% Just as in the case of automatic differentiation, the adjoint state method evaluates the gradient by moving forward in time and applying the chain rule sequentially over a discrete set of operations that dictate the updates by the numerical scheme for solving the differential equation. However, it does so by directly computing the gradient by solving a new system of equations.

The first step in order to derive the adjoint equation is to discretize the set of differential equations in \eqref{eq:original_ODE} into finite evaluations of the function $u(t; \theta)$. 
Given the set of timesteps $t_0, t_1, \ldots, t_N$, we evaluate the solution at $u_i = u(t_i; \theta)$. 
In the case of using an explicit numerical solver, these values will be constrained to satisfy a set of equations of the form 
\begin{equation}
    u_{i+1} = A_i (\theta) \, u_i + b_i
\end{equation}
with $A_i \in \R^{n \times n}$ a squared matrix defined by the numerical solver. 
If we call the super-vector $U = (u_1, u_2, \ldots, u_N) \in \R^{nN}$, we can combine all these equations in into one single system of linear equations 
\begin{equation}
    A(\theta) U 
    = 
    \begin{bmatrix}
        \I_{n \times n} & 0 &   &  & \\
        -A_1 & \I_{n \times n} & 0 &  &  \\
          & -A_2 & \I_{n \times n} & 0 &  \\
         &  &   & \ddots &   \\
         &  &  & -A_{N-1} & \I_{n \times n}
    \end{bmatrix}
    \begin{bmatrix}
        u_1 \\
        u_2 \\
        u_3 \\
        \vdots \\
        u_N
    \end{bmatrix}
    = 
    \begin{bmatrix}
        M_0 u_0 + b_0 \\
        b_1 \\
        b_2 \\
        \vdots \\
        b_{N-1}
    \end{bmatrix}
    = 
    b(\theta), 
\end{equation}
with $\I_{n \times n}$ the identity matrix of size $n \times n$.
It is usually convenient to write this system of linear equations in the residual form $G(U; \theta) = 0$, where $G(U; \theta) = A(\theta) U - b(\theta)$ is the residual between both sides of the equation. 
Different numerical schemes will lead to different design matrix $A(\theta)$ and vector $b(\theta)$, but ultimately every numerical method will lead to a system of linear equations with the form $G(U; \theta) = A(\theta) U - b(\theta) = 0$ after being discretized. 

We are interested in differentiating a function $h(U, \theta)$ constrained to satisfy the algebraic linear equation $G(U; \theta) = 0$.
Now,
\begin{equation}
    \frac{dh}{d\theta} = \frac{\partial h}{\partial \theta} + \frac{\partial h }{\partial U} \frac{\partial U}{\partial \theta},
    \label{eq:dhdtheta0}
\end{equation}
and also for the constraint $G(U; \theta)=0$ we can derive
\begin{equation}
    \frac{dG}{d\theta} 
    = 
    \frac{\partial G}{\partial \theta} 
    + 
    \frac{\partial G}{\partial U} \frac{\partial U}{\partial \theta}
    =
    0
\end{equation}
which is equivalent to 
\begin{equation}
    \frac{\partial U}{\partial \theta} 
    = 
    - \left( \frac{\partial G}{\partial U} \right)^{-1} \frac{\partial G}{\partial \theta}.
\end{equation}
If we replace this last expression into equation \eqref{eq:dhdtheta0}, we obtain
\begin{equation}
    \frac{dh}{d\theta} 
    =
    \frac{\partial h}{\partial \theta} - \frac{\partial h}{\partial U} \left( \frac{\partial G}{\partial U} \right)^{-1} \frac{\partial G}{\partial \theta}.
    \label{eq:dhdtheta}
\end{equation}
Now, let's define the adjoint $\lambda \in \R^{nN}$ as the solution of the linear system of equations 
\begin{equation}
    \left( \frac{\partial G}{\partial U}\right)^T \lambda =  \left( \frac{\partial h}{\partial U} \right)^T,
\end{equation}
that is,
\begin{equation}
    \lambda^T = \frac{\partial h}{\partial U} \left( \frac{\partial g}{\partial U} \right)^{-1}.
    \label{eq:def_adjoint}
\end{equation}
Finally, if we replace Equation \eqref{eq:def_adjoint} into \eqref{eq:dhdtheta}, we obtain 
\begin{equation}
    \frac{dh}{d\theta} 
    =
    \frac{\partial h}{\partial \theta} - \lambda^T \frac{\partial G}{\partial \theta}.
\end{equation}
The important trick to notice here is the rearrangement of the multiplicative terms involved in equation \eqref{eq:dhdtheta}. Computing the full Jacobian/sensitivity $\partial u / \partial \theta$ will be computationally expensive and involves the product of two matrices. However, we are not interested in the calculation of the Jacobian, but instead in the VJP given by $\frac{\partial h}{\partial U} \frac{\partial U}{\partial \theta}$. By rearranging these terms, we can make the same computation more efficient. 

\begin{example*}[Linear system]
Let's see this by considering the simple example of a linear system of equations. Suppose that the constraint $g(u, \theta)=0$ takes the form $A(\theta) u = b(\theta)$ \cite{Johnson}. In that case, we have
\begin{equation}
    \frac{\partial g}{\partial \theta} = \frac{\partial A }{\partial \theta} u - \frac{\partial b}{\partial \theta},
\end{equation}
so the gradient can be computed as 
\begin{equation}
    \frac{dh}{d\theta} = \frac{\partial h}{\partial \theta} - \lambda^T \left( \frac{\partial A }{\partial \theta} u - \frac{\partial b}{\partial \theta} \right)
    \label{eq:dhdtheta_linear}
\end{equation}
with $\lambda$ the solution of the linear system 
\begin{equation}
    A(\theta)^T \lambda = \frac{\partial h}{\partial u}^T.
\end{equation}
This is a linear system of equations with the same size of the original $Au = b$, but involving the adjoint matrix $A^T$. Computationally this also means that if we can solve the original system then we can also solve the adjoint (for example, with LU factorization $A=LU$ we have $A^T=U^TL^T$).
This example also allows us to see the improvement  in efficiency achieved by first computing the adjoint and then the full gradient. For a linear constraint, equation \eqref{eq:dhdtheta_linear} becomes
\begin{equation}
    \frac{dh}{d\theta} = \frac{\partial h}{\partial \theta} - 
    \underbrace{\frac{\partial h}{\partial u}}_{1 \times n}
    \underbrace{A^{-1}}_{n \times n} 
    \underbrace{\left( \frac{\partial A }{\partial \theta} u - \frac{\partial b}{\partial \theta} \right)}_{n \times p}.
    \label{eq:state_method_linear}
\end{equation}
Computing first the last product will cost $O(n^2p)$ to generate a $n \times p$ matrix. On the other side, if first we solve the first two terms in the product, this will cost $O(n)$ and them the product will be just $O(np)$. In order words, it is easy to compute the full gradient if we treat the last term in equation \eqref{eq:state_method_linear} as a VJP. 
\end{example*}

In order to compute the gradient of the full solution of the differential equation, we apply this method sequentially using the chain rule. One single step of the state method can be understood as the chain of operations $\theta \mapsto g \mapsto u \mapsto L$. This allows us to create adjoints for any primitive function $g$ (i.e. the numerical solver scheme) we want, and then incorporated it as a unit of any AD program. 

\subsubsection{Continuous adjoint method}
The continuous adjoint method, also known as continuous adjoint sensitivity analysis (CASA), operates by defining a convenient set of new differential equations for the adjoint variable and using this to compute the gradient in a more efficient manner. 
Mathematically speaking, the adjoint equations can be derived from a duality or Lagrangian point of view \cite{Giles_Pierce_2000}.
We prefer to derive the equation using the former methods since we believe it gives better insights to how the method works and also allow to generalize to other user cases. 
The derivation of both the discrete and continuous adjoint methods using Lagrangian multipliers can be found in Appendix \ref{appendix:lagrangian}.
We encourage the interested reader to make the effort of following how the continuous adjoint method follows the same logic than the discrete methods, but where the discretization of the differential equation does not happen until the very last step, when the solutions of the differential equations involved need to be numerically evaluated. 
%\footnote{Based on slides of Chris Rackauckas about "Data Efficient model discovery with Scientific Machine Learning", Neural ODE paper, video "The use and practice of scientific machine learning". Paper comparing CSA and DSA} 

Consider an integrated loss function of the form 
\begin{equation}
    L(u; \theta) = \inttime h(u(t;\theta), \theta) dt
\end{equation}
and its derivative with respect to the parameter $\theta$ given by
\begin{equation}
    \frac{dL}{d\theta}
    = 
    \inttime \left( \frac{\partial h}{\partial \theta} + \frac{\partial h}{\partial u} s(t) \right) dt.
    \label{eq:casa-loss}
\end{equation}
As explained in previous section, the complicated term to evaluate in the last expression is the sensitivity (Equation \eqref{eq:sensitivity-definition}).
Just as in the case of the discrete adjoint method, the trick is to evaluate the VJP $\frac{\partial h}{\partial u} \frac{\partial u}{\partial \theta}$ by defining an intermediate adjoint variable. 
The continuous adjoint equation now is obtained by finding the dual/adjoint equation of the sensitivity equation using the weak formulation of Equation \eqref{eq:sensitivity_equations}. 
The adjoint equation is obtained by writing the sensitivity equation in the form 
\begin{equation}
    \inttime \lambda(t)^T \left( \frac{ds}{dt} - f(u, \theta, t) \, s - \frac{\partial f}{\partial \theta}  \right) dt 
    = 
    0,
    \label{eq:integrated-sensitivity-equation}
\end{equation}
where this equation must be satisfied for every function $\lambda(t)$ in order to Equation \eqref{eq:continuous-adjoint} to be true. 
The next step is to get rid of all time derivative applied to the sensitivity $s(t)$ using integration by parts: 
\begin{equation}
    \inttime \lambda(t)^T \frac{ds}{dt} dt
    = 
    \lambda(t_1)^T s(t_1) - \lambda(t_0)^T s(t_0)
    -
    \inttime \frac{d\lambda^T}{dt} s(t)\, dt.
\end{equation}
Replacing this last expression into Equation \eqref{eq:integrated-sensitivity-equation} we obtain 
\begin{equation}
    \inttime \left( - \frac{d\lambda^T}{dt} -  \lambda(t)^T f(u, \theta, t) \right) s(t) dt
    =
    \inttime \lambda(t)^T \frac{\partial f}{\partial \theta} dt 
    - 
    \lambda(t_1)^T s(t_1)
    + 
    \lambda(t_0)^T s(t_0).
    \label{eq:casa-semiadjoint}
\end{equation}
At first glance, there is nothing particularly interesting about this last equation. 
However, both Equations \eqref{eq:casa-loss} and \eqref{eq:casa-semiadjoint} involved a VJP with $s(t)$. 
Since Equation \eqref{eq:casa-semiadjoint} must hold for every function $\lambda(t)$, we can pick $\lambda(t)$ to make the terms involving $s(t)$ in Equations \eqref{eq:casa-loss} and \eqref{eq:casa-semiadjoint} to perfectly coincide. 
This is done by defining the adjoint $\lambda(t)$ to be the solution of the new system of differential equations
\begin{equation}
    \frac{d\lambda}{dt} 
    = 
    - 
    f(u, \theta, t)^T \lambda  
    - 
    \frac{\partial h^T}{\partial u} 
    \qquad \quad \lambda(t_1) = 0. 
    \label{eq:casa-adjoint-equation}
\end{equation}
Notice that the adjoint equation is define with final condition at $t_1$, meaning that it needs to be solved backwards in time. 
The definition of the adjoint $\lambda(t)$ as the solution of this last ODE simplifies Equation \eqref{eq:casa-semiadjoint} to
\begin{equation}
    \inttime \frac{\partial h}{\partial u} s(t) dt
    = 
    \lambda(t_0)^T s(t_0)
    + 
    \inttime \lambda^T (t) \frac{\partial f}{\partial \theta} dt.
\end{equation}
Finally, replacing this inside the expression for the gradient of the loss function we have 
\begin{equation}
    \frac{dL}{d\theta}
    = 
    \lambda^T(t_0) s(t_0)
    + 
    \inttime
    \left( \frac{\partial h}{\partial \theta} + \lambda^T \frac{\partial f}{\partial \theta} \right) dt
    \label{eq:casa-final-loss-gradient}
\end{equation}
The full algorithm to compute the full gradient $\frac{dL}{d\theta}$ can be described as follows:
\begin{enumerate}
    \item Solve the original differential equation $\frac{du}{dt} = f(u, t, \theta)$;
    \item Solve the backwards adjoint differential equation \eqref{eq:casa-adjoint-equation};
    \item Compute the gradient using Equation \eqref{eq:casa-final-loss-gradient}.
\end{enumerate}
The bottleneck on this method is the calculation of the adjoint, since in order to solve the adjoint equation we need to know $u(t)$ at any given time. 
Effectively, notice that the adjoint equation involves the term $f(u, \theta, t)$ and $\frac{\partial h}{\partial u}$ which are both functions of $u(t)$. 
There are different ways of addressing the evaluation of $u(t)$ during the backward step:
\begin{enumerate}[label=(\roman*)]
    \item Solve for $u(t)$ again backwards.
    \item Store $u(t)$ in memory during the forward step.
    \item Store reference values in memory and interpolate in between. 
    This technique is known as \textit{checkpoiting} or windowing that tradeoffs computational running time and storage.
    This is implemented in \texttt{Checkpoiting.jl} \cite{Checkpoiting_2023}.
\end{enumerate} 
Computing the ODE backwards can be unstable and lead to exponential errors \cite{kim_stiff_2021}. 
In \cite{chen_neural_2019}, the solution is recalculated backwards together with the adjoint simulating an augmented dynamics: 
\begin{equation}
    \frac{d}{dt}
    \begin{bmatrix}
       u \\
       \lambda \\
       \frac{dL}{d\theta}
    \end{bmatrix}
    = 
    \begin{bmatrix}
       -f \\
       - \lambda^T \frac{\partial f}{\partial u} \\
       - \lambda^T \frac{\partial f}{\partial \theta}
    \end{bmatrix}
    = 
    - [ 1, \lambda^T, \lambda^T ]
    \begin{bmatrix}
       f & \frac{\partial f}{\partial u} & \frac{\partial f}{\partial \theta} \\
       0 & 0 & 0 \\
       0 & 0 & 0
    \end{bmatrix},
\end{equation}
with initial condition $[u(t_1), \frac{\partial L}{\partial u(t_1)}, 0]$. One way of solving this system of equations that ensures stability is by using implicit methods. However, this requires cubic time in the total number of ordinary differential equations, leading to a total complexity of $\mathcal O((n+p)^3)$ for the adjoint method. Two alternatives are proposed in \cite{kim_stiff_2021}, the first one called \textit{Quadrature Adjoint} produces a high order interpolation of the solution $u(t)$ as we move forward, then solve for $\lambda$ backwards using an implicit solver and finally integrating $\frac{dL}{d\theta}$ in a forward step. This reduces the complexity to $\mathcal O (n^3 + p)$, where the cubic cost in the number of ODEs comes from the fact that we still need to solve the original stiff differential equation in the forward step. A second but similar approach is to use a implicit-explicit (IMEX) solver, where we use the implicit part for the original equation and the explicit for the adjoint. This method also will have complexity $\mathcal O (n^3 + p)$.

\section{Numerical implementation}
\label{sec:computational-implementation}
% Giles (2000) has a good discussion on this.
% \cite{ma_comparison_2021}. 

In this section, we are going to address how these different methods are computationally implemented and how to decide which method to use depending on the scientific task.
In order to address this point, it is important to make one further distinction of the methods introduced in Section \ref{section:methods} between those that apply direct differentiation at the algorithmic level or those that are based on numerical solvers.  
The first are easier to implement since they are agnostic with respect to the mathematical and numerical properties of the ODE; however, they tend to be either inaccurate, memory-expensive, or unfeasible for large models. 
The family of methods that are based on numerical solvers include the sensitivity equations and the adjoint methods, both discrete and continuous; they are more difficult to implement and for real case applications require complex software implementations, but they are also more accurate and adequate. 

\subsection{Direct methods}

Direct methods are implemented independent of the structure of the ODE and the numerical solver used to solve it. 
Finite differences are easy to implement manually, do not require much software support, and provide a direct way of approximating a gradient. 
In Julia, these methods are implemented in \texttt{FiniteDiff.jl} and \texttt{FiniteDifferences.jl} and it is recommended to use establish libraries than implementing it yourself, since these already include subroutines to determine step-sizes.
However, finite differences are less accurate and as costly as forward AD \cite{Griewack-on-AD} and complex-step differentiation. 
Figure \ref{fig:finite-diff} illustrates the error in computing the gradient of a simple loss function for both true analytical solution and numerical solution of a system of ODEs as a function of the stepsize $\varepsilon$ using finite differences and complex-step differentiation.

\begin{figure}[tbh]
    \centering
    \includegraphics[width=0.85\textwidth]{../code/finite_differences/finite_difference_derivative.pdf}
    \caption{Absolute relative error when computing the gradient of the function $u(t) = \sin (\omega t)/\omega$ with respect to $\omega$ at $t=10.0$ as a function of the stepsize $\varepsilon$. Here $u(t)$ corresponds to the solution of the differential equation $u'' + \omega^2 u = 0$ with initial condition $u(0)=0$ and $u'(0)=1$. The blue dots correspond to the case where this is computed with finite differences. The red and orange lines are for the case where $u(t)$ is numerically computed using the default Tsitouras solver \cite{Tsitouras_2011} from \texttt{OrdinaryDiffEq.jl} using different tolerances. The error when using a numerical solver is larger and it is dependent on the numerical precision of the numerical solver. }
    \label{fig:finite-diff}
\end{figure}

% \subsubsection{Forward-mode AD}

Implementing forward AD using dual numbers is usually carried out using \textit{operator overloading} \cite{Neuenhofen_2018}. 
This means expanding the object associated to a numerical value to include the tangent and extending the definition of atomic algebraic functions. 
In Julia, this can be done by relying on multiple dispatch \cite{Julialang_2017}. 
The following example illustrates how to define a dual number and its associated binary addition and multiplication extensions. 
\begin{jllisting}
using Base: @kwdef

@kwdef struct DualNumber{F <: AbstractFloat}
    value::F
    derivative::F
end

# Binary sum
Base.:(+)(a::DualNumber, b::DualNumber) = DualNumber(value = a.value + b.value, derivative = a.derivative + b.derivative)

# Binary product 
Base.:(*)(a::DualNumber, b::DualNumber) = DualNumber(value = a.value * b.value, derivative = a.value*b.derivative + a.derivative*b.value)
\end{jllisting}
We further overload base operations for this new type to extend the definition of standard functions by simply applying the chain rule and storing the derivative in the dual variable following Equation \eqref{eq:dual-number-function}:
\begin{jllisting}
function Base.:(sin)(a::DualNumber)
    value = sin(a.value)
    derivative = a.derivative * cos(a.value)
    return DualNumber(value=value, derivative=derivative)
end
\end{jllisting}
% With all these pieces together, we are able to propagate forward the value of a single-valued derivative through a series of algebraic operations. 
In the Julia ecosystem, \texttt{ForwardDiff.jl} implements forward mode AD with multidimensional dual numbers \cite{RevelsLubinPapamarkou2016}. 
Notice that a major limitation of the dual number approach is that we need a dual variable for each variable we want to differentiate. 
Incorrect implementations of this aspect can lead to \textit{perturbation confusion} \cite{siskind2005perturbation, manzyuk2019perturbation}, an existing problem in some AD software where dual variables corresponding to different variables result indistinguishable, specially in the case of nested functions \cite{manzyuk2019perturbation}.   

Implementations of forward AD using dual numbers and computational graphs require a number of operations that increases with the number of variables to differentiate, since each computed quantity is accompanied by the corresponding gradient calculations \cite{Griewack-on-AD}. 
This consideration also applies to the other forward methods, including finite differences and complex-step differentiation, which makes forward models inefficient when differentiating with respect to many parameters. 

Notice that both AD based in dual number and complex-step differentiation introduce an abstract unit ($\epsilon$ and $i$, respectively) associated with the imaginary part of the extender value that carries forward the numerical value of the gradient.
% This resemblance between the methods makes them susceptible to the same advantages and disadvantages: easiness of implementation with operator overloading; and inefficient scaling with respect to the number of variables to differentiate. 
Although these methods seem similar, it is important to remark that AD gives the exact gradient, while complex step differentiation relies on numerical approximations that are valid just when the stepsize $\varepsilon$ is small. 
The next example shows how the calculation of the gradient of $\sin (x^2)$ is performed by these two methods:
\begin{equation}
\renewcommand{\arraystretch}{1.5}
\begin{tabular}{@{} l @{\qquad} l @{\qquad} l @{}}
Operation & AD with Dual Numbers  & Complex Step Differentiation \\
$x$ & $x + \epsilon$    & $x + i \varepsilon$ \\
$x^2$ & $x^2 + \epsilon \, (2x)$  & $x^2 - \varepsilon^2 + 2i\varepsilon x$\\
$\sin(x^2)$  & $\sin(x^2) + \epsilon \, \cos(x^2) (2x)$ &
$\sin(x^2 - \varepsilon^2) \cosh (2i\varepsilon) + i \, \cos(x^2 - \varepsilon^2) \sinh (2i\varepsilon)$
\end{tabular}
\label{eq:AD-complex-comparision}
\end{equation}
While the second component of the dual number has the exact derivative of $\sin(x^2)$, it is not until we take $\varepsilon \rightarrow 0$ than we obtain the derivative in the imaginary component for the complex step method
\begin{equation}
    \lim_{\varepsilon \rightarrow 0} \, \frac{1}{\varepsilon} \, \cos(x^2 - \varepsilon^2) \sinh (2i\varepsilon) 
    = 
    \, \cos(x^2) (2x).
\end{equation}
The stepsize dependence of the complex step differentiation method makes it resemble more to finite differences than AD with dual numbers. 
This difference between the methods also makes the complex step method sometimes more efficient than both finite differences and AD \cite{Lantoine_Russell_Dargent_2012}, an effect that can be counterbalanced by the number of extra unnecessary operations that complex arithmetic requires (see last column of \eqref{eq:AD-complex-comparision}) \cite{Martins_Sturdza_Alonso_2003_complex_differentiation}.
It is also important to remark that many modern software already have support for complex number arithmetic, making complex step differentiation very easy to implement.

\subsubsection{Reverse-mode AD}
\label{sec:software-reverse-AD}

In opposition to finite differences, forward AD and complex-step differentiation, reverse AD is the only of these family of methods that propagates the gradient in backwards mode. 
Since this requires the evaluation intermediate variables, reverse AD requires a more delicate protocol of how to store intermediate variables in memory and make them accessible during the backwards pass. 

Backwards AD can be implemented via \textit{pullback} functions \cite{Innes_2018}, a method also known as \textit{continuation-passing style} \cite{Wang_Zheng_Decker_Wu_Essertel_Rompf_2019}.
In the backward step, this executes a series of function calls, one for each elementary operation.
If one of the nodes in the graph $w$ is the output of an operation involving the nodes $v_1, \ldots, v_m$, where $v_i \rightarrow w$ are all nodes in the graph, then the pullback $\bar v_1, \ldots, \bar v_m = \mathcal B_w(\bar w)$ is a function that accepts gradients with respect to $w$ (defined as $\bar w$) and returns gradients with respect to each $v_i$ ($\bar v_i$) by applying the chain rule. 
Consider the example of the multiplicative operation $w = v_1 \times v_2$. Then
\begin{equation}
 \bar v_1, \bar v_2 = v_2 \times \bar w , \quad
 v_1 \times \bar w = \mathcal{B}_w (\bar w),
\end{equation}
which is equivalent to using the chain rule as
\begin{equation}
 \frac{\partial \ell}{\partial v_1} = \frac{\partial}{\partial v_1}(v_1 \times v_2) \frac{\partial \ell}{\partial w}.
\end{equation}
% There are many libraries that implement reverse AD acroos programming languages \cite{Baydin_Pearlmutter_Radul_Siskind_2015}.
% In Julia, the libraries \texttt{ReverseDiff.jl} and \texttt{Zygote.jl} use pullbacks to compute gradients \cite{Innes_2018}. 
% When gradients are being computed with less than $\sim 100$ parameters, the former is faster (see documentation).


% \subsubsection{Further remarks}

A crucial distinction between AD implementations based on computational graphs is between static and dynamical methods\cite{Baydin_Pearlmutter_Radul_Siskind_2015}. 
In the case of a static implementations, the computational graph is constructed before any code is executed, which are encoded and optimized for performance within the graph language. 
For static structures such as neural networks, this is ideal \cite{abadi-tensorflow}. 
However, two major drawbacks of static methods are composability with existing code, including support of custom types, and adaptive control flow, which is a common feature of numerical solvers. 
These issues are addressed in reverse AD implementations using \textit{tracing}, where the program structure is transformed into a list of pullback functions that built the graph dynamically at runtime. 
Popular libraries in this category are \texttt{Tracker.jl} and \texttt{ReverseDiff.jl}.
There also exist source-to-source AD system that achive highest performance at the same time they support arbitrary control flow structure. 
These include \texttt{Zygote.jl}\cite{Innes_Zygote}, \texttt{Enzyme.jl}\cite{moses_Enzyme}, and \texttt{Difractor.jl}.
The existence of multiple AD packages lead to the development of \texttt{AbstractDifferentiation.jl} which allows to combine different methods \cite{Schäfer_Tarek_White_Rackauckas_2021}. 


% Since perturbation confusion also affects reverse mode AD, maybe is better to cover that here.

Notice that the application of reverse AD on a numerical solver (without checkpointing) scales as $\mathcal O (n k)$, with $k$ the number of steps of the numerical solver. 
Furthermore, when reverse AD is applied on the numerical solver, the step-size needs to be adapted to ensure the stability of the backward stesps, further increasing the computational complexity. 

\subsection{Solver-based methods}

Sensitivity methods based on numerical solvers tend to be better adapted to the structure and properties of the underlying ODE (stiffness, stability, accuracy) but also more difficult to implement.  
This difficulty arises from the fact that the sensitivity method needs to deal with some numerical and computational considerations, including how to handle matrix/Jacobian-vector products; numerical stability of the forward/backward solver; and memory-time tradeoff. 
These factors are further exacerbated by the number of ODEs and parameters in the model. 
% While explicit methods can be preferable for non-stiff problems, Rosenblock methods can be 
Just a few modern scientific software have the capabilities of handling ODE solvers and computing their sensitivities at the same time. 
The include \texttt{CVODES} within \texttt{SUNDIALS} in C \cite{serban2005cvodes, SUNDIALS-hindmarsh2005sundials}; \texttt{ODESSA} \cite{ODESSA} and \texttt{FATODE} (discrete adjoints) \cite{FATODE2014} both in Fortram; \texttt{SciMLSensitivity.jl} in Julia \cite{rackauckas2020universal}; \texttt{Dolfin-adjoint} based on the \texttt{FEniCS} Project \cite{dolfin2013, dolfin2018}. 

It is important to remark that the underlying machinery of all solvers relies on solvers for linear systems of equations, which can be solved in dense, band (sparse), and Krylow mode. 
% This implies that methods based on numerical solvers are, in principle, more difficult to implement but also more efficient in computing gradients for complex differential equations. 
Another important consideration is that all these methods have subroutines to compute the VJPs involved in the sensitivity and adjoint equations. 
This calculation is carried out by another sensitivity method (finite differences, AD) and this also plays a central role at the moment of analyzing the accuracy and stability of the adjoint method. 

\subsubsection{Sensitivity equation}

\subsubsection{Solving the adjoint}

% Distinctio between discrete and continuous
% Discrete are the exact gradient of the computer program, continuous are not. 
% There is no flexibilty in discrete adjoitns, but there is in continuous
% Human effort required to compute discrete adjoints is large \cite{FATODE2014}

An equally important consideration when working with adjoints is when these are numerically stable. 
Some works have shown that continuous adjoints can lead to unstable sensitivities \cite{Jensen_Nakshatrala_Tortorelli_2014}.
Implicit forward schemes can give rise to explicit backwards schemes, leading to unstable solutions for the gradient. 

\vspace*{10px}
\noindent \textbf{\textit{Solving the backwards mode}}
\vspace*{5px}

The bottleneck of this method is the calculation of the adjoint since in order to solve the adjoint equation we need to know $u(t)$ at any given time. 
Effectively, notice that the adjoint equation involves the terms $f(u, \theta, t)$ and $\frac{\partial h}{\partial u}$ which are both functions of $u(t)$. 
There are different ways of addressing the evaluation of $u(t)$ during the backwards step.
\begin{enumerate}[label=(\roman*)]
    \item \textbf{Dense Store.} During the forward model, we can just store in memory all the intermediate states of the numerical solution. 
    This leads to heavy-memory expensive algorithms. 
    \item \textbf{Re-solve.} Solve again the original ODE together with the adjoint as the solution of the reversed augmented system \cite{chen_neural_2019}
    \begin{equation}
    \frac{d}{dt}
    \begin{bmatrix}
       u \\
       \lambda \\
       \frac{dL}{d\theta}
    \end{bmatrix}
    = 
    \begin{bmatrix}
       -f \\
       - \frac{\partial f}{\partial u}^T \lambda - \frac{\partial h}{\partial u}^T \\
       - \lambda^T \frac{\partial f}{\partial \theta} - \frac{\partial h}{\partial \theta}
    \end{bmatrix}
    % = 
    % - [ 1, \lambda^T, \lambda^T ]
    % \begin{bmatrix}
    %    f & \frac{\partial f}{\partial u} & \frac{\partial f}{\partial \theta} \\
    %    0 & 0 & 0 \\
    %    0 & 0 & 0
    % \end{bmatrix},
    \qquad 
    \begin{bmatrix}
       u \\
       \lambda \\
       \frac{dL}{d\theta}
    \end{bmatrix}(t_1)
    = 
    \begin{bmatrix}
       u(t_1) \\
       \frac{\partial L}{\partial u(t_1)} \\
       \lambda(t_0)^T s(t_0)
    \end{bmatrix}.
    \end{equation}
    However, computing the ODE backwards can be unstable and lead to large numerical errors \cite{kim_stiff_2021, Zhuang_2020}. 
    \item \textbf{Checkpointing. } Also known as windowing, checkpointing is a technique that trade-offs memory and time by saving intermediate states of the solution in the forward pass and recalculating the solution between intermediate states in the backwards mode \cite{Checkpoiting_2023, griewank2008evaluatingderivatives}. 
    This is implemented in \texttt{Checkpointing.jl} \cite{Checkpoiting_2023}.
\end{enumerate} 

One way of solving this system of equations that ensures stability is by using implicit methods. 
However, this requires cubic time in the total number of ordinary differential equations, leading to a total complexity of $\mathcal O((n+p)^3)$ for the adjoint method.
Two alternatives are proposed in \cite{kim_stiff_2021}, the first one called \textit{Quadrature Adjoint} produces a high order interpolation of the solution $u(t)$ as we move forward, then solve for $\lambda$ backwards using an implicit solver and finally integrating $\frac{dL}{d\theta}$ in a forward step.
This reduces the complexity to $\mathcal O (n^3 + p)$, where the cubic cost in the number of ODEs comes from the fact that we still need to solve the original stiff differential equation in the forward step. 
A second but similar approach is to use an implicit-explicit (IMEX) solver, where we use the implicit part for the original equation and the explicit for the adjoint. 
This method also will have complexity $\mathcal O (n^3 + p)$.

\vspace*{10px}
\noindent \textbf{\textit{Solving the quadrature}}
\vspace*{5px}

Another computational consideration is how the integral in Equation \eqref{eq:casa-final-loss-gradient} is numerically evaluated. 
Some methods save computation by noticing that the last step in the continuous adjoint method of evaluating $\frac{dL}{d\theta}$ is an integral instead of an ODE, and then can be evaluated as such without the need to include it in the tolerance calculation inside the numerical solver \cite{that-is-not-an-ode}.
Numerical integration, also known as quadrature integration, consists in approximating integrals by finite sums of the form
Numerical solutions of the integral 
\begin{equation}
    \int_{t_0}^{t_1} 
    F(t) dt
    \approx
    \sum_{i=1}^K \omega_i \, F(\tau_i),
\end{equation}
where the evaluation of the function occurs in certain knots $t_0 \leq \tau_1 < \ldots < \tau_K \leq t_1$, and $\omega_i$ are weights. 
Weights and knots are obtained in order to maximize the order in which polynomials are exactly integrated \cite{stoer2002-numerical}. 

Different quadrature methods are based on different choices of the knots and associated weights.
Between these methods, the Gaussian quadrature is the faster method to evaluate one-dimensional integrals \cite{Norcliffe_gaussquadrature_2023}.

\subsubsection{Computing VJPs}

All the methods analized in this section need to deal with the calculation of VJPs.  
The choice of the specific algorithm to compute VJPs can have significant impact in the overall performance of the sensitivity method. 

In SUNDIALS, the VJPs involved in the sensitivity and adjoint method are handled using finite differences unless specified by the user \cite{SUNDIALS-hindmarsh2005sundials}.
In FATODE, these can be computed with finite differences, AD or provides by the user.

In the Julia ecosystem, different AD packages are available for this task (see Section \ref{sec:software-reverse-AD}), including \texttt{ForwardDiff.jl}, \texttt{ReverseDiff.jl}, \texttt{Zygote.jl}\cite{Innes_Zygote}, \texttt{Enzyme.jl}\cite{moses_Enzyme}, \texttt{Tracker.jl}.


% \section{Extensions}

\section{Recommendations}
For sufficient small systems of less than 100 parameters and ODEs, Forward AD is the most efficient method, outperforming sensitivity and adjoint methods \cite{ma_comparison_2021}.
% \todo[inline]{We should pay attention that this section does not overlap with \cite{ma_comparison_2021}.}

% Mention to limit cases when working with chaotic systems
\subsection{Chaotic systems}

% Intro: why we take this average quantities in chaotic system? Need to motivate this in a sentense.
Both forward and adjoint sensitivity analysis methods from the previous chapters encounter challenges and become less useful when applied to chaotic systems.
To illustrate this, let us consider long-time-averaged quantities 
\begin{equation}\label{eq:long_time_averaged_quantities}
    \langle L(\theta) \rangle_T = \frac{1}{T} \int_0^T L(u(t), \theta) \, dt, 
\end{equation}
of chaotic systems, where $L(u(t), \theta)$ is the instantaneous objective and $u(t)$ denotes the state of the dynamical system at time $t$.
For ergodic dynamical systems, $\lim_{T\to\infty} \langle L(\theta) \rangle_T$ depends solely on the governing dynamical system and is independent of the specific choice of trajectory $u(t)$. 
In particular, $\lim_{T\to\infty} \langle L(\theta) \rangle_T$ does not depend on the initial condition. 
Under the assumption of uniform hyperbolic systems, it is possible to derive closed-form expressions and differentiability conditions for $ \langle L(\theta) \rangle_T$\cite{ruelle1997differentiation,ruelle2009review}. 
However, computing derivatives using numerical methods of statistical quantities of the form \eqref{eq:long_time_averaged_quantities} with respect to the vector parameter $\theta$ in chaotic dynamical systems remains challenging due to the \textit{butterfly effect}, i.e. small changes in the initial state or parameter can result in large differences in a later state. 
As a consequence, the solutions of the forward and adjoint sensitivity equations blow up (exponentially fast) instead of converging to the actual derivative.
To address these issues, various modifications and methods have been proposed, including approaches based on ensemble averages~\cite{lea2000sensitivity, eyink2004ruelle}, the Fokker-Planck equation~\cite{thuburn2005climate, blonigan2014probability}, the fluctuation-dissipation theorem~\cite{leith1975climate, abramov2007blended, abramov2008new}, shadowing lemma~\cite{wang2013forward, wang2014least, wang2014convergence, ni2017sensitivity, blonigan2017adjoint, blonigan2018multiple, ni2019adjoint, ni2019sensitivity}, and modifications of Ruelle's formula~\cite{chandramoorthy2022efficient, ni2020fast}.

% \section{Conclusions}
% In the present work, we presented a comprehensive overview of the different existing methods for calculating the sensitivity or gradients of forward maps involving numerical solutions of differential equations.
This task has been approached from three different angles.
First, we presented the existing literature in different scientific communities where adjoints and sensitivities have been used before and play a central modelling role, especially for inverse modeling.
Secondly, we reviewed the mathematical foundations of these methods and their classification as forward-reverse and discrete-continuous.
We further compare the mathematical foundations of these methods, which we believe enlightens the discussion on sensitivity methods and helps to demystify  misconceptions around the sometimes apparent differences between methods.  
Then, we have shown hot these different methods can be translated to different software implementations, evaluating different considerations that we must take into account when implementing or using a sensitivity algorithm. 
We further exemplified how these methods are implemented in the Julia programming language. 

There exists a myriad of options and combinations to compute sensitivities of functions involving differential equations, further complicated by jargon and scientific cultures of different communities. 
We hope this review paper provides a clearer overview on this topic, and can serve as an entry point to navigate this field and guide researchers in choosing the most appropriate method for their scientific application.

Differentiable programming is opening new ways of doing research across different domains of science and engineering. 
Arguably, its potential has so far been under-explored but is being rediscovered in the age of data-driven science. 
Realizing its full potential, requires collaboration between domain scientists, methodological scientists, computational scientists, and computer scientists in order to develop successful, scalable, practical, and efficient frameworks for real world applications.
As we make progress in the use of these tools, new methodological questions emerge. 
How do these methods compare? How can they be improved? 
In this review we present a comprehensive list of methods that exists at the intersection of differentiable programming and differential equation modelling. 

% We also encourage the interested reader to direct their attention to other comprehensive works on automatic differentiation \cite{Baydin_Pearlmutter_Radul_Siskind_2015}; adjoint methods, ...

% \section{Do we need full gradients?}

% \section{Notation}
% \setlength{\tabcolsep}{10pt} % Default value: 6pt
\renewcommand{\arraystretch}{1.2}
\begin{table}[H]
    \center
    \begin{tabular}{| p{3cm} | p{8cm} |}
    \hline
    Variable            & Meaning \\ [0.5ex] 
    \hline  
    $u$					& Solution of the differential equation \\
    $\theta$            & Parameters of the model \\
    $L$ 				& Loss function \\
    $s$                 & Sensitivity of the solution given by $\frac{\partial u}{\partial \theta}$ \\
    $n$                 & Number of ODEs \\
    $p$                 & Number of parameters \\
    \hline
\end{tabular}	
\end{table}


\newpage
\appendix
\section*{Appendices}
\addcontentsline{toc}{section}{Appendices}
\renewcommand{\thesubsection}{\Alph{subsection}}

\subsection{Lagrangian derivation of adjoints}
\label{appendix:lagrangian}

In this section we are going to derive the adjoint equation for both discrete and continuous methods using the Lagrange multiplier trick. 
Conceptually, the method is the same in both discrete and continuous case, with the difference that we manipulate linear algebra objects for the former and continuous operators for the later. 

For the continuous adjoint method, we proceed the same way by writing a new loss function $I(\theta)$, sometimes known as the \textit{Lagrangian}, identical to $L(\theta)$ as 
\begin{equation}
    I(\theta) = L(\theta) - \inttime \lambda(t)^T \left( \frac{du}{dt} - f(u, \theta, t) \right) dt
\end{equation}
where $\lambda(t) \in \mathbb R^n$ is the Lagrange multiplier of the continuous constraint defined by the differential equation. Now, 
\begin{equation}
    \frac{dL}{d\theta} = \frac{dI}{d\theta} = 
    \inttime \left( \frac{\partial h}{\partial \theta} + \frac{\partial h}{\partial u} \frac{\partial u}{\partial \theta} \right) dt
    - 
    \inttime \lambda(t)^T \left( \frac{d}{dt} \frac{du}{d\theta} - \frac{\partial f}{\partial u} \frac{du}{d\theta} - \frac{\partial f}{\partial \theta} \right) dt.
\end{equation}
Notice that the term involved in the second integral is the same we found when deriving the sensitivity equations. 
We can derive an easier expression for the last term using integration by parts. 
Using our usual definition of the sensitivity $s = \frac{du}{d\theta}$, and performing integration by parts in the term $\lambda^T \frac{d}{dt} \frac{du}{d\theta}$ we derive 
\begin{multline}
    \frac{dL}{d\theta}
    = 
    \inttime \left( \frac{\partial h}{\partial \theta} + \lambda^T \frac{\partial f}{\partial \theta} \right) dt 
    - 
    \inttime \left( - \frac{d\lambda^T}{dt} - \lambda^T \frac{\partial f}{\partial u} - \frac{\partial h}{\partial u} \right) \, s(t) \, dt \\
    -
    \bigg ( \lambda(t_1)^T s(t_1) - \lambda(t_0)^T s(t_0) \bigg ).
    \label{eq:continous-adjoint-loss}
\end{multline}
Now, we can force some of the terms in the last equation to be zero by solving the following adjoint differential equation for $\lambda(t)^T$ in backwards mode
\begin{equation}
    \frac{d\lambda}{d\theta} = - \left(\frac{\partial f}{\partial u}\right)^T \lambda - \left( \frac{\partial h}{\partial u} \right)^T,
    \label{eq:continuous-adjoint}
\end{equation}
with final condition $\lambda(t_1) = 0$. 

It is easy to see that this derivation is equivalent to solving the Karush-Kuhn-Tucker (KKT) conditions. 

\subsection{Supplementaty code}
% Appendix to mentioned the code provided in the GithUb repository.

% \section{Glossary} 

\newpage

\printbibliography[heading=bibintoc, title={References}]

\end{document}