The adjoint method is a very popular approach to compute the gradients of a loss function by first computing an intermediate variable (the adjoint) that serves as a bridge between the solution of the ODE and the final sensitivity. 
There is a large family of adjoint methods that a first order we can classify them between discrete and continuous adjoints. 
The former usually arises as the numerical discretization of the later, and when the discrete adjoint method is a consistent estimator of the continuous adjoint depends of the ODE and equation.  
Proofs of the consistency of discrete adjoint methods for Runge-Kutta methods had been provided in \cite{sandu2006properties, sandu2011solution}.
Depending the choice of the Runge-Kutta coefficients, we can have a numerical scheme that is both consistent for the original equation and consistent/inconsistent for the adjoint \cite{Hager_2000}.
An equal important consideration when working with adjoints is when these are numerically stable. 