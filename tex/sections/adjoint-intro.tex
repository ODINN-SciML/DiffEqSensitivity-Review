The first works on adjoint methods for sensitivity analysis can be traced back to ... and they had been broadly applied in the computational sciences. 
In the context of differential equations, the idea of the adjoint method is to treat the differential equation as a constrained and then differentiate an objective function subject to that constraint. 
Mathematically speaking, this can be treated both from a duality or Lagrangian perspective \cite{Giles_Pierce_2000}.
We prefer to derive the equation using the former since we believe it gives better insights to how the method works and also allow to generalize to other user cases. 
The derivation of adjoint methods using Lagrangian multipliers can be found in Appendix \ref{appendix:lagrangian}.

There is a large family of adjoint methods that in first order can be classified as discrete and continuous adjoints. 
The difference between these is that the former follows follows the discretize-differentiate approach (also known as finite difference of adjoints \cite{Sirkes_Tziperman_1997}, ...), meaning that we first discretize the system of continuous ODEs before applying any sensitivity strategy to compute gradients \cite{Giles_Pierce_2000, allaire2015review}. 
On the contrary, continuous adjoint equations are derived by directly operating on the differential equation, without a priori consideration of the numerical scheme used to solve it. 