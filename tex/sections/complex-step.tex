An alternative to finite differences that avoids rounding errors is based on complex variable analysis. 
The first proposals originated in 1967 using the Cauchy integral theorem involving the numerical evaluation of a complex-valued integral \cite{Lyness_1967, Lyness_Moler_1967}.
A new approach recently emerged that uses the Taylor expansion of a function to define its complex generalization \cite{Squire_Trapp_1998_complex_diff, Martins_Sturdza_Alonso_2003_complex_differentiation}. 
Assuming that we have one single scalar parameter $\theta \in \R$, then the function $L(\theta)$ can be expanded as 
the Taylor expansion
\begin{equation}
    L(\theta + i \varepsilon)
    = 
    L(\theta) + i \varepsilon L'(\theta) 
    - 
    \frac 1 2
    L''(\theta) \varepsilon^2
    + 
    \mathcal O (\varepsilon^3),
\end{equation}
where $i$ is the imaginary unit satisfying $i^2 = -1$. 
Computing the imaginary part $\text{Im}(L(\theta + i \varepsilon))$ leads to
\begin{equation}
    L'(\theta) 
    = 
    \frac{\text{Im}(L(\theta + i \varepsilon))}{\varepsilon}
    + 
    \mathcal{O} (\varepsilon^2)
\end{equation}
The method of \textit{complex step differentiation} consists then in estimating the gradient as $\text{Im}(L(\theta + i \varepsilon)) / \varepsilon$ for a small value of $\varepsilon$. 
Besides the advantage of being a method with precision $\mathcal{O}(\varepsilon^2)$, the complex step method avoids subtracting cancellation error and then the value of $\varepsilon$ can be reduced to almost machine precision error without affecting the calculation of the derivative. 
Extension to higher order derivatives can be done by introducing multicomplex variables \cite{Lantoine_Russell_Dargent_2012}. 

