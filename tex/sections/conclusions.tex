In the present work, we presented a comprehensive overview of the different existing methods for calculating the sensitivity or gradients of forward maps involving numerical solutions of differential equations.
This task has been approached from three different angles.
First, we presented the existing literature in different scientific communities where adjoints and sensitivities have been used before and play a central modelling role, especially for inverse modeling.
Secondly, we reviewed the mathematical foundations of these methods and their classification as forward-reverse and discrete-continuous.
Then, we have shown how the different methods are implemented in the Julia programming language and the different computational considerations that we must take into account when implementing or using a sensitivity algorithm.
There exist a myriad of options and combinations to compute sensitivities of functions involving differential equations, further complicated by jargon and scientific cultures of different communities. We believe this review paper provides a clearer overview on this topic, and can serve as an entry point to navigate this field and help in terms of decision-making of the most suitable methods for many different scientific problems. 


% We also encourage the interested reader to direct their attention to other comprehensive works on automatic differentiation \cite{Baydin_Pearlmutter_Radul_Siskind_2015}; adjoint methods, ...