% General statement: why gradients are important?
Evaluating how the value of a function changes with respect to its arguments and parameters plays a central role in optimization, sensitivity analysis, Bayesian inference, uncertainty quantification, among many. 
Modern machine learning applications require the use of gradients to explore and exploit more efficiently the space of parameters. 
When optimizing a loss function, gradient-based methods (for example, gradient descent and its many variants \cite{ruder2016overview-gradient-descent}) are more efficient at finding a minimum and converge faster to them than than gradient-free methods.
When numerically computing the posterior of a probabilistic model, gradient-based sampling strategies converge faster to the posterior distribution than gradient-free methods. 
Second derivatives further help to improve convergence rates of these algorithms and enable uncertainty quantification around parameter values.
\textit{A gradient serves as a compass in modern data science: it tells us in which direction in the open wide ocean of parameters we should move towards in order to increase our chances of success}.  

% Differential Programming
Dynamical systems governed by differential equations are not an exception to the rule.
Differential equations play a central role in describing the behaviour of systems in natural and social sciences. 
Some authors have recently suggested differentiable programming as the bridge between modern machine learning methods and traditional scientific models \cite{Ramsundar_Krishnamurthy_Viswanathan_2021, Shen_diff_modelling, Gelbrecht-differential-programming-Earth}. 
Being able to compute gradients and sensitivities of dynamical systems opens the door to more complex models.
This is very appealing in geophysical models, where there is a broad literature on physical models and a long tradition in numerical methods. 
The first goal of this work is to introduce some of the applications of this emerging technology and to motivate its incorporation for the modelling of complex systems in the natural and social sciences. 
\begin{quote}
    \textbf{Question 1. }
    \textit{What are the scientific applications of differentiable programming for complex dynamical systems?}
\end{quote}

% Some examples
Sensitivity analysis correspond to any method aiming to calculate the how much the output of a function or program changes when we vary one of the model parameters. 
This task is perform in different ways by different communities when working with dynamical systems. 
In statistics, the sensitivity equations enable the computation of gradients of the likelihood of the model with respect to the parameters of the dynamical system, which can be later used for inference \cite{ramsay2017dynamic}. 
In numerical analysis, sensitivities quantify how the solution of a differential equation fluctuates with respect to certain parameters. 
This is particularly useful in optimal control theory \cite{Giles_Pierce_2000}, where the goal is to find the optimal value of some control (e.g. the shape of a wing) that minimizes a given loss function. 
In recent years, there has been an increasing interest in designing machine learning workflows that include constraints in the form of differential equations. 
Examples of this include Physics-Informed Neural Networks (PINNs) \cite{PINNs_2019} and Universal Differential Equations (UDEs) \cite{rackauckas2020universal}.  
Furthermore, numerical solvers are used as forward models in the case of Neural ordinary differential equations \cite{chen_neural_2019}.

% soft / hard constrains

% Differentiation 
However, when working with differential equations, the computation of gradients is not an easy task, both regarding the mathematical framework and software implementation involved. 
Except for a small set of particular cases, most differential equations require numerical methods to calculate their solution and cannot be differentiated analytically. 
This means that solutions cannot be directly differentiated and require special treatment if, besides the numerical solution, we also want to compute first or second order derivatives. 
Furthermore, numerical solutions introduce approximation errors. 
These errors can be propagated and amplified during the computation of the gradient. 
Alternatively, there is a broad literature in numerical methods for solving differential equations. 
Although each method provides different guarantees and advantages depending on the use case, this means that the tools developed to compute gradients when using a solver need to be universal enough in order to be applied to all or 
at least to a large set of them. 
The second goal of this article is to review different methods that exist to achieve this goal.
\begin{quote}
    \textbf{Question 2. }
    \textit{How can I compute the gradient of a function that depends on the numerical solution of a differential equation?}
\end{quote}
%Notice here the phrase \textit{the gradient of a function that depends}, emphasizing the fact that in many cases we may be interested in computing the gradient of a function that depends on the solution of the differential equation.
%This is certainly the case in machine learning and optimization where the goal is to minimize a loss function that depends of some predicted and target responses. 

% AD
The broader set of tools known as Automatic Differentiation (AD) aims at computing derivatives by sequentially applying the chain rule to the sequence of unit operations that constitute a computer program. 
The premise is simple: every computer program, including a numerical solver, is ultimately an algorithm described by a chain of simple algebraic operations (addition, multiplication) that are easy to differentiate and their combination is easy to differentiate by using the chain rule. 
Although many modern differentiation tools use AD to some extend, there is also a family of methods that compute the gradient by relying on a auxiliary set of differential equations. 
We are going to refer to this family of methods as \textit{continuous}, and we will dedicate them a special treatment in future sections to distinguish them from the discrete algorithms that resemble more to pure AD. 

The differences between methods arise both from their mathematical formulation and their computational implementations. 
The first provides different guarantees on the method returning the actual gradient or a good approximation of it. 
The second involves how theory is translated to software, and what are the data structures and algorithms used to implement it. 
Different methods have different computational complexities depending on the number of parameters and differential equations, and these complexities are also balanced between total execution time and required memory. 
The third goal of this work is then to illustrate the different strengths and weaknesses of these methods, and how to use them in modern scientific software. 
\begin{quote}
    \textbf{Question 3. }
    \textit{What are the advantages and disadvantages of different differentiation methods and how can I incorporate them in my research?}
\end{quote}
Despite the fact that these methods can be (in principle) implemented in different programming languages, here we decided to use the Julia programming language for the different examples. 
Julia is a recently new but mature programming language that has already a large tradition in implementing packages aiming to advance differentiable programming \cite{Julialang_2017}. 

% The need to introduce all this methods in a common framework
Without aiming at making an extensive and specialized review of the field, we believe this study will be useful to other researchers working on problems that combine optimization and sensitivity analysis with differential equations.
Differentiable programming is opening new ways of doing research across sciences. 
As we make progress in the use of these tools, new methodological questions start to emerge. 
How do these methods compare? How can their been improved? 
We also hope this paper serves as a gateway to new questions regarding advances in these methods. 

% \begin{quote}
%     \textbf{Question 3. }
%     \textit{Are there opportunities for developing new sensitivity methods?}
% \end{quote}