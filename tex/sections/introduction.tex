% General statement: why gradients are important?
Evaluating how the value of a function changes with respect to its arguments and parameters plays a central role in optimization, sensitivity analysis, Bayesian inference, inverse methods, and uncertainty quantification, among many\cite{Razavi.2021}. 
Modern machine learning applications require the use of gradients to efficiently exploit the high-dimensional space of parameters to be inferred or learned (e.g., the weights of a neural network). 
When optimizing an objective function, gradient-based methods (for example, gradient descent and its many variants \cite{ruder2016overview-gradient-descent}) are more efficient and converge faster than gradient-free methods.
Furthermore, the \textit{curse of dimensionality} renders gradient-free optimization and sampling methods computationally intractable for most large-scale problems \cite{Oden:2010tv}.
When numerically computing the posterior of a probabilistic model, gradient-based sampling strategies tend to converge faster to the posterior distribution than gradient-free methods. 
Hessians further help to improve the convergence rates of these algorithms and enable uncertainty quantification around parameter values \cite{BuiThanh:2012ul}.
\textit{A gradient serves as a compass in modern data science: it tells us in which direction in the vast, open ocean of parameters we should move towards in order to increase our chances of success}.  

% Differential Programming
Models based in differential equations are not an exception to the rule\cite{Ghattas.2021}.
Differential equations arising in simulation-based science play a central role in describing the behaviour of systems in natural and social sciences. 
Some authors have recently suggested differentiable programming as the bridge between modern machine learning methods and traditional scientific models \cite{Ramsundar_Krishnamurthy_Viswanathan_2021, Shen_diff_modelling, Gelbrecht-differential-programming-Earth}. 
Being able to compute gradients or sensitivities of dynamical systems opens the door to more complex models.
This is very appealing in fields like computational physics, geophysics, and biology ,to mention a few, where there is a broad literature on physical models and a long tradition in numerical methods. 
The first goal of this work is to introduce some of the applications of this emerging technology and to motivate its incorporation for the modelling of complex systems in the natural and social sciences. 
\begin{quote}
    \textbf{Question 1. }
    \textit{What are the scientific applications of differentiable programming for complex dynamical systems?}
\end{quote}

% Some examples
Sensitivity analysis corresponds to any method aiming to calculate how much the output of a function or program changes when we vary one of the function (or model) parameters. 
This task is performed in different ways by different communities when working with dynamical systems. 
In statistics, the sensitivity equations enable the computation of gradients of the likelihood of the model with respect to the parameters of the dynamical system, which can be later used for inference \cite{ramsay2017dynamic}. 
In numerical analysis, sensitivities quantify how the solution of a differential equation fluctuates with respect to certain parameters. 
This is particularly useful in optimal control theory \cite{Giles_Pierce_2000}, where the goal is to find the optimal value of some control (e.g. the shape of a wing) that minimizes a given loss function. 
In recent years, there has been an increasing interest in designing machine learning workflows that include constraints in the form of differential equations. 
Examples of this include methods that numerically solve differential equations, such as physics-informed neural networks \cite{PINNs_2019}, as well as methods that augment and learn parts of the differential equation, such as universal differential equations \cite{rackauckas2020universal, Dandekar_2020}, which also includes the case of neural ODEs \cite{chen_neural_2019} and neural SDEs \cite{li2020scalable}.
% Examples of this include Physics-Informed Neural Networks (PINNs) \cite{PINNs_2019} and Universal Differential Equations (UDEs) \cite{rackauckas2020universal}.  
Furthermore, numerical solvers are used as forward models in the case of neural ordinary differential equations \cite{chen_neural_2019}.

% soft / hard constrains

% Differentiation 
However, when working with differential equations, the computation of gradients is not an easy task, both regarding the mathematical framework and software implementation involved. 
Except for a small set of particular cases, most differential equations require numerical methods to calculate their solution and cannot be differentiated analytically. 
This means that solutions cannot be directly differentiated and require special treatment if, besides the numerical solution, we also want to compute first or second-order derivatives. 
Furthermore, numerical solutions introduce approximation errors. 
These errors can be propagated and amplified during the computation of the gradient. 
Alternatively, there is a broad literature on numerical methods for solving differential equations. 
Although each method provides different guarantees and advantages depending on the use case, this means that the tools developed to compute gradients when using a solver need to be universal enough in order to be applied to all or 
at least to a large set of them. 
The second goal of this article is to review different methods that exist to achieve this goal.
\begin{quote}
    \textbf{Question 2. }
    \textit{How can I efficiently compute the gradient of a function that depends on the numerical solution of a differential equation?}
\end{quote}
%Notice here the phrase \textit{the gradient of a function that depends}, emphasizing the fact that in many cases we may be interested in computing the gradient of a function that depends on the solution of the differential equation.
%This is certainly the case in machine learning and optimization where the goal is to minimize a loss function that depends of some predicted and target responses. 

% AD
The broader set of tools known as Automatic (or Algorithmic) Differentiation (AD) aims at computing derivatives by sequentially applying the chain (and product) rule to the sequence of unit operations that constitute a computer program\cite{Griewank:2008kh, Naumann.2011}. 
The premise is simple: every computer program, including a numerical solver, is ultimately an algorithm described by a chain of elementary algebraic operations (e.g. addition, multiplication) that are easy to differentiate and their composition is easy to differentiate by using the chain rule\cite{Giering:1998in}. 
More broadly than AD, differentiable programming encapsulates a set of software tools that allows to compute efficient and robust gradients though complex algorithms, including numerical solvers \cite{Innes_Zygote}. 
Although many modern differentiation tools use AD to some extent, there is also a family of methods that compute the gradient by relying on an auxiliary set of differential equations and/or compute an intermediate adjoint. 
% We are going to refer to this family of methods as \textit{continuous}, and we will dedicate them a special treatment in future sections to distinguish them from the discrete algorithms that resemble more to pure AD. 
%We emphasize that AD is merely a \emph{tool} for generating derivatives of computer code. Alternatively, such code can be (and often has been) generated ``by hand'', e.g., to avoid restrictions incurred by the available AD tool.
%Down-sides of hand-written derivative code are, (i) it is error-prone, (ii) it is difficult to keep pace with the development of the parent code, and (iii) [I need to remember ...].

The differences between methods arise both from their mathematical formulation and their computational implementation. 
The first provides different guarantees on the method returning the actual gradient or a good approximation thereof. 
The second involves how theory is translated to software, and what are the data structures and algorithms used to implement it. 
Different methods have different computational complexities depending on the number of parameters and differential equations, and these complexities are also balanced between total execution time and required memory. 
As coined by Uwe Naumann, \textit{the automatic generation of optimal (in terms of robustness and efficiency) adjoint versions of large-scale simulation code is one of the great open challenges in the field of High-Performance Scientific Computing}\cite{Naumann.2011}.
The third goal of this work, then, is to illustrate the different strengths and weaknesses of these methods, and how to use them in modern scientific software. 
\begin{quote}
    \textbf{Question 3. }
    \textit{What are the advantages and disadvantages of different differentiation methods and how can I incorporate them in my research?}
\end{quote}

% The need to introduce all these methods in a common framework
% Without aiming at making an extensive and specialized review of the field, we believe this study will be useful to other researchers working on problems that combine optimization and sensitivity analysis with differential equations.
Differentiable programming is opening new ways of doing research across sciences. Arguably, its potential has so far been under-explored but is being rediscovered in the age of data-driven science. In order to realize its full potential, we need close collaboration between domain scientists, methodological scientists, computational scientists, and computer scientists in order to develop successful, scalable, practical, and efficient frameworks for real world applications.
As we make progress in the use of these tools, new methodological questions start to emerge. 
How do these methods compare? How can they be improved? 
We also hope this paper serves as a gateway to new questions regarding advances in these methods. 

% \begin{quote}
%     \textbf{Question 3. }
%     \textit{Are there opportunities for developing new sensitivity methods?}
% \end{quote}