% General statement: why gradients are important?
Evaluating the changes of a given function as we change the values of its arguments plays a central role in optimization, sensitivity analysis, Bayesian inference, uncertainty quantification, among many. 
Modern machine learning applications require the use of gradients to exploit more efficiently the space of parameters. 
When optimizing a loss function, gradient-based methods (for example, gradient descent and its many variants \cite{ruder2016overview-gradient-descent}) are more efficient at finding minimal and converge faster to them than than gradient-free methods.
When numerically computing the posterior of a probabilistic model, gradient-based sampling strategies converge faster to the posterior distribution than gradient-free methods. 
Second derivatives further help to improve convergence rates of these algorithms and allow uncertainty quantification around parameter values.
A gradient serves as a compass in modern data science: it tell us where in the open vast ocean of parameter combinations we should move in order to increase our changes of success.  

% Differential equations
Dynamical systems where the goal is to model observations governed by differential equations is not an exception to the rule.
In the field of statistics, the sensitivity equations provide a way to compute gradients of the likelihood of the model with respect to the parameters of the dynamical system, which can be used for inference \cite{ramsay2017dynamic}. 
In numerical analysis, sensitivities are used to understand how fluctuating is the solution of a differential equation with respect to certain parameter. 
In recent years, there has been an increasing interest in designing machine learning pipelines that include constraints to the physical system in the form of differential equations. 
Examples of this include physics-informed neural networks (PINNs) \cite{PINNs_2019} and universal differential equations (UDEs) \cite{rackauckas2020universal}.  
% soft / hard constrains

% Differentiation 
However, when working with differential equations the computation of gradients is not an easy task, both regarding the mathematical framework and software implementation involved. 
Except for a small set of particular cases, most differential equations require numerical methods to estimate their solution. 
This means that solutions cannot be directly differentiated and require special treatment if, besides the numerical solution, we want to extract first or second order derivatives. 
Furthermore, numerical solutions introduce approximation errors. 
These errors can be propagated and even amplified during the computation of the gradient. 
On the other side, there is a broad literature in numerical methods for solving differential equations. 
If well each method provides different guarantees and advantages depending of the use case, this means that the tools developed to compute gradients when using a solver need to be universal enough in order to apply to all or at least a large set of these. 
The first goal of this article is making a review of the different methods that exists to archive this goal.
\begin{quote}
    \textbf{Question 1. }
    \textit{How can we compute the gradient of a function that depends on the numerical solution of a differential equation?}
\end{quote}
Notice here the phrasing \textit{the gradient of a function that depends}, emphasizing the fact that in many cases we may be interested in computing the gradient of a function that depends on the solution of the differential equation.
This is certainly the case in machine learning and optimization where the goal is to minimize a loss function that depends of some predicted and target responses. 

% AD
The broader set of tools known as Automatic Differentiation (AD) aims to compute derivatives by sequentially applying the chain rule to the sequence of unit operations that constitute a computer program. 
The premise is simple: every computer program, including a numerical solver, is ultimately an algorithm described by a chain of simple algebraic operations (addition, multiplication) that are i) easy to differentiate and ii) their combination is easy to differentiate by using the chain rule. 
If well many modern applications use AD at some extend, there is also a family of methods that compute the gradient by relying in a secondary set of differential equations. 
We are going to refer to this family of methods as \textit{continuous}, and we will dedicate them a special treatment in future sections to distinguish them from the discrete algorithms that resemble more to pure AD. 

The difference between the different methods is set by both their mathematical formulation and their computational implementations. 
The first serves provides different guarantees that the method is actually returning the gradient or a good approximation of it. 
The second involves how the theory is translated to software, and what are the data structures we use to implemented. 
Different methods have different computational complexities depending the number of parameters and differential equations, and these complexities are also balanced between total execution time and required memory. 
The second goal of this work is then to illustrate the different advantages and limitations of this methods, and how to use them in modern scientific software. 
\begin{quote}
    \textbf{Question 2. }
    \textit{What are the pros and cons of these methods and how can I implement them in my research?}
\end{quote}
If well these methods can be (in principle) implemented in different programming languages, here we decided to use the Julia programming language for the different examples. 
Julia is a recently new but mature programming language that has already a large tradition in implementing packages aiming to advance differential programming \cite{Julialang_2017}. 

% The need to introduce all this methods in a common framework
Without aiming at making an extensive and specialized review on the field, we find this resource useful for other researchers and students working on problems that combine optimization and sensitivity analysis with differential equations.
Differential programming is opening new ways of doing research across sciences. 
As we make progress in the use of these tools, new methodological questions start to emerge. 
How these methods compare? Can their been improved? 
We then also hope this paper serves as a gate to new question regarding new advents in these methods. 
\begin{quote}
    \textbf{Question 3. }
    \textit{Are there opportunities for developing new methods?}
\end{quote}


%\subsection{Use cases}