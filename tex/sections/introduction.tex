% Introduction paragram with model based 
Models based on differential equations (DEs), including ordinary differential equations (ODEs) and partial differential equations (PDEs), play a central role in describing the behaviour of dynamical systems in applied, natural, and social sciences.
% Differential equations are the cornerstone of the physical sciences.
For instance, DEs are used for the modelling of the dynamics of the atmospheric and ocean circulation in climate science, for the modelling of ice or mantle flow in solid Earth geophysics, for the modelling of the spatio-temporal dynamics of species abundance in ecology, and for the modelling of cancer cell evolution in biology.  
For centuries, scientists have relied on theoretical and analytical methods to solve DEs. 
Allowing to approximate the solutions of large, nonlinear DE-based models, numerical methods and computers have lead to fundamental advances in the understanding and prediction of physical, biological, and social systems, among others \cite{Dahlquist_1985, hey2009, Rude:2018jv}.

% In a more abstract sense, the solution of differential equations can be seen as functions that map model parameters parameter and initial conditions to state variables and observations, similar to forward models, statistics, and even machine learning. 

% General statement: why gradients are important?
Quantifying how much the output of a DE-based model changes with respect to its input parameters is fundamental to many scientific computing and machine learning applications, including optimization, sensitivity analysis, Bayesian inference, inverse methods, and uncertainty quantification, among many \cite{Razavi.2021}. 
Mathematically, quantifying this change involves evaluating the gradient of the model, i.e., calculating a vector whose components are the partial derivatives of the model evaluated at the model parameter values. 
In sensitivity analysis, gradients are crucial for comprehending the relationships between model inputs and outputs, assessing the influence of each parameter, and evaluating the robustness of model predictions. 
In optimization and inverse modelling, where the goal is to fit models to data and/or invert for unknown or uncertain parameters, gradient-based methods are more efficient at finding a minimum and converge faster to them than gradient-free methods.
In Bayesian inference, gradient-based sampling strategies are better at estimating the posterior distribution than gradient-free methods \cite{neal2011mcmc}.
Therefore, accurately determining model gradients is essential for robust model understanding and effective data assimilation that leverage strong physical priors while offering flexibility to adapt to observations. 
This is very appealing in fields such as computational physics, geophysics, and biology, to mention a few, where there is a broad literature on DE-based models.
%and a long tradition in numerical methods. 
The techniques used to compute these gradients fall within the framework of differentiable programming.
% In other words:
% \begin{displayquote}
% \emph{A gradient serves as a compass: it tells us in which direction in the vast, open ocean of parameters we should move towards in order to increase our chances of success. }
% \end{displayquote} 

% Gradiens are obtained with DP
Differentiable programming (DP) refers to a set of techniques and software tools for computing gradients/sensitivities of a model's output, evaluated using a computer program, with respect to the model's input variables or parameters \cite{Shen_diff_modelling, Innes_Zygote, blondel2024elements}.
% This set of tools are particularly useful for the estimation and computation of the gradient of the model output with respect to certain input parameter.  
% In the case of DE-based models, DP techniques include automatic or algorithmic differentiation (AD), sensitivity equations, and adjoint methods. 
Arguably, the most celebrated DP method is automatic differentiation.
The set of tools known as automatic or algorithmic differentiation (AD) aims to compute derivatives of a model rendered on a computer by applying the chain rule to the sequence of unit operations that constitute a computer program \cite{Griewank:2008kh, Naumann.2011}. 
The premise is simple: every computer program is ultimately an algorithm described by a nested concatenation of elementary algebraic operations, such as addition and multiplication.
These operations are individually easy to differentiate, and their composition can be easily differentiated using the chain rule \cite{Giering_Kaminski_1998}.
During the last decades, reverse mode AD, also known as backpropagation, has enabled the fast growth of deep learning by efficiently computing gradients of neural networks \cite{griewank2012invented}.
% One of the challenges in implementing efficient routines to compute gradients of DE-based models is the integration of the different methods with automatic differentiation.
Some authors have recently suggested DP as the bridge between modern machine learning and traditional scientific models \cite{Ramsundar_Krishnamurthy_Viswanathan_2021, Shen_diff_modelling, Gelbrecht-differential-programming-Earth, rackauckas2021generalized}. 
More broadly than AD, DP tools for DE-models further include forward sensitivity and adjoint methods that compute the gradient by relying on an auxiliary set of differential equations.


% DP in different fields: traditional appraches and modern ML
The development of DP for DE-based models has a long tradition across different scientific communities.
In statistics, gradients of the likelihood function of DE-based models enable inference on the model parameters \cite{ramsay2017dynamic}. 
In numerical analysis, sensitivities quantify how the solution of a differential equation fluctuates with respect to certain parameters. 
This is particularly useful in optimal control theory, where the goal is to find the optimal value of some control (e.g. the shape of a wing) that minimizes a given loss function \cite{Giles_Pierce_2000}. 
In recent years, there has been an increasing interest in designing machine learning workflows that include constraints in the form of DEs and are trained using gradient descent techniques.
This emerging sub-field is usually referred as physics-based or physics-informed machine learning \cite{Karniadakis_Kevrekidis_Lu_Perdikaris_Wang_Yang_2021, thuerey2021pbdl, vadyala2022review}.

% SciML and large scale models 
The need for model gradients is even more critical as the total number of parameters and the expressivity of the model increases, especially when dealing with highly non-linear processes.
In such circumstances, the curse of dimensionality renders gradient-free optimization and sampling methods computationally intractable. 
This is the case in inverse methods \cite{Tarantola:2007wu, Ghattas.2021} and in machine learning applications \cite{LeCun2015}, where highly parametrized regressor functions (e.g., neural networks) are used to approximate unknown non-linear function.
Furthermore, for stochastic forward models, the intractability of the likelihood function represents a major challenge for statistical inference.
The integration of DP has provided new tools for resolving complex simulation-based inference problems \cite{Cranmer_Brehmer_Louppe_2020}.

% Difficulties with ODE and numerical solvers
Computing gradients of functions, represented by DE-based simulation codes, with respect to their (high-dimensional) inputs is challenging due to the complexities in both the mathematical framework and the software implementation involved.
Except for a small set of particular cases, most DEs require numerical methods to approximate their solution, which means that solutions cannot be differentiated analytically. 
Furthermore, numerical solutions introduce approximation errors. 
These errors can be propagated to the computation of the gradient, leading to inaccurate or inconsistent gradient values.  
Additional to these complexities, the broad family of numerical methods, each one of them with different advantages depending on the DE \cite{hairer-solving-1, hairer-solving-2}, means that the tools developed to compute gradients need to be universal enough in order to be applied to all or at least to a large set of them.


% challenges and knowledge gap
There exists a large family of methods to compute derivatives of DE-based models. 
The differences between methods to compute derivatives arise both from their mathematical formulation, numerical stability, and their computational implementation. 
They can be roughly classified as continuous (differentiate-then-discretize) or discrete (discretize-then-differentiate) and forward or reverse. 
% The first provides different guarantees on the method returning the actual gradient or a good approximation thereof. 
% The second involves how theory is translated to software, and what are the data structures and algorithms used to implement it. 
% These various flavours affect the accuracy, the computational complexity, the required memory and application suitability for computing the gradient of DE-based models. 
Different methods guarantee different levels of accuracy, have different computational complexity, and require different trade-offs between run time and memory usage. 
These properties further depend of the total number of parameters and size of the DE. 
Despite their independent success, integrating DP with DE-based models remains a significant challenge in high-performance scientific computing \cite{Naumann.2011}.
% As coined by Uwe Naumann, \textit{the automatic generation of optimal (in terms of robustness and efficiency) adjoint versions of large-scale simulation code is one of the great open challenges in the field of High-Performance Scientific Computing} \cite{Naumann.2011}.
\\ \\
This paper presents a comprehensive review of methods for calculating derivatives of the numerical solution of differential equations, with a focus on efficiently computing gradients. 
We review differentiable programming methods for differential equations from three different perspectives: a domain science perspective (Section \ref{sectopn:motivation}), a mathematical perspective (Section \ref{section:methods}) and a computer science perspective (Section \ref{sec:computational-implementation}). 
In Section \ref{sectopn:motivation} we introduce some of the applications of DP for the modelling of complex systems in the natural and social sciences. 
In Section \ref{section:methods} we present a coherent mathematical framework to understand the theoretical differences between the DP methods.
In Section \ref{sec:computational-implementation} we show how these methods can be computationally implemented and what are their numerical advantages and disadvantages.
For simplicity, all the methods introduced in Sections \ref{section:methods} and \ref{sec:computational-implementation} focus exclusively on first-order ODEs. 
How these methods generalize to other DE-based models, including PDEs, is discussed in Section \ref{section:generalization}.
We conclude the paper with a series of recommendations in Section \ref{section:recomendations}.
By providing a common framework across all methods and applications, we hope to facilitate the development of scalable, practical, and efficient differentiable DE-based models. 

