Differential equations are mathematical tools that explicitly describe the processes and dynamics within various systems, based on prior knowledge. They are fundamental in many scientific disciplines for modeling phenomena such as physical processes, population dynamics, social interactions, and chemical reactions.
% \todo[inline]{here, it would be nice to explain how they differ to data-drive models. Consider:}

By contrast, data-driven models do not necessarily require a detailed understanding of the underlying processes, and learn patterns and relationships directly from data. Data-driven models are particularly useful in scenarios where the underlying processes are poorly understood or too complex to be captured by traditional mathematical models.
The combination of mechanistic models with data-driven models is becoming increasingly common in many scientific domains. 
In order to achieve this, these models need to leverage both domain knowledge and data, to have an accurate representation of the underlying dynamics. 
Being able to determine which model parameters are most influential and further compute derivatives of such a model is key to correctly assimilating and learning from data, but a myriad of sensitivity methods exist to do so. 
In this review, we present an overview of the different sensitivity methods that exist, providing (i) guidelines on the best use cases for different scientific domain problems, (ii) detailed mathematical analyses of their characteristics, and (iii) computational implementations on how to solve them efficiently. 
