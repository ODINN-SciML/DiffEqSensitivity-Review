Scientific models are used to predict and understand a vast array of different dynamics, ranging from physical processes, ecological, biological and social interactions or chemical reactions, among many. 
The combination of mechanistic models with data-driven models is becoming increasingly common through many scientific domains. 
In order to achieve so, these models need to leverage both domain knowledge and data, in order to have an accurate representation of the underlying dynamics. 
Being able to determine which model parameters are most influential and further compute derivatives of such a model is key to correctly assimilate and learn from data, but a myriad of sensitivity methods exist to do so. 
We provide an overview of the different sensitivity methods that exist, providing (i) guidelines on the best usecases for different scientific domain problems, (ii) detailed mathematical analyses of their characteristics, and (iii) computational implementations on how to solve them efficiently. 
