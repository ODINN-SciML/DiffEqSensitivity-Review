Consider a system of first order ordinary differential equations (ODEs) given by
\begin{equation}
 \frac{du}{dt} = f(u, \theta, t),
 \label{eq:original_ODE}
\end{equation}
where $u \in \mathbb{R}^n$ is the unknown solution; $f: \R^n \times \R^p \times \R \mapsto \R^n$ is a function that depends on the state $u$, some vector parameter $\theta \in \mathbb R^p$, and potentially the independent variable $t$ which we will refer as time; and with initial condition $u(t_0) = u_0$.
Here $n$ denotes the total number of ODEs and $p$ the dimension of a parameter embedded in the functional form of the differential equation.
Although we here consider the case of ODEs, that is, when the derivatives are just with respect to the time variable $t$, the ideas presented here can be extended to the case of partial differential equations (PDEs; for example, via the method of lines \cite{ascher2008numerical}) and differential algebraic equations (DAE).
In fact, PDEs play an essential role when formulating equations of motion via local conservation (and constitutive) laws in physics-based simulations.
% Furthermore, the fact that both $u$ and $\theta$ are one-dimensional vectors does not prevent the use of higher-dimension objects (e.g. when $u$ is a matrix or a tensor). 
Except for a minority of functions $f(u,\theta, t)$, solutions to Equation \eqref{eq:original_ODE} need to be computed using a numerical solver. 

\subsubsection{Numerical solvers for ordinary differential equations}
\label{section:intro-numerical-solvers}
% Work in progress due to refactoring

Numerical solvers for the solution of ODEs or initial value problems can be classified as one-step methods, among which Runge-Kutta methods are the most widely used, and multi-step methods \cite{hairer-solving-1}.
With a long historical record in numerical analysis, Runge-Kutta methods generalize quadrature rules to iterative solve for the time discretization $u^{n} \approx u(t_n)$. 
Given an integer $s$, a $s$-stage Runge-Kutta methods is defined by generalizing numerical integration quadrature rules as follows
\begin{align}
\begin{split}
    u^{n+1} 
    &= 
    u^n 
    + 
    \Delta t_n \sum_{i=1}^s b_i k_i \\
    k_i 
    &= 
    f \left(u^n + \sum_{j=1}^s a_{ij} k_j , \, \theta , \, t_n + c_i \Delta t_n \right) \qquad i=1,2, \ldots, s.
    \label{eq:Runge-Kutta-scheme}
\end{split}
\end{align}
where $u^{n} \approx u(t_n)$ approximates the solution at time $t_n$; timesteps $\Delta t_n = t_{n+1}-t_n$; and coefficients $a_{ij}$, $b_i$, and $c_j$, with $i,j=1, 2,\ldots, j$, usually represented in the form of a tableau. 
A Runge-Kutta methods is called explicit if $a_{ij}=0$ for $i \leq j$; diagonally implicit if $a_{ij}=0$ for $i < j$; and implicit otherwise. 
Different choices of number of stages and coefficients give different orders of convergence of the numerical scheme \cite{Butcher_Wanner_1996, Butcher_2001}. 

On the contrary, multisteps linear solvers are of the form 
\begin{equation}
    \sum_{i=0}^{k_1} \alpha_{ni} u^{n-i} 
    =
    \Delta t_n \sum_{j=0}^{k_2} \beta_{nj} f(u^{n-j}, \theta, t_{n-j})
\end{equation}
where $\alpha_{ni}$ and $\beta_{nj}$ are numerical coefficients \cite{hairer-solving-1}.
In most cases, including Adam and BFG methods, we have the coefficients $\alpha_{ni} = \alpha_i$ and $\beta_{nj}=\beta_j$, meaning that the coefficient do not depend on the iteration. 
Notice that multisteps linear methods are linear in the function $f$, which is not the case in Runge-Kutta methods with intermediate evaluations \cite{ascher2008numerical}.
Explicit methods are characterized by $\beta_{n, 0} = 0$ and are easy to solve by direct iterative updates. 
For implicit methods, the usually non-linear equation 
\begin{equation}
    g_i(u_i; \theta) = u_i - h \beta_{n0} f(u_i, \theta, t_i) - \alpha_i = 0,
    \label{eq:solver-constriant-example}
\end{equation}
with $\alpha_i$ a computed coefficient that includes the information of all the past iterations, can be solved using predictor-corrector methods \cite{hairer-solving-1} or iteratively using Newton's method\cite{SUNDIALS-hindmarsh2005sundials}.  
% \begin{equation}
%     u_i^{(j+1)} 
%     = 
%     u_i^{(j)} - \left( \frac{\partial g_i}{\partial u_i} (u_i^{(j)}; \theta) \right)^{-1} g(u_i^{(j)}; \theta).
%     \label{eq:newton-method}
% \end{equation}

% Stiffness
There are many considerations at the moment of picking a numerical solver. 
One of the most important ones is the stiffness of the differential equation we are trying to solve. 
Although stiffness is a known phenomena in the study of differential equation solver, different definitions (and types) of instability exist in the literature. 
This is due to historical reasons \cite{Dahlquist_1985} as well as the fact that different stiff equations suffer from different types of instabilities. 
Among them we select
\begin{itemize}
    \item Stiff differential equations are characterized by dynamics with different time scales \cite{kim_stiff_2021}, also characterized by the phenomena of increasing oscillations\cite{Dahlquist_1985}.
    \item Stiff equations are equations for which explicit methods do not work and implicit methods work better \cite{hairer-solving-2}.
\end{itemize}
Stability properties can be achieved by the use of implicit methods over explicit methods. 
When using explicit methods, smaller timesteps may be required to guaranteed stability. 

% Timestep controller
Another important consideration is how to pick the time-steps $\Delta t_i$ in a numerical solver \cite{hairer-solving-1}. 
Modern solvers include stepsize controllers that pick $\Delta t_i$ as large as possible to minimize the total number of steps at the same time that they control for large errors in the numerical solution controlled by adjustable relative and absolute tolerances (see Appendix \ref{appendix:dual-number-solver}). 



% Generalization 
% It is important to remark that direct methods to solve higher order systems of ODEs exist, including Nystrom methods, but we can usually reduce them to a extended systems of ODEs \cite{hairer-solving-1}. 
% Partial differential equations (PDEs) can be solved via the method of lines by discretizing the spatial derivatives and just keeping the temporal derivatives in the form of a system of ODEs \cite{ascher2008numerical}.  
% Furthermore, it is important to remark that finite elements methods also exist to solve PDEs.  

% to a system of ODEs Replacing derivatives by finite differences is also a common practice when solving partial differential equations (PDEs), a technique known as the \textit{method of lines} 
% PDE, meshes, ... 


% Krilov methods work well for non-stiff methods, but the inverse condition number coincides with the stiff index, meaning that in stiff cases we should solve the system with full inversion models, which requires cubic cost in the number of ODEs 


\subsubsection{What to differentiate and why?}

We are interested in computing the gradient of a given function $L(u(\cdot, \theta))$ with respect to the parameter $\theta$.
This formulation is very general and allows to include many different applications, including the following. 
% Maybe organize this in sections, like 
\begin{itemize}
    \item \textbf{Loss function and empirical risk function}. This is usually a real-valued function that quantifies the level of agreement between the model prediction and observations. Examples of loss functions include the squared error
    \begin{equation}
         L(\theta) = \frac{1}{2} \| u(t_1; \theta) - u^{\text{target}}(t_1) \|_2^2,
         \label{eq:quadratic-loss-function}
    \end{equation}
    where $u^{\text{target}}(t_1)$ is the desired target observation at some later time $t_1$.
    More generally, we can evaluate the loss function at points of the time series for which we have observations, 
    \begin{equation}
        L(\theta) 
        = 
        \frac{1}{2} \sum_{i=1}^N 
        \, \omega_i \,
        \| u(t_i; \theta) - u^{\text{target}}(t_i) \|_2^2.
    \end{equation}
    with $\omega_i$ some arbitrary non-negative weights.
    More generally, misfit functions used in optimal estimation and control problems map from the model's state space, in this case the solution $u(t)$, to the observation space define by a new variable $y(t) = H(u(t, \theta))$, where $H: \R^n \mapsto \R^o$ is a given function mapping the latent state to observational space \cite{1975-Bryson-Ho-optimal-control}. 
    In these cases, the lost function is instead 
    \begin{equation}
        L(\theta) 
        =
        \frac{1}{2} 
        \sum_{i=1}^N
        \, \omega_i \,
        \| H(u(t_i; \theta)) - y^{\text{target}}(t_i) \|_2^2.
        \label{eq:loss-state-observation}
    \end{equation}
    We can also consider the continuous evaluated loss function of the form
    \begin{equation}
         L(u(\cdot, \theta)) = \int_{t_0}^{t_1} h( u(t;\theta), \theta)  dt, 
         \label{eq:integrated-loss-function}
    \end{equation}
    with $h$ being a function that quantifies the contribution of the error term at every time $t \in [t_0, t_1]$. 
    Defining a loss function where just the empirical error is penalized is known as trajectory matching \cite{ramsay2017dynamic}. 
    Other methods like gradient matching and generalized smoothing the loss depends on smooth approximations of the trajectory and their derivatives. 
    % \todo{this is unclear}
    \item \textbf{Likelihood function.} From a statistical perspective, it is common to assume that observations correspond to noisy observations of the underlying dynamical system, $y_i = H(u(t_i; \theta)) + \varepsilon_i$, with $\varepsilon_i$ errors or residual that are independent of each other and of the trajectory $u(\cdot ; \theta)$ \cite{ramsay2017dynamic}.
    When $H$ is the identity, each $y_i$ corresponds to the noise observation of the state $u(t_i; \theta)$.
    If $p(Y | t , \theta)$ is the probability distribution of $Y=(y_1, y_2, \ldots, y_N)$, maximum likelihood estimation consists in finding the maximum a posteriori (MAP) estimate of the parameter $\theta$ as
    \begin{equation}
        \theta^* 
        = 
        \argmax{\theta} \,\, \ell (Y | \theta) 
        = 
        \prod_{i=1}^n p(y_i | \theta, t_i) .
    \end{equation}
    When $\varepsilon_i \sim N(0, \sigma_i^2 \I)$ is the isotropic multivariate normal distribution, the maximum likelihood principle is the same as minimizing $- \log \ell(Y | \theta)$ which coincides with the mean squared error of Equation \eqref{eq:loss-state-observation}\cite{hastie2009elements},
    \begin{equation}
        \theta^* 
        = 
        \argmin{\theta} \, \left \{ - \log \ell (Y | \theta) \right \}
        = 
        \argmin{\theta} \, \sum_{i=1}^N 
        \, \frac{1}{2\sigma_i^2} \,
        \| y_i - H(u(t_i; \theta)) \|_2^2 .
    \end{equation}
    Provided with a prior distribution $p(\theta)$ for the parameter $\theta$, we can further compute a posterior distribution for $\theta$ given the observations $Y$ following Bayes theorem 
    \begin{equation}
        p(\theta | Y) = \frac{p(Y | \theta) \, p (\theta)}{p(Y)}. 
    \end{equation}
    In practice, the posterior is difficult to evaluate and needs to be approximated using Markov chain Monte Carlo (MCMC) sampling methods \cite{gelman2013bayesian}. Being able to further compute gradients of the likelihood allows to design more efficient sampling methods, such as Hamiltonian MCMC \cite{Betancourt_2017}.
    % Add mention for variatinal inference problems
    \item \textbf{Quantity of interest.} Another important example is when $L$ returns the value of the solution at one or many points, which is useful when we want to know how the solution itself changes as we move the parameter values. 
    \item \textbf{Diagnosis of the solution.} In many cases we are interested in optimizing the value of some variable that is a function of the solution of a differential equation. This is the case in design control theory, a popular approach in aerodynamics modelling where goals include maximizing the speed of an airplane or the lift of a wing given the solution of the flow equation for a given geometry profile \cite{Jameson_1988,Giles:2000wp,Mohammadi:2004dg}. 
\end{itemize}
In the rest of the manuscript we will use letter $L$ to emphasize that in many cases this will be a loss function, but without loss of generality this includes the richer class of functions included in the previous examples. 

\subsubsection{Optimization}

% Add one line adding general framework for optimization, since the following formula is no the more general optimization strategy.
In the context of optimization, the gradient of the loss allows performing gradient-based updates on the parameter $\theta$ by 
\begin{equation}
    \theta^{k+1} 
    = 
    \theta^k 
    - 
    \alpha_k 
    \frac{dL}{d\theta^k}.
\end{equation}
Gradient-based methods tend to outperform gradient-free optimization schemes, as they are not prone to the curse of dimensionality \cite{Schartau2017}. 
While a direct implementation of gradient descent is prone to converge to a local minimum and slow down in a neighborhood of saddle points, variants employing more advanced updating strategies have been proposed to avoid convergence to local minima \cite{ruder2016overview-gradient-descent}.
These methods include Newton-type methods \cite{second-order-optimization}, quasi-Newton methods, acceleration techniques \cite{JMLR:v22:20-207}, and natural gradient descent methods \cite{doi:10.1137/22M1477805}. 
For instance, ADAM is an adaptive, momentum-based algorithm  that remembers the solution update at each iteration, and determines the next update as a linear combination of the gradient and the previous update \cite{Kingma2014}
It has been widely adopted to train highly parametrized neural networks (up to the order of $10^8$ parameters \cite{NIPS2017_3f5ee243}).
Other widely employed algorithms are the Broyden–Fletcher–Goldfarb–Shanno (BFGS) and its limited-memory version algorithm (L-BFGS), which determine the descent direction by preconditioning the gradient with curvature information. 
ADAM is less prone to converge to a local minimum, while (L-)BFGS has a faster converge rate. 
Using ADAM for the first iterations followed by (L-)BFGS proves to be a successful strategy to minimize a loss function with best accuracy. 
% Furthermore, gradient-free methods (also known as global optimization techniques \todo{Some gradient free methods are not necessarily global optimization techniques, e.g. evolutionary algorithms \cite{wilke2001evolution,Rodriguez-Fernandez2006} }) rely in heuristics\cite{Pearl-heuristics} that are not guaranteed to find the solution. 
% I will mentioned that other methods for optimization based on the gradient exists (e.g., majorization) but not giving much details on it. 

% 

\subsubsection{Sensitivity matrix}

In the general case, we are going to work with loss functions of the form $L(\theta) = L(u(\cdot, \theta), \theta)$. 
Using the chain rule we can derive 
\begin{equation} 
 \frac{dL}{d\theta} = \frac{\partial L}{\partial u} \frac{\partial u}{\partial \theta} + \frac{\partial L}{\partial \theta}.
 \label{eq:dLdtheta_VJP}
\end{equation} 
The two partial derivatives of the loss function on the right-hand side are usually easy to evaluate.
For example, for the loss function in Equation \eqref{eq:quadratic-loss-function} this are simply given by 
\begin{equation}
    \frac{\partial L}{\partial u} = u - u^{\text{target}}(t_1)
    \qquad 
    \frac{\partial L}{\partial \theta} = 0.
    \label{eq:dLdu}
\end{equation}
Just as in this last example, in most applications the loss function $L(\theta)$ will depend on $\theta$ just thought $u$, meaning $\frac{\partial L}{\partial \theta} = 0$.
The complicated term to compute is the matrix of derivatives $\frac{\partial u}{\partial \theta}$, usually referred to as the \textit{sensitivity} $s$, and represents how much the full solution $u$ varies as a function of the parameter $\theta$, 
\begin{equation}
 s 
 = 
 \frac{\partial u}{\partial \theta} 
 =
 \begin{bmatrix}
   \frac{\partial u_1}{\partial \theta_1} & \dots & \frac{\partial u_1}{\partial \theta_p} \\
   \vdots & \ddots & \vdots \\
   \frac{\partial u_n}{\partial \theta_1} & \dots & \frac{\partial u_n}{\partial \theta_p}
 \end{bmatrix}
 \in \mathbb R^{n \times p}.
 \label{eq:sensitivity-definition}
\end{equation}
The sensitivity $s$ defined in Equation \eqref{eq:sensitivity-definition} is what is called a \textit{Jacobian}, that is, a matrix of first derivatives for general vector-valued functions.
Some of the methods we will discuss here will directly compute the sensitivity, while others will only deal with Jacobian-vector products (JVPs) of the form $\frac{\partial u}{\partial \theta} v$, for some vector $v \in \R^p$. 
The product $\frac{\partial u}{\partial \theta}v$ is the directional derivative of the function $u(\theta)$, also known as the Gateaux derivative of $u(\theta)$ in the direction $v$, given by 
\begin{equation}
    \frac{\partial u}{\partial \theta} v 
    = 
    \lim_{h \rightarrow 0} \frac{u(\theta + h v) - u(\theta)}{h},
    \label{eq:directional-derivative}
\end{equation}
representing how much the function $u$ changes when we perturb $\theta$ in the direction of $v$. 

% Notice here the distinction between the total derivative (indicated with the $d$) and partial derivative symbols ($\partial$). 
% When a function depends on more than one argument, we use the partial derivative symbol to emphasize this distinction (e.g., Equation \eqref{eq:sensitivity-definition}). 
% On the other hand, when this is not the case, we will use the total derivative symbol (e.g., Equation \eqref{eq:dLdu}).


