Consider the first-order ODE given by
\begin{equation}
 \frac{du}{dt} = f(u, \theta, t)
 \label{eq:original_ODE}
\end{equation}
subject to the initial condition $u(t_0) = u_0$, where $u \in \mathbb{R}^n$ is the unknown solution vector of the ODE, $f: \R^n \times \R^p \times \R \mapsto \R^n$ is a function that depends on the state $u$, $\theta \in \mathbb R^p$ is a vector parameter, and $t \in [t_0, t_1]$ refers to time.
Here, $n$ denotes the size of the ODE and $p$ the number of parameters.
Except for a minority of functions $f(u,\theta, t)$, solutions to Equation \eqref{eq:original_ODE} need to be computed using numerical solvers. 

\subsubsection{Numerical solvers for ODEs}
\label{section:intro-numerical-solvers}
% Work in progress due to refactoring

Numerical solvers for the solution of ODEs or initial value problems can be classified as one-step methods, among which Runge-Kutta methods are the most widely used, and multi-step methods \cite{hairer-solving-1}.
% With a long historical record in numerical analysis, Runge-Kutta methods generalize quadrature rules to iteratively solve for the time discretization $u^{n} \approx u(t_n)$. 
Given an integer $s$, $s$-stage Runge-Kutta methods are defined by generalizing numerical integration quadrature rules as follows
\begin{align}
\begin{split}
    u^{m+1} 
    &= 
    u^m 
    + 
    \Delta t_m \sum_{i=1}^s b_i k_i \\
    k_i 
    &= 
    f \left(u^m + \sum_{j=1}^s a_{ij} k_j , \, \theta , \, t_m + c_i \Delta t_m \right) \qquad i=1,2, \ldots, s.
    \label{eq:Runge-Kutta-scheme}
\end{split}
\end{align}
where $u^{m} \approx u(t_m)$ approximates the solution at time $t_m$, $\Delta t_m = t_{m+1}-t_m$, and $a_{ij}$, $b_i$, and $c_j$ are scalar coefficients with $i,j=1, 2,\ldots, j$, usually represented in the form of a tableau. 
A Runge-Kutta method is called explicit if $a_{ij}=0$ for $i \leq j$; diagonally implicit if $a_{ij}=0$ for $i < j$; and fully implicit otherwise. 
Different choices of the number of stages $s$ and coefficients give different orders of convergence of the numerical scheme \cite{Butcher_Wanner_1996, Butcher_2001}. 

In contrast, multi-step linear solvers are of the form 
\begin{equation}
    \sum_{i=0}^{k_1} \alpha_{mi} u^{m-i} 
    =
    \Delta t_m \sum_{j=0}^{k_2} \beta_{mj} f(u^{m-j}, \theta, t_{m-j})
\end{equation}
where $\alpha_{mi}$ and $\beta_{mj}$ are numerical coefficients \cite{hairer-solving-1}.
In most cases, including the Adams methods and backwards differentiation formulas (BDF), we have the coefficients $\alpha_{mi} = \alpha_i$ and $\beta_{mj}=\beta_j$, meaning that the coefficient do not depend on the iteration $m$. 
Notice that multi-step linear methods are linear in the function $f$, which is not the case in Runge-Kutta methods with intermediate evaluations \cite{ascher2008numerical}.
Explicit methods are characterized by $\beta_{m0} = 0$ and are easy to solve by direct iterative updates. 
For implicit methods, the usually non-linear equation 
\begin{equation}
    g_m(u^m; \theta) = u^m - \Delta t_m  \beta_{m0} f(u^m, \theta, t_m) - \bar{\alpha}_m = 0,
    \label{eq:solver-constriant-example}
\end{equation}
with $\bar{\alpha}_m$ a computed coefficient that includes the information of all past iterations, can be solved using predictor-corrector methods \cite{hairer-solving-1} or iteratively using Newton's method and preconditioned Krylov solvers at each nonlinear iteration \cite{SUNDIALS-hindmarsh2005sundials}.  
% \begin{equation}
%     u_i^{(j+1)} 
%     = 
%     u_i^{(j)} - \left( \frac{\partial g_i}{\partial u_i} (u_i^{(j)}; \theta) \right)^{-1} g(u_i^{(j)}; \theta).
%     \label{eq:newton-method}
% \end{equation}

% Stiffness
When choosing a numerical solver for differential equations, one crucial factor to consider is the stiffness of the equation.
Stiffness encompasses various definitions, reflecting its historical development and different types of instabilities \cite{Dahlquist_1985}.
Two definitions are noteworthy
\begin{enumerate}
    \item[$ \blacktriangleright$] Stiff equations are equations for which explicit methods do not work and implicit methods work better \cite{hairer-solving-2}.
    \item[$ \blacktriangleright$] Stiff differential equations are characterized by dynamics with different time scales \cite{hairer-solving-2, kim_stiff_2021}, also characterized by the phenomena of increasing oscillations \cite{Dahlquist_1985}.
\end{enumerate} 
Stability properties can be achieved by different means, for example, by using implicit methods or stabilized explicit methods, such as Runge–Kutta–Chebyshev \cite{van1980internal, hairer-solving-2}. 
When using explicit methods, smaller timesteps may be required to guarantee stability. 

% Timestep controller
% Another important consideration is the choice of the time-steps $\Delta t_i$ in a numerical solver \cite{hairer-solving-1}. 
% Modern solvers include stepsize controllers that pick $\Delta t_i$ as large as possible to minimize the total number of steps while preventing large errors in the numerical solution controlled by adjustable relative and absolute tolerances (see Section \ref{section:AD-incorrect}). 

Numerical solvers usually estimate internally a scaled error computed as 
\begin{equation}
    \text{Err}_\text{scaled}^{m}
    =
    \sqrt{
    \frac{1}{n} \sum_{i=1}^n \left( \frac{\text{err}_i^{m}}{\mathfrak{abstol} + \mathfrak{reltol} \, \times \, M^m_i} \right)^2 },
    \label{eq:internal-norm-wrong}
\end{equation}
with $\mathfrak{abstol}$ and $\mathfrak{reltol}$ the adjustable absolute and relative solver tolerances, respectively, $M$ is the maximum expected value of the numerical solution, and $\text{err}^m$ is an estimation of the numerical error at step $m$ \cite{hairer-solving-1, Rackauckas_Nie_2016}. 
Estimations of the local error $\text{err}^{m}$ can be based on two approximation to the solution based on embedded Runge-Kutta pairs \cite{Ranocha_Dalcin_Parsani_Ketcheson_2022, hairer-solving-1}, or in theoretical upper bounds provided by the numerical solver. 
In the first case, common choices for these include $M^m_i = \max \{ u_i^{m}, \hat u_i^{m} \}$ and $\text{err}_i^{m} = u_i^{m} - \hat u_i^{m}$ with $u^m$ and $\hat u^m$ two different approximations for $u(t_m)$, but these can vary between solvers. 
% The choice of the norm $\frac{1}{\sqrt n} \| \cdot \|_2$ for computing the total error $\text{Err}_\text{scaled}$, sometimes known as Hairer norm, has been the standard for a long time \cite{Ranocha_Dalcin_Parsani_Ketcheson_2022} and it is based on the assumption that a small increase in the size of the systems of ODEs (e.g., by simply duplicating the ODE system) should not affect the stepsize choice, but other options can be considered \cite{hairer-solving-1}.   

Modern solvers include stepsize controllers that pick $\Delta t_m$ as large as possible to minimize the total number of steps while preventing large errors by keeping $\text{Err}^m_\text{scaled} \leq 1$. 
One of the most used methods to archive this is the proportional-integral controller (PIC) that updates the stepsize according to 
\begin{equation}
    \Delta t_{m} = \eta \, \Delta t_{m-1}
    \qquad 
    \eta = w_{m}^{\beta_1 / q} w_{m-1}^{\beta_2 / q} w_{m-2}^{\beta_3 / q}
    \label{eq:PIC}
\end{equation}
with $w_{m} = 1 / \text{Err}_\text{scaled}^{m}$ the inverse of the scaled error estimates; $\beta_1$, $\beta_2$, and $\beta_3$ numerical coefficients defined by the controller; and $q$ the order of the numerical solver \cite{hairer-solving-2, Ranocha_Dalcin_Parsani_Ketcheson_2022}. 
If the stepsize $\Delta t_{m}$ proposed in Equation \eqref{eq:PIC} to update from $u^{m}$ to $u^{m+1}$ does not satisfy $\text{Err}_\text{scaled}^{m+1} \leq 1$, a new smaller stepsize is proposed. 
When $\eta < 1$ (which is the case for simple controllers with $\beta_2 = \beta_3 = 0$), Equation \eqref{eq:PIC} can be used for the local update. 
It is also common to restrict $\eta \in [\eta_\text{min}, \eta_\text{max}]$ so the stepsize does not change abruptly \cite{hairer-solving-1}. 



\subsubsection{What to differentiate?}

In most applications, the need for differentiating the solution of ODEs stems from the need to obtain the gradient of a function $L(\theta) = L(u(\cdot, \theta))$ with respect to the parameter $\theta$, where $L$ can denote:
\begin{itemize}
    \item[$ \blacktriangleright$] \textbf{A loss function or an empirical risk function}. This is usually a real-valued function that quantifies the level of agreement between the model prediction and observations. Examples of loss functions include the squared error
    \begin{equation}
         L(\theta) = \frac{1}{2} \left \| u(t_1; \theta) - u^{\text{target}}(t_1) \right \|_2^2,
         \label{eq:quadratic-loss-function}
    \end{equation}
    where $u^{\text{target}}(t_1)$ is the desired target observation at some later time $t_1$, and $\| \cdot \|_2$ is the Euclidean norm.
    More generally, we can evaluate the loss function at points of the time series for which we have observations, 
    \begin{equation}
        L(\theta) 
        = 
        \frac{1}{2} \sum_{i=1}^N 
        \, w_i 
        \left \| u(t_i; \theta) - u^{\text{target}}(t_i) \right \|_2^2.
        \label{eq:quadratic-loss-point}
    \end{equation}
    with $w_i$ some arbitrary non-negative weights.
    More generally, misfit functions used in optimal estimation and control problems are composite maps from the parameter space $\theta$ via the model's state space (in this case, the solution $u(t; \theta)$) to the observation space defined by a new variable $y(t) = H(u(t; \theta))$, where $H: \R^n \mapsto \R^o$ is a given function mapping the latent state to observational space \cite{1975-Bryson-Ho-optimal-control}. 
    In these cases, the loss function generalizes to 
    \begin{equation}
        L(\theta) 
        =
        \frac{1}{2} 
        \sum_{i=1}^N
        \, w_i 
        \left \| H(u(t_i; \theta)) - y^{\text{target}}(t_i) \right \|_2^2.
        \label{eq:loss-state-observation}
    \end{equation}
    We can also consider the continuous evaluated loss function of the form
    \begin{equation}
         L(u(\cdot; \theta)) = \int_{t_0}^{t_1} h( u(t;\theta), \theta)  dt, 
         \label{eq:integrated-loss-function}
    \end{equation}
    with $h$ being a function that quantifies the contribution of the error term at every time $t \in [t_0, t_1]$. 
    Defining a loss function where just the empirical error is penalized is known as trajectory matching \cite{ramsay2017dynamic}. 
    Other methods like gradient matching and generalized smoothing the loss depends on smooth approximations of the trajectory and their derivatives. 
    % \todo{this is unclear}
    \item[$ \blacktriangleright$] \textbf{The likelihood function or posterior probability.} From a statistical and physical perspective, it is common to assume that observations correspond to noisy observations of the underlying dynamical system, $y_i = H(u(t_i; \theta)) + \varepsilon_i$, with $\varepsilon_i$ errors or residual that are independent of each other and of the trajectory $u(\cdot ; \theta)$ \cite{ramsay2017dynamic}.
    When $H$ is the identity, each $y_i$ corresponds to the noisy observation of the state $u(t_i; \theta)$.
    If $p(Y | t , \theta)$ is the probability distribution of $Y=(y_1, y_2, \ldots, y_N)$, 
    the maximum likelihood estimator (MLE) of $\theta$ is defined as 
    \begin{equation}
        \theta^\text{MLE} 
        = 
        \argmax{\theta} \,\, \ell (Y | \theta) 
        = 
        \prod_{i=1}^N p(y_i | \theta, t_i) .
    \end{equation}
    When $\varepsilon_i \sim N(0, \sigma_i^2 \I)$ is the isotropic multivariate normal distribution, the maximum likelihood principle is the same as minimizing $- \log \ell(Y | \theta)$ which coincides with the mean squared error of Equation \eqref{eq:loss-state-observation} \cite{hastie2009elements},
    \begin{equation}
        \theta^\text{MLE} 
        = 
        \argmin{\theta} \, \left \{ - \log \ell (Y | \theta) \right \}
        = 
        \argmin{\theta} \, \sum_{i=1}^N 
        \, \frac{1}{2\sigma_i^2} \,
        \left \| y_i - H(u(t_i; \theta)) \right \|_2^2 .
        \label{eq:MLE}
    \end{equation}
    A Bayesian formulation of equation \eqref{eq:MLE} would consist in deriving a point estimate $\theta^\text{MLE}$, the posterior mean of the maximum a posteriori (MAP), based on the posterior distribution for $\theta$ following Bayes theorem as $p(\theta | Y) = {p(Y | \theta) \, p (\theta)}/{p(Y)}$, where $p(\theta)$ is the prior distribution \cite{pml1Book}.
    % \begin{equation}
    %     p(\theta | Y) = \frac{p(Y | \theta) \, p (\theta)}{p(Y)}. 
    % \end{equation}
    In most realistic cases, the posterior distribution is approximated using Markov chain Monte Carlo (MCMC) sampling methods \cite{gelman2013bayesian}. 
    Being able to further compute gradients of the likelihood allows to design more efficient sampling methods, such as Hamiltonian Monte Carlo \cite{Betancourt_2017}.
    % Add mention for variatinal inference problems
    \item[$ \blacktriangleright$] \textbf{A quantity of interest.} In some applications we are interested in quantifying how the solution of the differential equation changes as we vary the parameter values; or more generally when it returns the value of some variable that is a function of the solution of a differential equation. The latter corresponds to the case in design control theory, a popular approach in aerodynamics modelling where goals include maximizing the speed of an airplane or the lift of a wing given the solution of the flow equation for a given geometry profile \cite{Jameson_1988,Giles_Pierce_2000,Mohammadi:2004dg}. 
\end{itemize}
In the rest of the manuscript we will use letter $L$ to emphasize that in many cases this will be a loss function, but without loss of generality this includes the richer class of functions included in the previous examples. 

\subsubsection{Gradient-based optimization}

% Add one line adding general framework for optimization, since the following formula is no the more general optimization strategy.
In the context of optimization, the goal is to find the parameter $\theta$ that is a minimizer of $L(\theta)$. 
There exists a broad literature of optimization methods based on gradients, including gradient descent and its many variants \cite{ruder2016overview-gradient-descent}.
Gradient-based methods tend to outperform gradient-free optimization schemes when $1 \ll p$, as they are not prone to the curse of dimensionality \cite{Schartau2017}. 
In the case of gradient descent, the parameter $\theta$ is updated based on the iterative procedure given by
\begin{equation}\label{eq:gradient-descent}
    \theta^{m+1} 
    = 
    \theta^m 
    - 
    \alpha_m 
    \frac{dL}{d\theta}(\theta^m),
\end{equation}
with $\alpha_m$ some choice of the stepsize and some initialization $\theta^0 \in \R^p$. 
A direct implementation of gradient descent following Equation \eqref{eq:gradient-descent} is prone to converge to a local minimum and slows down in a neighborhood of saddle points. 
To address these issues, variants of this scheme employing more advanced updating strategies have been proposed, including Newton-type methods \cite{second-order-optimization}, quasi-Newton methods, acceleration techniques \cite{JMLR:v22:20-207}, and natural gradient descent methods \cite{doi:10.1137/22M1477805}. 
For instance, ADAM is an adaptive, momentum-based algorithm  that stores the parameter update at each iteration, and determines the next update as a linear combination of the gradient and the previous update \cite{Kingma2014}.
ADAM been widely adopted to train highly parametrized neural networks (up to the order of $10^8$ parameters \cite{NIPS2017_3f5ee243}).
Other widely employed algorithms are the Broyden–Fletcher–Goldfarb–Shanno (BFGS) and its limited-memory version algorithm (L-BFGS), which determine the descent direction by preconditioning the gradient with curvature information. 
% ADAM is less prone to converge to a local minimum, while (L-)BFGS has a faster converge rate. 
% Using ADAM for the first iterations followed by (L-)BFGS proves to be a successful strategy to minimize a loss function with best accuracy. 
% Furthermore, gradient-free methods (also known as global optimization techniques \todo{Some gradient free methods are not necessarily global optimization techniques, e.g. evolutionary algorithms \cite{wilke2001evolution,Rodriguez-Fernandez2006} }) rely in heuristics\cite{Pearl-heuristics} that are not guaranteed to find the solution. 
% I will mentioned that other methods for optimization based on the gradient exists (e.g., majorization) but not giving much details on it. 

% 

\subsubsection{Sensitivity matrix}

In general, loss functions considered are of the form $L(\theta) = L(u(\cdot, \theta), \theta)$. 
Using the chain rule we can derive 
\begin{equation} 
 \frac{dL}{d\theta} = \frac{\partial L}{\partial u} \frac{\partial u}{\partial \theta} + \frac{\partial L}{\partial \theta}.
 \label{eq:dLdtheta_VJP}
\end{equation} 
Notice here the distinction between the direct derivative $\frac{d}{d\theta}$ and partial derivative $\frac{\partial}{\partial \theta}$.
The two partial derivatives involving the loss function on the right-hand side are usually easy to evaluate.
For example, for the loss function in Equation \eqref{eq:quadratic-loss-function}, these are simply given by 
\begin{equation}
    \frac{\partial L}{\partial u} = u - u^{\text{target}}(t_1)
    \qquad 
    \frac{\partial L}{\partial \theta} = 0.
    \label{eq:dLdu}
\end{equation}
In most applications, the empirical component of the loss function $L(\theta)$, that is, the part of the loss that is a function on the data, will depend on $\theta$ just through $u$, meaning $\frac{\partial L}{\partial \theta} = 0$. 
However, regularization terms added to the loss can directly depend on the parameter $\theta$, that is $\frac{\partial L}{\partial \theta} \neq 0$.
In both cases, the complicated term to compute is the matrix of derivatives $\frac{\partial u}{\partial \theta}$, usually referred to as the \textit{sensitivity} $s$, and represents how much the full solution $u$ varies as a function of the parameter $\theta$, 
\begin{equation}
 s 
 = 
 \frac{\partial u}{\partial \theta} 
 =
 \begin{bmatrix}
   \frac{\partial u_1}{\partial \theta_1} & \dots & \frac{\partial u_1}{\partial \theta_p} \\
   \vdots & \ddots & \vdots \\
   \frac{\partial u_n}{\partial \theta_1} & \dots & \frac{\partial u_n}{\partial \theta_p}
 \end{bmatrix}
 \in \mathbb R^{n \times p}.
 \label{eq:sensitivity-definition}
\end{equation}
The sensitivity $s$ defined in Equation \eqref{eq:sensitivity-definition} is a \textit{Jacobian}, that is, a matrix of first derivatives of a vector-valued function. 
% Methods involved in the calculation of $s$ are naturally part of sensitivity methods.
% As mentioned earlier, most of forward sensitivity methods compute the full sensitivity matrix $s$, while reverse methods only deal with Jacobian-vector products (JVPs) of the form $\frac{\partial u}{\partial \theta} v$, for some vector $v \in \R^p$, saving unnecessary calculations at the expenses of larger memory cost.

Notice that the product $s v$, with $v \in \R^p$, is the directional derivative of the function $u(\theta)$ in the direction $v$, that is 
\begin{equation}
    s v
    = 
    \frac{\partial u}{\partial \theta} v 
    = 
    \lim_{h \rightarrow 0} \frac{u(\theta + h v) - u(\theta)}{h},
    \label{eq:directional-derivative}
\end{equation}
which represents how much the function $u$ changes when we perturb $\theta$ in the direction of $v$. 

% Notice here the distinction between the total derivative (indicated with the $d$) and partial derivative symbols ($\partial$). 
% When a function depends on more than one argument, we use the partial derivative symbol to emphasize this distinction (e.g., Equation \eqref{eq:sensitivity-definition}). 
% On the other hand, when this is not the case, we will use the total derivative symbol (e.g., Equation \eqref{eq:dLdu}).


