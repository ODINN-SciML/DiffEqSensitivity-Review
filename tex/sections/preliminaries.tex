Consider a system of ordinary differential equations (ODEs) given by
\begin{equation}
 \frac{du}{dt} = f(u, \theta, t),
 \label{eq:original_ODE}
\end{equation}
where $u \in \mathbb{R}^n$ is the unknown solution; $f$ is a function that depends on the state $u$, some parameter $\theta \in \mathbb R^p$, and an independent variable $t$ which we will refer as time, but it can represent another quantity; and with initial condition $u(t_0) = u_0$.
Here $n$ denotes the total number of ODEs and $p$ the size of a parameter embedded in the functional form of the differential equation.
Although here we consider the case of ODEs, that is, when the derivatives are just with respect to the time variable $t$, the ideas presented here can be extended of the case of partial differential equations (for example, via the method of lines \cite{ascher2008numerical}) and differential algebraic equations (DAE).
% Furthermore, the fact that both $u$ and $\theta$ are one-dimensional vectors does not prevent the use of higher-dimension objects (e.g. when $u$ is a matrix or a tensor). 
Except for a minority of functions $f(u,\theta, t)$, solutions of the Equation \eqref{eq:original_ODE} need to be computed using a numerical solver. 

We are interested in computing the gradient of a given function $L(u(\cdot, \theta))$ with respect to the parameter $\theta$, which can be useful within different contexts, which we present in the next subsections.

\subsubsection{Inverse modelling and optimization}

In the context of optimization and inverse modelling, $L$ expresses a distance between the ODE model simulation and some data. 

\begin{itemize}
    \item \textbf{Empirical loss functions}. This is usually a real-valued function that quantifies the distance between the model prediction and the data. Examples of loss functions include the squared error
    \begin{equation}
         L(u(\cdot, \theta)) = \frac{1}{2} \| u(t_1; \theta) - u^{\text{target}}(t_1) \|_2^2,
         \label{eq:quadratic-loss-function}
    \end{equation}
    where $u^{\text{target}}(t_1)$ is the desired target observation at some later time $t_1$.
    More generally, we can evaluate the loss function at points of the time series for which we have observations, 
    \begin{equation}
        L(u(\cdot, \theta)) = \frac{1}{2} \sum_{i=1}^N \| u(t_i; \theta) - u^{\text{target}}(t_i) \|_2^2.
    \end{equation}
    We can also consider the continuous evaluated loss function of the form
    \begin{equation}
         L(u(\cdot, \theta)) = \int_{t_0}^{t_1} h( u(t;\theta), \theta) ) dt, 
    \end{equation}
    with $h$ being a function that quantifies the contribution of the error term at every time $t \in [t_0, t_1]$. 
    Defining a loss function where just the empirical error is penalized is known as trajectory matching. 
    Other methods like gradient matching and generalized smoothing the loss depends on smooth approximations of the trajectory and their derivatives. 
    \todo{this is unclear}
    \item \textbf{Likelihood profiles.} From a statistical perspective, it is common to assume that observations correspond to noisy observations of the underlying dynamical system, $y_i = u(t_i; \theta) + \varepsilon_i$, with $\varepsilon_i$ errors or residual that are independent of each other and of the trajectory $u(\cdot ; \theta)$ \cite{ramsay2017dynamic}.
    If $p(y | t , \theta)$ is the probability distribution of $y$, maximum likelihood estimation consists in finding the parameter $\theta$ as
    \begin{equation}
        \theta^* 
        = 
        \argmax{\theta} \, \ell (y | \theta) 
        = 
        \prod_{i=1}^n p(y_i | \theta, t_i) .
    \end{equation}
    When $\varepsilon \sim N(0, \sigma_i^2)$ is Gaussian, the maximum likelihood principle is the same as minimizing $- \log \ell(y | \theta)$ which results in the mean squared error
    \begin{equation}
        \theta^* 
        = 
        \argmin{\theta} \, \left \{ - \log \ell (y | \theta) \right \}
        = 
        \argmin{\theta} \, \sum_{i=1}^n \left( y_i - u(t_i; \theta) \right)^2 .
    \end{equation} \todo[inline]{There is a correspondance between this equation and the empirical loss function above, which would be nice to show. See the proposed section "Maximum likelihood estimation and loss function". This is also discussed in e.g. https://www.deeplearningbook.org/contents/prob.html}
    Provided with a prior distribution $p(\theta)$ for the parameter $\theta$, we can further compute a posterior distribution for $\theta$ given the observations $y_1, y_2, \ldots, y_n$ following Bayes theorem 
    \begin{equation}
        p(\theta | y) = \frac{p(y | \theta) p (\theta)}{p(y)}. 
    \end{equation}
    In practice, the posterior is difficult to evaluate and needs to be approximated using Markov chain Monte Carlo sampling methods \cite{gelman2013bayesian}. being able to further compute gradients of the likelihood allows to design more efficient sampling methods, such as Hamiltonian MCMC \cite{Betancourt_2017}. 
\end{itemize}

In this context, the gradient of the loss allows performing gradient-based updates on the parameter $\theta$ by 
\begin{equation}
    \theta^{k+1} 
    = 
    \theta^k 
    - 
    \alpha_k 
    \frac{dL}{d\theta^k}.
\end{equation}

Gradient-based methods tend to outperform gradient-free optimization schemes, as they are not prone to the curse of dimensionality \cite{Schartau2017}. While a direct implementation of gradient descent is prone to converge to a local minimum and slow down in a neighborhood of saddle points, variants employing more advanced updating strategies have been proposed \cite{ruder2016overview-gradient-descent} to avoid convergence to local minima, and are widely adopted in the field of artificial intelligence to train highly parametrized neural networks (up to the order of $10^8$ parameters \cite{NIPS2017_3f5ee243}). For instance, Adam \cite{Kingma2014} is an adaptive, momentum-based algorithm  that remembers the solution update at each iteration, and determines the next update as a linear combination of the gradient and the previous update, reducing the risk to converge to local minima. Other broadly employed algorithms are the Broyden–Fletcher–Goldfarb–Shanno (BFGS) and its limited-memory version algorithm (L-BFGS), which determine the descent direction by preconditioning the gradient with curvature information. ADAM is less prone to converging to a local minimum, while (L-)BFGS has a faster converge rate. Using ADAM for the first iterations followed by (L-)BFGS proves to be a successful strategy to minimize a loss function with best accuracy. 



% Furthermore, gradient-free methods (also known as global optimization techniques \todo{Some gradient free methods are not necessarily global optimization techniques, e.g. evolutionary algorithms \cite{wilke2001evolution,Rodriguez-Fernandez2006} }) rely in heuristics\cite{Pearl-heuristics} that are not guaranteed to find the solution. 



    \subsubsection{Model sensitivity to parameters}
    % 
    Another important example is when $L$ returns the value of the solution at one or many points, which is useful when we want to know how the solution itself changes as we move the parameter values. 
    \subsubsection{Diagnosis of the solution}
    
    In many cases we are interested in optimizing the value of some variable that is a function of the solution of a differential equation. This is the case in design control theory, a popular approach in aerodynamics modelling where goals include maximizing the speed of an airplane or the lift of a wing given the solution of the flow equation for a given geometry profile \cite{Jameson_1988}. 

Notice that these cases also include sensitivities of $L$ with respect to the initial condition of the system.


\subsubsection{Sensitivity matrix}
Using the chain rule we can derive
\begin{equation} 
 \frac{dL}{d\theta} = \frac{dL}{du} \frac{\partial u}{\partial \theta}.
 \label{eq:dLdtheta_VJP}
\end{equation} 
The first term on the right-hand side is usually easy to evaluate since it just involves the partial derivative of the scalar loss function with respect to the solution.
For example, for the loss function in Equation \eqref{eq:quadratic-loss-function} this is simply
\begin{equation}
    \frac{dL}{du} = u - u^{\text{target}}(t_1).
    \label{eq:dLdu}
\end{equation}
The second term on the right-hand side is more difficult to compute and it is usually referred to as the \textit{sensitivity},
\begin{equation}
 s 
 = 
 \frac{\partial u}{\partial \theta} 
 =
 \begin{bmatrix}
   \frac{\partial u_1}{\partial \theta_1} & \dots & \frac{\partial u_1}{\partial \theta_p} \\
   \vdots & \ddots & \vdots \\
   \frac{\partial u_n}{\partial \theta_1} & \dots & \frac{\partial u_n}{\partial \theta_p}
 \end{bmatrix}
 \in \mathbb R^{n \times p}.
 \label{eq:sensitivity-definition}
\end{equation}
Notice here the distinction between the total derivative (indicated with the $d$) and partial derivative symbols ($\partial$). 
When a function depends on more than one argument, we use the partial derivative symbol to emphasize this distinction (e.g., Equation \eqref{eq:sensitivity-definition}). 
On the other side, when this is not the case, we will use the total derivative symbol (e.g., Equation \eqref{eq:dLdu}).
Also notice that the sensitivity $s$ defined in Equation \eqref{eq:sensitivity-definition} is what is called a \textit{Jacobian}, that is, a matrix of first derivatives for general vector-valued functions.

% In this article we are going to use the word gradient or derivative to refer to the first order derivatives of a given function. 
% Although the names adjoint and tangent are sometime used to refer to the same object, we are going to skip the use of these to avoid confusion.
% The same nature of the adjoint methods deserves to be treated entirely in Section \ref{section:adjoint-methods}.