For sufficient small systems of less than 100 parameters and ODEs, Forward AD is the most efficient method, outperforming sensitivity and adjoint methods \cite{ma_comparison_2021}.
% \todo[inline]{We should pay attention that this section does not overlap with \cite{ma_comparison_2021}.}

% Mention to limit cases when working with chaotic systems
\subsection{Chaotic systems}

% Intro: why we take this average quantities in chaotic system? Need to motivate this in a sentense.
Both forward and adjoint sensitivity analysis methods from the previous chapters encounter challenges and become less useful when applied to chaotic systems.
To illustrate this, let us consider long-time-averaged quantities 
\begin{equation}\label{eq:long_time_averaged_quantities}
    \langle L(\theta) \rangle_T = \frac{1}{T} \int_0^T L(u(t), \theta) \, dt, 
\end{equation}
of chaotic systems, where $L(u(t), \theta)$ is the instantaneous objective and $u(t)$ denotes the state of the dynamical system at time $t$.
For ergodic dynamical systems, $\lim_{T\to\infty} \langle L(\theta) \rangle_T$ depends solely on the governing dynamical system and is independent of the specific choice of trajectory $u(t)$. 
In particular, $\lim_{T\to\infty} \langle L(\theta) \rangle_T$ does not depend on the initial condition. 
Under the assumption of uniform hyperbolic systems, it is possible to derive closed-form expressions and differentiability conditions for $ \langle L(\theta) \rangle_T$\cite{ruelle1997differentiation,ruelle2009review}. 
However, computing derivatives using numerical methods of statistical quantities of the form \eqref{eq:long_time_averaged_quantities} with respect to the vector parameter $\theta$ in chaotic dynamical systems remains challenging due to the \textit{butterfly effect}, i.e. small changes in the initial state or parameter can result in large differences in a later state. 
As a consequence, the solutions of the forward and adjoint sensitivity equations blow up (exponentially fast) instead of converging to the actual derivative.
To address these issues, various modifications and methods have been proposed, including approaches based on ensemble averages~\cite{lea2000sensitivity, eyink2004ruelle}, the Fokker-Planck equation~\cite{thuburn2005climate, blonigan2014probability}, the fluctuation-dissipation theorem~\cite{leith1975climate, abramov2007blended, abramov2008new}, shadowing lemma~\cite{wang2013forward, wang2014least, wang2014convergence, ni2017sensitivity, blonigan2017adjoint, blonigan2018multiple, ni2019adjoint, ni2019sensitivity}, and modifications of Ruelle's formula~\cite{chandramoorthy2022efficient, ni2020fast}.