An easy way to derive an expression for the sensitivity $s$ is by deriving the sensitivity equations \cite{ramsay2017dynamic}, a method also referred to as continuous local sensitivity analysis (CSA). 
If we consider the original system of ODEs given by Equation \eqref{eq:original_ODE} and we differentiate with respect to $\theta$, we then obtain
\begin{equation}
    \frac{d}{d\theta} \left( \frac{du}{dt}  - f(u(\theta), \theta, t) \right) = 0.
\end{equation}
Assuming that an unique solution exists and both $\frac{\partial f}{\partial u}$ and $\frac{\partial f}{\partial \theta}$ are continuous in the neighbour of the solution; or under the guarantee of interchangeability of the derivatives \cite{gronwall1919note}, for example by assuming that both $\frac{du}{dt}$ and $\frac{du}{d\theta}$ are differentiable \cite{math8111947}, we can derive
\begin{equation}
 \frac{d}{d\theta} \frac{du}{dt} 
 =
 \frac{d}{d\theta} f(u(\theta), \theta, t)
 = 
 \frac{\partial f}{\partial \theta}
 + 
 \frac{\partial f}{\partial u} \frac{\partial u}{\partial \theta}.
\end{equation}
Identifying the sensitivity matrix $s(t)$ defined in Equation \eqref{eq:sensitivity-definition}, we obtain the \textit{sensitivity differential equation} 
\begin{equation}
 \frac{ds}{dt} = \frac{\partial f}{\partial u} s + \frac{\partial f}{\partial \theta}.
 \label{eq:sensitivity_equations}
\end{equation}
Both the original system of $n$ ODEs and the sensitivity equation of $np$ ODEs are solved simultaneously, which is necessary since the sensitivity differential equation directly depends on the value of $u(t)$.  
This implies that as we solve the ODEs, we can ensure the same level of numerical precision for the two of them inside the numerical solver.

In contrast to the methods previously introduced, the sensitivity equations find the gradient by solving a new set of continuous differential equations.
Notice also that the obtained sensitivity $s(t)$ can be evaluated at any given time $t$. 
This method can be labeled as forward, since we solve both $u(t)$ and $s(t)$ as we solve the differential equation forward in time, without the need of backtracking any operation though the solver.
By solving the sensitivity equation and the original differential equation for $u(t)$ simultaneously, we ensure that by the end of the forward step we have calculated both $u(t)$ and $s(t)$. 

% \subsubsection{Forward AD as the discretization of the sensitivity equations}

%Notice that since $s(t)$ is a matrix of size $n \times p$, the complexity of solving the sensitivity equation scales as $\mathcal O(np)$. From a computational perspective, the sensitivity $s$ appears in the sensitivity equation as a VJP, meaning the the term $\frac{\partial f}{\partial u} s$ can be efficiently computed using backwards automatic differentiation. 