An easy way to derive an expression for the sensitivity $s$ is by deriving the sensitivity equations \cite{ramsay2017dynamic}, a method also referred to as continuous local sensitivity analysis (CSA). 
If we consider the original system of ODEs given by Equation \eqref{eq:original_ODE} and we differentiate with respect to $\theta$, we then obtain
\begin{equation}
 \frac{d}{d\theta} \frac{du}{dt} 
 =
 \frac{d}{d\theta} f(u(\theta), \theta, t)
 = 
 \frac{\partial f}{\partial \theta}
 + 
 \frac{\partial f}{\partial u} \frac{\partial u}{\partial \theta},
\end{equation}
which gives the \textit{sensitivity differential equation} 
\begin{equation}
 \frac{ds}{dt} = \frac{\partial f}{\partial u} s + \frac{\partial f}{\partial \theta}.
 \label{eq:sensitivity_equations}
\end{equation}
Both the original system of $n$ ODEs and the sensitivity equation of $np$ ODEs are solved simultaneously, which is necessary since the sensitivity differential equation directly depends on the value of $u(t)$.  
This implies that as we solve the ODEs, we can ensure the same level of numerical precision for the two of them.

In opposition to the methods previously introduced, the sensitivity equations find the gradient by solving a new set of continuous differential equations.
Notice also that the obtained sensitivity $s(t)$ can be evaluated at any given time $t$. 
This method can be labeled as forward, since we solve both $u(t)$ and $s(t)$ as we solve the differential equation forward in time, without the need of backtracking any operation though the solver.
By solving the sensitivity equation at the same time we solve the original differential equation for $u(t)$, we ensure that by the end of the forward step we have calculated both $u(t)$ and $s(t)$. 

\subsubsection{Forward AD as the discretization of the sensitivity equations}

It is important to remark that the sensitivity equations can be also solved in discrete forward mode by numerically discretizing the original ODE and later deriving the discrete sensitivity equations \cite{ma2021comparison}. 
For most cases, this leads to the same result that in the continuous case \cite{FATODE2014}.
% To illustrate this, consider the simple forward Euler method applied to the original ODE given by $u_{t+1} = u_t + \Delta t \, f(u_t, \theta, t)$.
We can numerically solve for the sensitivity $s$ by extending the parameter $\theta$ to an multidimensional dual number \cite{Neuenhofen_2018, RevelsLubinPapamarkou2016}, 
\begin{equation}
    \theta =
    \begin{bmatrix}
    \theta_1 \\
    \theta_2 \\
    \vdots \\
    \theta_p
    \end{bmatrix}
    \rightarrow
    \begin{bmatrix}
    \theta_1 + \epsilon_1 \\
    \theta_2 + \epsilon_2 \\
    \vdots \\
    \theta_p + \epsilon_p
    \end{bmatrix}
\end{equation}
where $\epsilon_i \epsilon_j = 0$ for all pairs of $i$ and $j$. 
The dependency of the solution $u$ of the ODE with respect to the parameter $\theta$ is now expanded following Equation \eqref{eq:dual-number-function} as 
\begin{equation}
    u =
    \begin{bmatrix}
    u_1 \\
    u_2 \\
    \vdots \\
    u_n
    \end{bmatrix}
    \rightarrow
    \begin{bmatrix}
    u_1 + \sum_{j=1}^p \frac{\partial u_1}{\partial \theta_j} \epsilon_j \\
    u_2 + \sum_{j=1}^p \frac{\partial u_2}{\partial \theta_j} \epsilon_j \\
    \vdots \\
    u_p + \sum_{j=1}^p \frac{\partial u_n}{\partial \theta_j} \epsilon_j
    \end{bmatrix}
    = 
    u \, + \, s \, 
    \begin{bmatrix}
    \epsilon_1 \\
    \epsilon_2 \\
    \vdots \\
    \epsilon_p
    \end{bmatrix},
\end{equation}
that is, the dual component of the vector $u$ corresponds exactly to the sensitivity matrix $s$. 
This implies forward AD applied to any multistep linear solver will result in the application of the same solver on the sensitivity equations \eqref{eq:sensitivity_equations}.  
For example, for the forward Euler method this gives 
\begin{align}
    u^{t+1} + \epsilon \, s^{t+1}
    &= 
    u^t + \epsilon \, u^t + \Delta t \, f (u^t + \epsilon \, s^t, \theta + \epsilon, t) \nonumber \\
    &= 
    u^t + f(u^t, \theta, t) 
    + 
    \epsilon \, \Delta t 
    \left( 
    \frac{\partial f}{\partial u} s^t + 
    \frac{\partial f}{\partial \theta}
    \right).
    \label{eq:sensitivity-equation-AD}
\end{align}
The dual component corresponds to the forward Euler discretization of the sensitivity equation \eqref{eq:sensitivity_equations}, with $s^t$ the temporal discretization of the sensitivity $s(t)$.
For non-linear numerical solver (e.g., Runge-Kutta methods), the discretization forward AD will lead to a different numerical solver for the sensitivity.  

%Notice that since $s(t)$ is a matrix of size $n \times p$, the complexity of solving the sensitivity equation scales as $\mathcal O(np)$. From a computational perspective, the sensitivity $s$ appears in the sensitivity equation as a VJP, meaning the the term $\frac{\partial f}{\partial u} s$ can be efficiently computed using backwards automatic differentiation. 