An easy way to derive an expression for the sensitivity $s$ defined in Equation \eqref{eq:sensitivity-definition} is by deriving the forward sensitivity equations \cite{ramsay2017dynamic}, a method also referred to as continuous local sensitivity analysis (CSA). 
If we consider the original ODE given by Equation \eqref{eq:original_ODE} and we differentiate with respect to $\theta$, we then obtain
\begin{equation}
    \frac{d}{d\theta} \left( \frac{du}{dt}  - f(u(t; \theta), \theta, t) \right) = 0.
\end{equation}
Assuming that a unique solution exists and both $\frac{\partial f}{\partial u}$ and $\frac{\partial f}{\partial \theta}$ are continuous in the neighbourhood of the solution, or under the guarantee of interchangeability of the derivatives \cite{gronwall1919note}, for example by assuming that both $\frac{du}{dt}$ and $\frac{du}{d\theta}$ are differentiable \cite{math8111947}, we can derive
\begin{equation}
 \frac{d}{d\theta} \frac{du}{dt} 
 =
 \frac{d}{d\theta} f(u(t; \theta), \theta, t)
 = 
 \frac{\partial f}{\partial \theta}
 + 
 \frac{\partial f}{\partial u} \frac{\partial u}{\partial \theta}.
\end{equation}
Identifying the sensitivity matrix $s(t)$ now as a function of time, we obtain the \textit{sensitivity differential equation} 
\begin{equation}
 \frac{ds}{dt} = \frac{\partial f}{\partial u} s + \frac{\partial f}{\partial \theta}.
 \label{eq:sensitivity_equations}
\end{equation}
The initial condition is simply given by $s(t_0) = \frac{du_0}{d\theta}$, which is zero unless the initial condition explicitly depends on the parameter $\theta$.
Both the original ODE of size $n$ and the forward sensitivity equation of size $np$ are solved simultaneously, which is necessary since the forward sensitivity DE directly depends on the value of $u(t; \theta)$.  
This implies that as we solve the ODE, we can ensure the same level of numerical precision for the two of them inside the numerical solver.

In contrast to the methods previously introduced, the forward sensitivity equations find the derivative by solving a new set of continuous differential equations.
Notice also that the obtained sensitivity $s(t)$ can be evaluated at any given time $t$. 
This method can be labeled as forward, since we solve both $u(t; \theta)$ and $s(t)$ as we solve the DE forward in time, without the need of backtracking any operation though the solver.
By solving the forward sensitivity equation and the original ODE for $u(t; \theta)$ simultaneously, we ensure that by the end of the forward step we have calculated both $u(t; \theta)$ and $s(t)$. 