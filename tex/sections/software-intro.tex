% Giles (2000) has a good discussion on this.
% \cite{ma_comparison_2021}. 

In this section, we address how these different methods are computationally implemented and how to decide which method to use depending on the scientific task.
In order to address this point, it is important to make one further distinction between the methods introduced in Section \ref{section:methods}, i.e., between those that apply direct differentiation at the algorithmic level and those that are based on numerical solvers.  
The former require a much different implementation since they are agnostic with respect to the mathematical and numerical properties of the system of ODEs.
% ; however, they tend to be either inaccurate, memory-expensive, or at times unfeasible for large models. 
The latter family of methods that are based on numerical solvers include the sensitivity equations and the adjoint methods, both discrete and continuous.
% ; they are more difficult to implement and for real case applications require complex software implementations, but they are also more accurate and adequate. 
This section is then divided in two parts:
\begin{itemize}
    \item \textbf{Direct methods.} (Section \ref{section:direct-methods}) Their implementation occurs at a higher hierarchy than the numerical solver software. They include finite differences, AD, complex step differentiation.
    \item \textbf{Solver-based methods.} Their implementation occurs at the same level of the numerical solver. They include 
    \begin{itemize}
        \item Sensitivity equations (Section \ref{section:computing-sensitivity-equations})
        \item Adjoint methods, both discrete and continuous (Section \ref{section:computing-adjoints})
    \end{itemize}
\end{itemize}
While these methods can be implemented in different programming languages, here we decided to use the Julia programming language for the different examples. 
Julia is a recent but mature programming language that has already a large tradition in implementing packages aiming to advance differentiable programming \cite{Bezanson_Karpinski_Shah_Edelman_2012, Julialang_2017}, which a strong emphasis in differential equation solvers \cite{Rackauckas_Nie_2016} and sensitivity analysis \cite{rackauckas2020universal}.
Nevertheless, in reviewing existing work, we will also point to applications developed in other programming languages.

The GitHub repository \href{https://github.com/ODINN-SciML/DiffEqSensitivity-Review}{\texttt{DiffEqSensitivity-Review}} contains both text and code used to generate this manuscript. 
See Appendix \ref{appedix:code} for a complete description of the scrips provided. 
The symbol $\clubsuit$ will be use to reference code generated figures. 