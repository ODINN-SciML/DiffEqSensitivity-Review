We now move our discussion to DP methods based on numerical solvers.
These need to deal with some numerical and computational considerations, including:
% \begin{enumerate}[label=(\roman*)]
\begin{enumerate}
    \item[$ \blacktriangleright$] How to handle JVPs and/or VJPs 
    \item[$ \blacktriangleright$] Stability of the numerical solver, including the original ODE but also the sensitivity/adjoint equations
    \item[$ \blacktriangleright$] Memory-time tradeoff
\end{enumerate}
These factors are further exacerbated by the size $n$ of the ODE and the number $p$ of parameters in the model. 
% While explicit methods can be preferable for non-stiff problems, Rosenblock methods can be 
Just a few modern scientific software implementations have the capabilities of solving ODE and computing their sensitivities at the same time. 
These include 
\texttt{CVODES} within \texttt{SUNDIALS} in C \cite{serban2005cvodes, SUNDIALS-hindmarsh2005sundials}; 
\texttt{ODESSA} \cite{ODESSA} and \texttt{FATODE} (discrete adjoints) \cite{FATODE2014} both in Fortram; 
\texttt{SciMLSensitivity.jl} in Julia \cite{rackauckas2020universal}; 
\texttt{Dolfin-adjoint} based on the \texttt{FEniCS} Project \cite{dolfin2013, dolfin2018};
\texttt{DENSERKS} in Fortran \cite{alexe2007denserks}; 
\texttt{DASPKADJOINT} \cite{Cao_Li_Petzold_2002};
and \texttt{Diffrax} \cite{kidger2021on} and \texttt{torchdiffeq} \cite{torchdiffeq} in Python. 

% It is important to remark that the underlying machinery of all solvers relies on solvers for linear systems of equations, which can be solved in dense, band (sparse), and Krylov mode. 
% This implies that methods based on numerical solvers are, in principle, more difficult to implement but also more efficient in computing gradients for complex differential equations. 
% Another important consideration is that all these methods have subroutines to compute the JVPs and VJPs involved in the sensitivity and adjoint equations, respectively. 
% This calculation is carried out by another sensitivity method, usually finite differences or AD, which plays a central role when analyzing the accuracy and stability of the adjoint method. 

\subsubsection{Forward sensitivity equation}
\label{section:computing-sensitivity-equations}

For systems of equations with few number of parameters, the forward sensitivity equation is useful since the system of $n(p+1)$ equations composed by Equations \eqref{eq:original_ODE} and \eqref{eq:sensitivity_equations} can be solved using the same precision for both solution and sensitivity numerical evaluation. 
Furthermore, it does not required saving the solution in memory. 
The following example illustrates how Equation \eqref{eq:example-ode-direct-methods} and the forward sensitivity equation can be solved simultaneously using the simple explicit Euler method $\clubsuit_\text{\code{code:sensitivity-equation}}$: 
\begin{jllisting}
p = [0.2]
u0 = [0.0, 1.0]
tspan = [0.0, 10.0]

# Dynamics
function f(u, p, t)
    du₁ = u[2]
    du₂ = - p[1]^2 * u[1]
    return [du₁, du₂]
end

# Jacobian ∂f/∂p
function ∂f∂p(u, p, t)
    Jac = zeros(length(u), length(p))
    Jac[2,1] = -2*p[1]*u[1]
    return Jac
end

# Jacobian ∂f/∂u
function ∂f∂u(u, p, t)
    Jac = zeros(length(u), length(u))
    Jac[1,2] = 1
    Jac[2,1] = -p[1]^2
    return Jac
end

# Explicit Euler method
function sensitivityequation(u0, tspan, p, dt)
    u = u0
    sensitivity = zeros(length(u), length(p))
    for ti in tspan[1]:dt:tspan[2]
        sensitivity += dt * (∂f∂u(u, p, ti) * sensitivity + ∂f∂p(u, p, ti))
        u += dt * f(u, p, ti) 
    end
    return u, sensitivity   
end

u, s = sensitivityequation(u0, tspan , p, 0.001)
\end{jllisting}
The simplicity of the sensitivity method makes it available in most software for sensitivity analysis. 
In \texttt{SciMLSensitivity} in the Julia SciML ecosystem, the \texttt{ODEForwardSensitivityProblem} method implements continuous sensitivity analysis, which generates the JVPs required as part of the forward sensitivity equations via \texttt{ForwardDiff.jl} (see Section \ref{section:forwardAD-sensitivity}) or finite differences.
Using \texttt{SciMLSensitivity} reduces the code above to
\begin{jllisting}
using SciMLSensitivity

function f!(du, u, p, t)
    du[1] = u[2]
    du[2] = - p[1]^2 * u[1]
end

prob = ODEForwardSensitivityProblem(f!, u0, tspan, p)
sol = solve(prob, Tsit5())
\end{jllisting}
% Notice that in the last example we needed to re-define the out-of-place function \texttt{f} to the in-place version \texttt{f!}.

For stiff systems of ODEs the use of the forward sensitivity equations can be computationally unfeasible \cite{kim_stiff_2021}.
This is because stiff ODEs require the use of stable solvers with cubic cost with respect to the number of ODEs \cite{hairer-solving-2}, making the total complexity of the sensitivity method $\mathcal{O}(n^3p^3)$. 
This complexity makes this method expensive for models with large $n$ and/or $p$ unless the solver is able to further specialize on sparsity or properties of the linear solver (i.e. through Newton-Krylov methods). 

\paragraph{Computing JVPs and VJPs inside the solver}
\label{section:computing-vjp-inside-solver}

An important consideration is that all solver-based methods have subroutines to compute the JVPs and VJPs involved in the sensitivity and adjoint equations, respectively. 
This calculation is carried out by another sensitivity method, usually finite differences or AD, which plays a central role when analyzing the accuracy and stability of the adjoint method. 
In the case of the forward sensitivity equation, this correspond to the JVPs resulting form the product $\frac{\partial f}{\partial u} s $ in Equation \eqref{eq:sensitivity_equations}.
For the adjoint equations, we need to evaluate the term $\lambda^T \frac{\partial f}{\partial \theta}$ for the continuous adjoint method in Equation \eqref{eq:casa-final-loss-gradient}, while for the discrete adjoint method we need to compute the term $\lambda^T \frac{\partial G}{\partial \theta}$ in Equation \eqref{eq:def_adjoint} (which may further coincide with $\lambda^T \frac{\partial f}{\partial \theta}$ for some numerical solvers, but not in the general case). 
Therefore, the choice of the algorithm to compute JVPs/VJPs can have a significant impact in the overall performance \cite{Schäfer_Tarek_White_Rackauckas_2021}. 

In \texttt{SUNDIALS}, the JVPs/VJPs involved in the sensitivity and adjoint method are handled using finite differences unless specified by the user \cite{SUNDIALS-hindmarsh2005sundials}.
In \texttt{FATODE}, they can be computed with finite differences, AD, or it can be provided by the user.
In the Julia SciML ecosystem, the options \texttt{autodiff} and \texttt{autojacvec} allow one to customize if JVPs/VJPs are computed using AD, finite differences, or alternatively these are provided by the user. 
Different AD packages with different performance trade-offs are available for this task (see Section \ref{sec:software-reverse-AD}), including \texttt{ForwardDiff.jl} \cite{RevelsLubinPapamarkou2016}, \texttt{ReverseDiff.jl}, \texttt{Zygote.jl} \cite{Innes_Zygote}, \texttt{Enzyme.jl} \cite{moses_Enzyme}, \texttt{Tracker.jl}.


\subsubsection{Adjoint methods}
\label{section:computing-adjoints}

For complex and large systems (e.g. for $n + p > 50$, as we will discuss in Section \ref{section:recomendations}), direct methods for computing the gradient on top of the numerical solver can be memory expensive due to the large number of function evaluations required by the solver and the later store of the intermediate states. 
For these cases, adjoint-based methods allows us to compute the gradients of a loss function by instead computing the adjoint that serves as a bridge between the solution of the ODE and the final gradient. 
% If done well the adjoint method offers considerate advantages when working with complex systems.
Since adjoint methods rely on an additional set of ODEs which are solved, numerical efficiency and stability must further taken into account at the moment of implementing adjoint methods.

\paragraph{Discrete adjoint method}

In order to illustrate how the discrete adjoint method can be implemented, the following example shows how to manually solve for the gradient of the solution of \eqref{eq:example-ode-direct-methods} using an explicit Euler method $\clubsuit_\text{\code{code:discrete-adjoint}}$. 
\begin{jllisting}
function discrete_adjoint_method(u0, tspan, p, dt)
    u = u0
    times = tspan[1]:dt:tspan[2]

    λ = [1.0, 0.0]
    ∂L∂θ = zeros(length(p))
    u_store = [u]

    # Forward pass to compute solution
    for t in times[1:end-1]
        u += dt * f(u, p, t)
        push!(u_store, u)
    end

    # Reverse pass to compute adjoint
    for (i, t) in enumerate(reverse(times)[2:end])
        u_memory = u_store[end-i+1]
        λ += dt * ∂f∂u(u_memory, p, t)' * λ
        ∂L∂θ += dt * λ' * ∂f∂p(u_memory, p, t)
    end

    return ∂L∂θ
end

∂L∂θ = discrete_adjoint_method(u0, tspan, p, 0.001) 
\end{jllisting}
In this case, the full solution in the forward pass is stored in memory and used to compute the adjoint and integrate the loss function during the reverse pass. 
While the previous example illustrates a manual implementation of the adjoint method, the discrete adjoint method can be directly implemented using reverse AD (Section \ref{section:comparison-discrete-adjoint-AD}).
In the Julia SciML ecosystem, \texttt{ReverseDiffAdjoint} performs reverse AD on the numerical solver via \texttt{ReverseDiff.jl},  and \texttt{TrackerAdjoint} via \texttt{Tracker.jl}. 
As in the case of reverse AD, checkpointing can be used here. 


\paragraph{Continuous adjoint method}

\begin{table}[bt]
\centering
\setlength{\tabcolsep}{6pt} % Default value: 6pt
\renewcommand{\arraystretch}{1.75} % Default value: 1
\small
\begin{tabular}{ c |c c c c c|} 
 \cline{2-6}
 &\textbf{Method} & \textbf{Stability} & \textbf{Non-Stiff Performance} & \textbf{Stiff Performance} & \textbf{Memory} 
 \\ [0.5ex] 
 \cline{2-6} 
 \hline 
 \multirow{2}*{\rotatebox{90}{\scriptsize\textbf{Discrete}}} 
 &ReverseDiffAdjoint & Good & $\mathcal O (n + p)$ & $\mathcal O (n^3 + p)$ & High \\
 % &ZygoteAdjoint & Good & $\mathcal O (n + p)$ & $\mathcal O (n^3 + p)$ & High \\
 &TrackerAdjoint & Good & $\mathcal O (n + p)$ & $\mathcal O (n^3 + p)$ & High
 \\ [0.5ex] 
 \hline\hline
 \multirow{7}*{\rotatebox{90}{\scriptsize\textbf{Continuous}}} 
 & Forward sensitivity eq. & Good & $\mathcal O (np)$ & $\mathcal O (n^3p^3)$ & $\mathcal O(1)$ \\
 \cline{2-6}
 &Backsolve adjoint & Poor & $\mathcal O (n + p)$ & $\mathcal O ((n+p)^3)$ & $\mathcal O(1)$ \\ 
 &Backsolve adjoint$^\blacktriangleleft$ & Medium & $\mathcal O (n + p)$ & $\mathcal O ((n+p)^3)$ & $\mathcal O (nK)$ \\
 &Interpolating adjoint & Good & $\mathcal O (n + p)$ & $\mathcal O ((n+p)^3)$ & High \\ 
 &Interpolating adjoint$^\blacktriangleleft$ & Good & $\mathcal O (n + p)$ & $\mathcal O ((n+p)^3)$ & $\mathcal O (nK)$ \\
 &Quadrature adjoint & Good & $\mathcal O (n + p)$ & $\mathcal O (n^3 + p)$ & High \\
 &Gauss adjoint & Good & $\mathcal O (n + p)$ & $\mathcal O (n^3 + p)$ & High \\
 &Gauss adjoint$^\blacktriangleleft$ & Good & $\mathcal O (n + p)$ & $\mathcal O (n^3 + p)$ & $\mathcal O(nK)$ \\
 % name & .. & .. & .. \\
 \hline
\end{tabular}
% \vspace
\caption{Comparison in performance and cost of solver-based methods. Methods that can be checkpointed are indicated with the symbol $\blacktriangleleft$, with $K$ the total number of checkpoints. The nomenclature of the different adjoint methods here follows the naming in the documentation of \texttt{SciMLSensitivity.jl} \cite{rackauckas2020universal}.}
\label{table:adjoint}
\end{table}

The continuous adjoint methods offers a series of advantages over the discrete method and the rest of the forward methods previously discussed. 
Just as the discrete adjoint methods and reverse AD, the bottleneck is how to solve for the adjoint $\lambda(t)$ due to its dependency with VJPs involving the state $u(t)$.
Effectively, notice that Equation \eqref{eq:casa-adjoint-equation} involves the terms $f(u, \theta, t)$ and $\frac{\partial h}{\partial u}$, which are both functions of $u(t)$. 
However, in contrast to the discrete adjoint methods, here the full continuous trajectory $u(t)$ is needed, instead of its discrete pointwise evaluation. 
There are two solutions for addressing the evaluation of $u(t)$ during the computation of $\lambda (t)$:
% \begin{enumerate}[label=(\roman*)]
\begin{enumerate}
    \item[$ \blacktriangleright$] \textbf{Interpolation.} During the forward model, we can store in memory intermediate states of the numerical solution allowing the dense evaluation of the numerical solution at any given time. 
    % This leads to heavy-memory expensive algorithms. 
    % However, in opposition to discrete methods, solving for the adjoint in reverse mode requires to be able to evaluate the solution $u(t)$ at any given time $t$. 
    This can be done using dense output formulas, for example by adding extra stages to the Runge-Kutta scheme that allows to define a continuous interpolation, a method known as continuous Runge-Kutta \cite{hairer-solving-2, Alexe_Sandu_2009}. 
    % When using checkpointing, intermediate variables are saved and the interpolation between them is re-computed on demand. 
    \item[$ \blacktriangleright$] \textbf{Backsolve.} Solve again the original ODE together with the adjoint as the solution of the following reversed augmented system \cite{chen_neural_2019}:
    \begin{equation}
    \frac{d}{dt}
    \begin{bmatrix}
       u \\
       \lambda \\
       \frac{dL}{d\theta}
    \end{bmatrix}
    = 
    \begin{bmatrix}
       -f \\
       - \frac{\partial f}{\partial u}^T \lambda - \frac{\partial h}{\partial u}^T \\
       - \lambda^T \frac{\partial f}{\partial \theta} - \frac{\partial h}{\partial \theta}
    \end{bmatrix}
    % = 
    % - [ 1, \lambda^T, \lambda^T ]
    % \begin{bmatrix}
    %    f & \frac{\partial f}{\partial u} & \frac{\partial f}{\partial \theta} \\
    %    0 & 0 & 0 \\
    %    0 & 0 & 0
    % \end{bmatrix},
    \qquad 
    \begin{bmatrix}
       u \\
       \lambda \\
       \frac{dL}{d\theta}
    \end{bmatrix}(t_1)
    = 
    \begin{bmatrix}
       u(t_1) \\
       \frac{\partial L}{\partial u(t_1)} \\
       \lambda(t_0)^T s(t_0)
    \end{bmatrix}.
    \end{equation}
    An important problem with this approach is that computing the ODE backwards $\frac{du}{dt} = - f(u,\theta, t)$ can be unstable and lead to large numerical errors \cite{kim_stiff_2021, Zhuang_2020}. 
    Implicit methods may be used to ensure stability when solving this system of equations. 
    However, this requires cubic time in the total number of ordinary differential equations, leading to a total complexity of $\mathcal O((n+p)^3)$ for the adjoint method. In practice, this method is hardly stable for most complex (even non-stiff) differential equations. 
\end{enumerate} 

The following example shows how to implement the continuous adjoint method of the solution of Equation \eqref{eq:example-ode-direct-methods} using the backsolve strategy $\clubsuit_\text{\code{code:continuous-adjoint}}$. 
\begin{jllisting}
using RecursiveArrayTools

# Augmented dynamics
function f_aug(z, p, t)
    u, λ, L = z
    du = f(u, p, t)
    dλ = ∂f∂u(u, p, t)' * λ
    dL = λ' * ∂f∂p(u, p, t)
    VectorOfArray([du, vec(dλ), vec(dL)])
end

# Solution of original ODE
prob = ODEProblem(f, u0, tspan, p)
sol = solve(prob, Euler(), dt=0.001)

# Final state 
u1 = sol.u[end]
z1 = VectorOfArray([u1, [1.0, 0.0], zeros(length(p))])

aug_prob = ODEProblem(f_aug, z1, reverse(tspan), p)
u0_, λ0, dLdp_cont = solve(aug_prob, Euler(), dt=-0.001).u[end]
\end{jllisting}
Notice that here we used the final state of the solution $u$ of the ODE as starting point, which is then recalculated in backwards direction (implemented via a negative stepsize \texttt{dt=-0.001}). 

When dealing with stiff DEs, special considerations need to be taken into account.
Two alternatives are proposed in \cite{kim_stiff_2021}, the first referred to as \textit{Quadrature Adjoint} produces a high order interpolation of the solution $u(t)$, then solve for $\lambda$ in reverse using an implicit solver and finally integrating $\frac{dL}{d\theta}$ in a forward step.
This reduces the complexity to $\mathcal O (n^3 + p)$, where the cubic cost in the size $n$ of the ODE comes from the fact that we still need to solve the original stiff ODE in the forward step. 
A second similar approach is to use an implicit-explicit (IMEX) solver, where we use the implicit part for the original equation and the explicit for the adjoint. 
This method also has a complexity of $\mathcal O (n^3 + p)$. 

\paragraph{Continuous checkpointing}
\label{section:checkpointint-cont}

Both interpolating and backsolve adjoint methods can be implemented along with a checkpointing scheme. 
This can be done by choosing saved points in the forward pass. 
For the interpolating methods, the interpolation is reconstructed in the backwards pass between two save points. 
This reduces the total memory requirement of the interpolating method to simply the maximum cost of holding an interpolation between two save points, but requires a total additional computational effort equal to one additional forward pass. 
In the backsolve variation, the value $u$ in the reverse pass can be corrected to be the saved point, thus resetting the numerical error introduced during the backwards evaluation and thus improving the accuracy.

\paragraph{Solving the quadrature}

Another computational consideration is how the integral in Equation \eqref{eq:casa-final-loss-gradient} is numerically evaluated. 
While one can solve the integral simultaneously with the other equations using an ODE solver, this is only recommended with explicit methods as with implicit methods these additional ODE is of size $p$ and thus can increase the complexity of an implicit solve by $O(p^3)$. 
The interpolating adjoint and backsolve adjoint method use this ODE solver approach for computing the integrand. 
On the other side, the quadrature adjoint approach avoids this computational cost by computing the dense solution $\lambda(t)$ and then computing the quadrature 
\begin{equation}
    \inttime
    \left( \frac{\partial h}{\partial \theta} + \lambda^T \frac{\partial f}{\partial \theta} \right) dt 
    \approx
    \sum_{j=1}^N \omega_j \left( \frac{\partial h}{\partial \theta} + \lambda^T \frac{\partial f}{\partial \theta} \right) (\tau_i)
\end{equation}
where $\omega_i$, $\tau_i$ are the weights and knots of a Gauss-Kronrod quadrature method for numerical integration from \texttt{QuadGK.jl} \cite{laurie1997calculation, gonnet2012review}. 
This method results in global error control on the integration and removes the cubic scaling within implicit solvers. Nonetheless, it requires a larger memory cost by storing the adjoint pass continuous solution.

Solvers designed for large implicit systems allow for solving explicit integrals based on the ODE solution simultaneously without including the equations in the ODE evaluation in order to avoid this expense. 
The Sundials CVODE solver introduced this technique specifically for BDF methods \cite{SUNDIALS-hindmarsh2005sundials}. 
In the Julia \texttt{DifferentialEquations.jl} solvers, this can be done using a callback (specifically the numerical integration callbacks form the \texttt{DiffEqCallbacks.jl} library). 
The Gauss adjoint method uses the callback approach to allow for a simultaneous explicit evaluation the integral using Gaussian quadrature, similar to \cite{Norcliffe_gaussquadrature_2023} but using a different approximation to improve convergence.
\\ \\ 
These differences in the strategies for computing $u(t)$ and the final quadrature give rise to the set of methods in Table \ref{table:adjoint}. 
Excluding the forward sensitivity equation which was added for reference, all of these are \textit{the adjoint method} with differences being in the way steps of the adjoint method are approximated, and notably Table \ref{table:adjoint} shows a general trade-off in stability, performance, and memory across the methods. 
While the Gauss adjoint achieves good properties according to this chart, the quadrature adjoint notably uses a global error control of the quadrature as opposed to the local error control of the Gauss adjoint, and thus can achieve more robust bounding of the error with respect to user chosen tolerances.

