In symbolic differentiation, functions are represented algebraically instead of algorithmically, which is why many symbolic differentiation tools are included inside computer algebra systems (CAS) \cite{Symbolics_jl_2022}. 
Instead of numerically evaluating the final value of a derivative, symbolic systems assign variable names, expressions, operations, and literals to \textit{algebraic} objects. 
For example, the relation $y = x^2$ is interpreted as expression with two variables, $x$ and $y$, and the symbolic system generates the derivative $y' = 2 \times x$ with $2$ a numeric literal, $\times$ a binary operation, and $x$ the same variable assignment as in the original expression.

The general issue with symbolic differentiation is \textit{expression swell}, i.e. the fact that the size of a derivative expression can be much larger than the original expression \cite{Baydin_Pearlmutter_Radul_Siskind_2015}. 
One way to visualize this swell is to note that the product rule grows and expression of $f(x)g(x)$ into two expressions, namely $\frac{d}{dx}(f(x)g(x)) = \frac{df}{dx}g(x) + f(x)\frac{dg}{dx}$, and thus the composition of many functions leads to a large derivative expression. 
AD avoids expression swell by instead numerically calculating the derivative of a given expression at some fixed value, never representing the general derivative but only at the values obtained by the forward pass. 
This eager evaluation of the derivative around a given value forces the intermediate computation into the JVPs or VJPs form as a way to continually pass forward/reverse the current state. 
Meanwhile, symbolic differentiation can represent the complete derivative expression and thus avoid being forced into a given computation order, but at the memory cost of having to represent larger expressions.
% Simplification routines implemented in CAS may however reduce the size and complexity of algebraic expressions by finding common sub-expressions, making symbolic differentiation very efficient when computing derivatives multiple times and for different input values \cite{Dürrbaum_Klier_Hahn_2002}. 

However, it is important to acknowledge the close relationship between AD and symbolic differentiation.
AD uses symbolic differentiation in its definition of primitives which are then chained together in a specific way to form VJPs and vector products. 
Forward AD can be expressed as a form of symbolic differentiation with a specific choice of common subexpression elimination, i.e. forward AD can be expressed as a symbolic differentiation with a specific choice of how to accumulate the intermediate calculations so that expression growth can be avoided \cite{juedes1991taxonomy, Elliott_2018, Laue2020, Dürrbaum_Klier_Hahn_2002}.
However, general symbolic differentiation can have many other choices for the differentiation order, and does not in general require computation using the JVPs or VJPs \cite{Baydin_Pearlmutter_Radul_Siskind_2015}. 
This is apparent for example when computing sparse Jacobians, where generally symbolic differentiation computes entries element-by-element while forward AD computes the matrix column-by-column and reverse AD computes row-by-row (see Section \ref{section:sparsity}).