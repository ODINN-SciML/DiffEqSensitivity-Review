In symbolic differentiation, functions are represented algebraically instead of algorithmically, reason why many symbolic differentiation tools are included inside computer algebra systems (CAS)\cite{Symbolics_jl_2022}. 
Instead of numerically evaluating the final value of a derivative, symbolic systems define \textit{algebraic} objects, including variable names, expressions, operations, and literals. 
For example, the relation $y = x^2$ is interpreted as expression with two variables, $x$ and $y$, and the symbolic system need to generate the derivative $y' = 2 \times x$ with $2$ a numeric literal, $\times$ a binary operation, and $x$ the same variable assignment than in the original expression.
When the function to differentiate is large, symbolic differentiation can lead to \textit{expression swell}, that is, exponentially large or complex symbolic expressions \cite{Baydin_Pearlmutter_Radul_Siskind_2015}.
Here, an important piece of CAS is simplification routines that reduce the size and complexity of algebraic expressions by finding common sub-expressions.  
This can make symbolic differentiation very efficient when computing derivatives multiple times and for different input values \cite{Dürrbaum_Klier_Hahn_2002}. 

It is important to remark on the close relationship between AD and symbolic differentiation.
There is no agreement as to whether symbolic differentiation should be classified as AD\cite{juedes1991taxonomy, Elliott_2018, Laue2020} or as a different method \cite{Baydin_Pearlmutter_Radul_Siskind_2015}.  
Both are equivalent in the sense that they perform the same operations but the underlying data structure is different \cite{Laue2020}. 
Here, expression swell is a consequence of the underlying representation when this does not allow for common sub-expressions. 
This can also be understood as if AD is symbolic differentiation performed by a compiler \cite{Elliott_2018}, meaning that different AD can be classified based on the level of integration with the underlying source language \cite{juedes1991taxonomy}.
% This means that the mathematical calculations involved are the same, but they need to be interpreted by the compiler at the moment of computing the derivative. 
% Something like $x + 2$ needs to be understood as an \textit{expression} composed by a variable $x$, a literal $2$, and a binary operation $+$ binding them. 

\subsubsection{Sparse Symbolic Differentiation vs Sparse Colored AD}

Sparse Jacobians are commonplace in several large-scale nonlinear systems, discretized PDEs, etc., and are often a major computational bottleneck for solving those problems. We will consider 2 possibilities for computing these sparse Jacobians -- Symbolic Differentiation and Colored AD. Consider a toy example, with a Jacobian with known sparsity pattern $\mathcal{J}_{\text{sparse}}$:
%
\begin{equation}
    \mathcal{J}_{\text{sparse}} = \begin{bmatrix}
        \bullet &         &         &         &         \\
                & \bullet & \bullet &         &         \\
                &         &         & \bullet &         \\
        \bullet & \bullet &         &         & \bullet \\
                &         &         &         & \bullet
    \end{bmatrix}
\end{equation}
%
where $\bullet$ denotes the non-zero elements of the Jacobian. AD tools compute Jacobians column-wise or row-wise by composing multiple JVPs or VJPs respectively. This is done to avoid perturbation confusion~\cite{manzyuk2019perturbation}. Sparse Colored AD can chunk multiple JVPs or VJPs using the colored Jacobian~\cite{gebremedhin2005color}. More concretely, we can color the above matrix as follows:
%
\begin{equation}
    \mathcal{J}^{(\text{col})}_{\text{sparse}} = \begin{bmatrix}
        \color{red}{\blacktriangleright} &                            &                                  &                                  &                              \\
                                         & \color{blue}{\blacksquare} & \color{red}{\blacktriangleright} &                                  &                              \\
                                         &                            &                                  & \color{red}{\blacktriangleright} &                              \\
        \color{red}{\blacktriangleright} & \color{blue}{\blacksquare} &                                  &                                  & \color{green}{\blacklozenge} \\
                                         &                            &                                  &                                  & \color{green}{\blacklozenge}
    \end{bmatrix} \qquad \mathcal{J}^{(\text{row})}_{\text{sparse}} = \begin{bmatrix}
        \color{blue}{\blacksquare}   &                              &                            &                            &                              \\
                                     & \color{blue}{\blacksquare}   & \color{blue}{\blacksquare} &                            &                              \\
                                     &                              &                            & \color{blue}{\blacksquare} &                              \\
        \color{green}{\blacklozenge} & \color{green}{\blacklozenge} &                            &                            & \color{green}{\blacklozenge} \\
                                     &                              &                            &                            & \color{blue}{\blacksquare}
    \end{bmatrix}
\end{equation}
%
To compute $\mathcal{J}^{(\text{col})}_{\text{sparse}}$, we need to perform 3 JVPs (once each for $\color{red}{\blacktriangleright}$, $\color{blue}{\blacksquare}$, and $\color{green}{\blacklozenge}$) compared to 5 JVPs for a $5 \times 5$ dense Jacobian. Similarly, since reverse mode materializes the Jacobian one row at a time, we need 2 VJPs (once each for $\color{blue}{\blacksquare}$, and $\color{green}{\blacklozenge}$) compared to 5 VJPs for the dense counterpart. However, colored AD has the limitation that an extremely sparse matrix can have no rows or columns that are independent of each other. Consider the arrowhead matrix:
%
\begin{equation}
    \mathcal{J}_{\text{arrowhead}} = \begin{bmatrix}
        \bullet & \bullet & \bullet & \bullet & \bullet \\
        \bullet & \bullet &         &         &         \\
        \bullet &         & \bullet &         &         \\
        \bullet &         &         & \bullet &         \\
        \bullet &         &         &         & \bullet
    \end{bmatrix}
\end{equation}
%
In this case, both reverse mode AD and forward mode AD will have to perform $N$ VJPs and JVPs, respectively (for a $N \times N$ arrowhead matrix), i.e., there is no computational benefit of coloring the matrix here. Instead, Symbolic Differentiation constructs a symbolic representation of the Sparse Jacobian and can fill the Jacobian with $N + 2 \cdot (N - 1)$ computations, where each computation is significantly cheaper than each VJP or JVP.
